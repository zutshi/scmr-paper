In this section, we present the model of the hybrid dynamical system
under test, and the background for PWA relational abstractions.


\subsection{System-under-test ($\System$)}

We assume that the system-under-test is a hybrid dynamical system, \ie, a
system which has discrete modes and a continuous state-space, and which
evolution is either (1) continuous in time according to the dynamics
associated with its current discrete mode (typically described using
differential equations over its continuous state), or (2) through
discrete jumps that possibly change the discrete mode of the system and
possibly reset its continuous state to some value.  However, we assume
that the system is {\em effectively black-box}. This means that we do not
assume any knowledge of the symbolic dynamical equations describing the
system behavior. We do assume some knowledge of the system: an {\em
interface} view, in which we are allowed to stimulate the system with
inputs and observe its outputs.

We formally define a system-under-test $\System$ as a tuple $(\Modes,
\ContStates, \InitModes, \InitStates, \Delta, \Inputs, \Outputs)$.
Here, system state-space (denoted $\HybridStates$) is a subset of
$\Modes \times \ContStates$, where $\Modes$ is a finite set of
discrete modes, and $\ContStates$ is a compact subset of $\Reals^n$
representing the set of continuous states.  The sets $\InitModes
\subseteq \Modes$ and $\InitStates \subseteq \ContStates$ represent
initial modes and states respectively.  We assume that the system is
{\em initialized} to some initial mode $\initmode \in \InitModes$ and
an initial state $\initstate \in \InitStates$ at time $t=0$.  We
assume that the system has $m$ exogenous inputs which take values
from the set $\Inputs$ which is a compact subset of $\Reals^m$. We
assume that the system has $\ell \le n$ outputs, where $\Outputs
\subseteq \HybridStates$. In this paper, we assume that the entire
system state is observable.  In other words, $\Outputs =
\HybridStates$.

Finally, we restrict our attention to switched-mode systems, \ie, those in
which there is a partition imposed on $\ContStates$, and there is a
bijective map between each element $\ContStates_i$ in the partition and
each mode in $\Modes$. Thus, we can omit the discrete modes from the
system description, and set $\HybridStates$ = $\ContStates$.

The operational semantics of $\System$ is defined in terms of a {\em
forward simulation map}\footnote{We require that $\System$ can always
be simulated forward in time with deterministic results. A few
technical conditions are required on the system dynamics are required
to ensure that this assumption is valid, see for
example~\cite{Meiss/2007/Differential} and \cite{goebel2012hybrid} for
further details. In practice, such conditions are usually impossible
to check for a black-box model. We then assume that the black-box
model is described as a model in a deterministic simulation framework
such as Simulink\textregistered.} $\simmap_\Delta$, which is a
function that maps a time $t$, the current state of the system, and an
input value in $\Inputs$ to the state of the system at time
$t+\Delta$, \ie, $\simmap_\Delta$ is a function from $\ContStates
\times \Inputs \times \Reals^{\ge 0}$ to $\ContStates$.

\subsection{$\epsilon$-precision Transition System Abstraction}

We now show how we can abstract a hybrid dynamical system $\System$
which has trajectories in dense-time by a discrete transition system.
A discrete transition system $\mathcal{T}$ can be defined as the tuple
$(\Cells, \CellEdges, \InitCells, \CellLabelingFunction)$, where
$\Cells$ is a finite set of cells, $\CellEdges \subseteq \Cells \times
\Cells$ is an unlabeled transition relation, and
$\CellLabelingFunction$ is a function that assigns one or more labels
to the cells from a finite set $L$ of labels. The operational
semantics of the transition system are as follows: the system starts
in some cell in $\Cells_\init$. At each step, if the transition system
is in cell $C$, it transitions to cell $C'$ if $(C,C') \in
\CellEdges$. A finite-time trace property $\varphi$ of the system is
some subset of the free monoid $(2^L)^*$. We say that $\mathcal{T}
\not\models \varphi$ if there exists some trace of $\mathcal{T}$ such
that the sequence of labels in that trace does not belong to
$\varphi$.

We now give an example of a specific kind of transition system
abstraction that we can build given $\System$ and the simulation map
$\simmap_\Delta$.  We call the transition system an
$\epsilon$-precision transition system ($\epsilon$-TS for short) and
denote it by $\TSAbstraction{\epsilon}{\System}$.  Let $\epsilon  =
10^{-k}$, for some positive integer $k$.  We define a quantization
function $\quant_\epsilon$ that maps a state $\contState$ in
$\ContStates$ to $10^{-k}*\lfloor \contState*10^k \rfloor$. This
transformation essentially truncates the decimal representation of the
numeric value of the state $\state$ to at most $k$ digits after the
decimal point.  This function induces an equivalence relation
$\contState_i \equiv_\epsilon \contState_j$ iff
$\quant_\epsilon(\contState_i) = \quant_\epsilon(\contState_j)$.

Using the equivalence relation $\equiv_\epsilon$, we can now define
the tuple $(\Cells, \CellEdges, \CellLabelingFunction)$ for
$\TSAbstraction{\epsilon}{\System}$ by considering the quotient
structure $\ContStates/\equiv_\epsilon$.  Essentially, each
equivalence class in the quotient structure is a cell in $\Cells$.
For cells $C$ and $C'$ in $\TSAbstraction{\epsilon}{\System}$, the
edge $(C,C')$ can be an element of $\CellEdges$ iff:
$\simmap_\Delta(\contState, u, t)$ = $\contState'$,
$\quant_\epsilon(\contState)$ = $C$, and,
$\quant_\epsilon(\contState') = C'$.

We assume that there is a specified set of states $\UnsafeStates
\subseteq \ContStates$ that are unsafe\footnote{Our framework is
general enough to reason about richer class of correctness properties
for trajectories such as bounded-time LTL properties. In this case, we
assume the LTL property is defined over Boolean-valued predicates over
$\ContStates$. The labeling function $\CellLabelingFunction$ would
then label cells with the predicates that evaluate to $\mathit{true}$
for some state in the cell.}, and thus restrict our attention to
simple safety properties. Thus, our label set has only two elements:
$\setof{\init,\unsafe}$. We then define $\CellLabelingFunction$ as
$\CellLabelingFunction(C) = \init$ if $C$ contains some initial state,
$\CellLabelingFunction(C) = \unsafe$ if $C$ contains some unsafe
state, and empty otherwise.

In what follows, we review the procedure
$\mathsf{ScatterAndSimulate}$($\System$,$\simmap_\Delta$,$\epsilon$)
from \cite{zutshi2014multiple} that, given the $\System$, the forward
simulation map $\simmap_\Delta$, and a precision $\epsilon$ creates an
$\epsilon$-TS abstraction of $\System$.  The
essence of the procedure is a breadth-first exploration of
$\ContStates$ to build $\TSAbstraction{\epsilon}{\System}$ =
$(\Cells,\CellEdges,\CellLabelingFunction)$. It consists of the
following steps:
\begin{enumerate}[leftmargin=1em,labelsep=1em,label={\arabic*.}]
\item
Add a set of random samples from the initial set of states to a queue
(\ie some states $\contState$ at time $0$).
\item
Pop the head of the queue (say state $\contState$ at time $t$), identify
the corresponding cell by applying $\quant_\epsilon$, and add it to
$\Cells$.
\item
Obtain $\contState' = \simmap_\Delta(\contState, t, u)$. Obtain a a
fixed number of states in the neighborhood of $\contState'$, and add
$\contState'$ and these states (all time $t+\Delta$) to the queue.
\item
Add an edge between the cells corresponding to $\contState$ and
$\contState'$.
\item
Go to step 2 till either the queue is empty or a fixed number of cells
or edges is added to $\TSAbstraction{\epsilon}{\System}$.
\end{enumerate}

% \inclfig[width=0.95\linewidth]{vdp_cont_traces.pdf}{Van der
% Pol: continuous trajectories. Red and green boxes indicate unsafe and
% initial sets.}{vdp-cont}
\inclfig[width=0.95\linewidth]{vdp_abs.pdf}{The discovered
$0.2$-TS abstraction. Red cells are unsafe cells and green cells are
initial cells.}{vdp-abs}
\begin{example}[Van der Pol Oscillator]~\label{ex:vdp}
We now describe $\epsilon$-TS abstraction using the Van der Pol
    oscillator benchmark from~\cite{zutshi2014multiple}. Simulations
    generated using uniformly random samples are shown in
    \figref{vdp-abs}. We want to check the property $P_3$:
    $x_{unsafe}\in[-1,-6.5]$ and $y_{unsafe}\in[-0.7, -5.6]$, given
    the initial set indicated $x0\in[-0.4, -0.4],\; y_0\in[0.4, 0.4]$.
    We consider the $\epsilon$-TS abstraction defined by the
    quantization function $\quant_{0.2}(\x)$. This results in an
    evenly gridded state-space, where each cell is of size $0.2 \times
    0.2$ units.  $\mathsf{ScatterAndSimulate}$ is then used to
    construct $\TSAbstraction{0.2}{\System}$ using $2$ samples per
    cell and a time step of $\Delta = 0.1s$. The complete process
    follows.  The $\epsilon$-TS abstraction thus discovered is shown
    in \figref{vdp-abs}. The red cells are unsafe and green cells are
    initial cells.
\end{example}


\subsection{CEGAR for falsifying $\TSAbstraction{\epsilon}{\System}$}

We remark that our abstraction is {\em neither an over-approximation
nor an under-approximation} of the continuous-time continuous-space
reachability relation between states. The purpose of the $\epsilon$-TS
is to discover falsifying trajectories. This can be accomplished by
Algorithm~\ref{algo:s3camBMC} that uses CounterExample-Guided
Abstraction and Refinement (CEGAR).

Here, we construct $\TSAbstraction{\epsilon}{\System}$
(Line~\ref{algoline:construct}), we check if there is a path in
$\TSAbstraction{\epsilon}{\System}$ that leads to a cell labeled
unsafe. This can be done by a bounded model checking procedure
(Line~\ref{algoline:bmc}) that explores at most $\ell$ transitions of
$\TSAbstraction{\epsilon}{\System}$ (\ie, traces that are at most
$\ell\Delta$ seconds in length). The model checker may return a set of
counterexample trajectories $\Pi$. If this set is empty
(Line~\ref{algoline:nocex}), then either the exploration performed by
$\mathsf{ScatterAndSimulate}$ is not enough or the system could be
safe. In the former case, the user can increase the scope of
exploration for $\mathsf{ScatterAndSimulate}$.  The next step is to
check if the abstract trajectories are spurious, by performing a
concretization step by performing an actual simulation of the system
(Line~\ref{algoline:checkspurious}). If all abstract counterexample
trajectories are found to be spurious, then
$\TSAbstraction{\epsilon}{\System}$ can be refined, basically by
increasing $\epsilon$.
\begin{algorithm}[t]
\DontPrintSemicolon
\caption{CEGAR for $\TSAbstraction{\epsilon}{\System}$\label{algo:s3camBMC}}
\KwIn{$\System$, $\simmap_\Delta$, $\epsilon_0$ , Smallest Precision $\epsilonmin$}
$\epsilon \assign \epsilon_0$ \;
\While{$\epsilon \ge \epsilonmin$}{
    $\TSAbstraction{\epsilon}{\System}$ $\assign$
    $\mathsf{ScatterAndSimulate}$($\System$,$\simmap_\Delta$,$\epsilon$)
    \nllabel{algoline:construct} \;
    $\Pi$ $\assign$ $\mathsf{BoundedModelCheck}$($\TSAbstraction{\epsilon}{\System}$)
    \nllabel{algoline:bmc} \;
    \lIf {$\Pi = \emptyset$}{ return FAIL \nllabel{algoline:nocex} }
    \lIf {$\mathsf{checkIfConcretizable}(\Pi)$}{ return VIOLATION \nllabel{algoline:checkspurious} }
    $\epsilon$ $\assign$ $\frac{\epsilon}{10}$ \nllabel{algoline:refine} \;
}
\end{algorithm}

The biggest drawback of Algorithm~\ref{algo:s3camBMC} is that with
increasing $\epsilon$, the size of $\Cells$ in
$\TSAbstraction{\epsilon}{\System}$ increases exponentially.  Thus,
every subsequent run of the falsification algorithm is slower. In the
next section, we propose an alternative abstraction that enriches the
basic abstraction proposed here by adding labels to the transitions.
We can then perform refinement by enriching the transition relation,
rather than increasing $\epsilon$.

% \subsection{Example: Van der Pol Oscillator}
% \todo{Takes 4 iterations}
% \inclfig[width=0.90\linewidth]{vdp_disc_map.pdf}{Refinement of
% $\epsilon$-TS.}{vdp-map}
%
\begin{example}
\inclfig[width=0.90\linewidth]{vdp_disc_map.pdf}{
Abstract trajectories discovered by $ScatterAndSimulate$. Each
    rectangle denotes a cell which is uniquely identified  by the
    tuple against it.
}{vdp-abs-paths}

Going back to the~\exref{vdp}, $ScatterAndSimulate$ is able to find
    several abstract trajectories as shown in~\figref{vdp-abs-paths}.
    However, typically, the refinement by increasing the quantization
    precision requires $4$ refinement iterations and finally finds a
    concrete counter-example for $\epsilon=0.125$.

\end{example}
