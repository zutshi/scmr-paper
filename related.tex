\mypara{Automatic falsification techniques for hybrid systems} bridge
the gap between the manual (and random) testing approaches and
formal verification approaches.

The current state-of-the-art approaches (Cf. related
work~\cite{nghiem2010monte}) for testing hybrid systems are based on
numerical simulations. These can be further classified into robustness
guided testing: S-Taliro~\cite{annpureddy2011s} and
Breach~\cite{donze2010breach}, and RRT (Rapidly-exploring Random Tree)
based:~\cite{nahhal_test_2007,Dang09,dreossi2015efficient}. An
alternative approach based on multiple shooting and CEGAR
(Counterexample Guided Abstraction Refinement), was proposed by the
authors in~\cite{zutshi2014multiple}. Unlike the previous approaches,
which directly use numerical simulations, it used the computed
trajectories to explore an implicit state-space abstraction and guided the
search towards abstract counter-examples. In this paper, we add
another step which identifies the affine dynamics associated with each
relation of the abstraction. This is akin to identifying a pwa model
from simulations.

\mypara{Formal analysis of pwa systems.} Discrete time, continuous
state piece-wise affine models are often used to model both continuous
and hybrid dynamical systems. They are very expressive, and can model
non-linear continuous dynamics with arbitrary
accuracy~\cite{wen2008basis} and a large family of hybrid
systems~\cite{heemels2001equivalence}. Several approaches have been
proposed formal analysis of pwa systems,
including~\cite{batt2007model, yordanov2007model, yordanov2010formal,
koutsoukos2003safety, asarin2000approximate}

\mypara{Identification of hybrid systems.} Several approaches have been put
forth for the identification of PWA models~\cite{paoletti2007identification}.
They are based on Bayesian methods~\cite{juloski2005bayesian}, bounded-error
methods~\cite{bemporad2003greedy,bemporad2005bounded,roll2004identification},
clustering based methods~\cite{ferrari2003clustering} and algebraic
methods~\cite{vidal2003algebraic}. Unlike them, we peopose a simpler approach
based on fixed hyper-rectangular domains~\cite{billings1987piecewise}, but
parameterized by the size of the rectangles ($\epsilon$) and the time
discretization step $\Delta$.

% Alternatively, verification approaches have also been used, where a
% formal model (usually hybrid automata) is first constructed, often
% manually and then exhaustively checked for violations. By restricting
% ourselves to falsification

%Model checking of pwa systems.



%\item Fixed time caveats
%\item Effects of refining parameters: time step $\Delta$ and grid size $\epsilon$.

%\mypara{Model Checking PWA Models.} Model checking transition systems
