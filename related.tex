Discrete time and continuous state Piece-Wise Affine (PWA) models are
quite popular in the modeling of both continuous and hybrid dynamical
systems. They are very expressive, and can model non-linear continuous
dynamics with arbitrary accuracy~\cite{wen2008basis} and a large
family of hybrid systems~\cite{heemels2001equivalence}.

\mypara{Falsification of properties of hybrid systems.} The current
state-of-the-art falsification approaches are based on numerical
simulations and can be grouped into robustness guided
S-Taliro~\cite{annpureddy2011s}, Breach~\cite{donze2010breach} and RRT
(Rapidly-exploring Random Tree)
based~\cite{nahhal_test_2007,Dang09,dreossi2015efficient}.  An
alternative approach, based on multiple shooting and CEGAR
(Counterexample Guided Abstraction Refinement) was proposed by the
authors in ~\cite{zutshi2014multiple}.  The alternative approach is
the verification approach, where a formal model (usually hybrid
automata) is first constructed (often manually) and then exhaustively
checked for violations.

\mypara{Identification of hybrid systems.} Several approaches have been put
forth for the identification of PWA models~\cite{paoletti2007identification}.
They are based on Bayesian methods~\cite{juloski2005bayesian}, bounded-error
methods~\cite{bemporad2003greedy,bemporad2005bounded,roll2004identification},
clustering based methods~\cite{ferrari2003clustering} and lastly algebraic
methods~\cite{vidal2003algebraic}. Unlike them, we peopose a simpler
approach based on fixed hyper-rectangular domains, but parameterized
by the size of the rectangles ($\epsilon$) and the time discretization
step $\Delta$.


%\item Fixed time caveats
%\item Effects of refining parameters: time step $\Delta$ and grid size $\epsilon$.

%\mypara{Model Checking PWA Models.} Model checking transition systems
