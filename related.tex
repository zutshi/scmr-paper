Piece-Wise Affine (PWA) models are quite popular in the modeling of
both continuous and hybrid dynamical systems. They are very
expressive, and can model non-linear continuous dynamics with
arbitrary accuracy~\cite{wen2008basis} and a large family of hybrid
systems~\cite{heemels2001equivalence}.

\mypara{Falsification of Properties of Hybrid Systems.} The current
state-of-the-art falsification approaches are based on numerical
simulations and can be grouped into robustness guided
S-Taliro~\cite{annpureddy2011s}, Breach~\cite{donze2010breach} and RRT
(Rapidly-exploring Random Tree)
based~\cite{nahhal_test_2007,Dang09,dreossi2015efficient}.  An
alternative approach, based on multiple shooting and CEGAR
(Counterexample Guided Abstraction Refinement) was proposed by the
authors in ~\cite{zutshi2014multiple}.  The alternative approach is
the verification approach, where a formal model (usually hybrid
automata) is first constructed (often manually) and then exhaustively
checked for violations.

\mypara{Identification of hybrid systems} using Piece-Wise affine
models is surveyed in~\cite{paoletti2007identification}.
\begin{enumerate}
\item Expressiveness/power of PWA models?
\item Fixed time caveats
\item Effects of refining parameters: time step $\Delta$ and grid size $\epsilon$.
\end{enumerate}

%\mypara{Model Checking PWA Models.} Model checking transition systems
