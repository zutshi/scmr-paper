% related work

\mypara{Sampling based falsification techniques for hybrid systems}
improve upon the commonly used testing approaches. However, a wide gap
between falsification and verification approaches. The current
state-of-the-art approaches (Cf. related work~\cite{nghiem2010monte})
for testing hybrid systems are based on numerical simulations and are
guided by robustness: S-Taliro~\cite{annpureddy2011s} and
Breach~\cite{donze2010breach}, and RRT (Rapidly-exploring Random
Tree):~\cite{nahhal_test_2007,Dang09} or
both~\cite{dreossi2015efficient}. An alternative approach based on
multiple shooting and CEGAR (Counterexample Guided Abstraction
Refinement), was proposed in~\cite{zutshi2014multiple}. Unlike the
previous approaches which directly use numerical simulations, it
searches an implicit abstraction by using the computed trajectories to
enumerate the relations between abstract states. The resulting graph
over relations is searched for paths corresponding to abstract
counter-examples. In this paper, we show how this search procedure can
be combined with the identification of the affine dynamics associated
with each relation of the abstraction, effectively computing a
`relational abstraction'. We now overview the several existing PWA
identification techniques for hybrid systems.

%another step to identify the affine dynamics
%associated with each relation of the abstraction, effectively
%computing a `relational abstraction'. This is akin to identifying a
%pwa model from simulations.

\mypara{Identification of hybrid systems.} Discrete time - continuous
state piece-wise affine models are often used to model both continuous
and hybrid dynamical systems. They are very expressive and can model
non-linear continuous dynamics (with arbitrary
accuracy)~\cite{wen2008basis} and a large family of hybrid
systems~\cite{heemels2001equivalence}. In the past, several approaches
have been put forth for the identification of PWA
models~\cite{paoletti2007identification}.  They are based on Bayesian
methods~\cite{juloski2005bayesian}, bounded-error
methods~\cite{bemporad2003greedy, bemporad2005bounded,
roll2004identification, alur2014precise}, clustering based
methods~\cite{ferrari2003clustering} and algebraic
methods~\cite{vidal2003algebraic}. We propose a simpler approach based
on fixed hyper-rectangular domains, also explored
in~\cite{billings1987piecewise}, but parameterized by the size of the
rectangles $\epsilon$ and the time discretization step $\Delta$.
\todo{other tech.}
%Other identification approaches have been explored in~\cite{saha2015alchemist}.
%nimit singhania

\mypara{Formal analysis of pwa systems.} Formal verification of pwa
systems using reachability analysis has been proposed
in~\cite{yordanov2010formal, koutsoukos2003safety,
asarin2000approximate}, and extended to model checking temporal
properties in~\cite{yordanov2007model, batt2007model}. PWA systems can
be equivalently translated to non-deterministic infinite state
transition systems and then model checked.  Related approaches using a
bisimilar quotient have been proposed in~\cite{pappas2003bisimilar,
tabuada2006linear, yordanov2007model}. Finally, because hybrid
automata models can equivalently represent pwa systems, existing
falsification and verification tools can be used for their analysis.

\mypara{Relational abstractions} were first introduced
by~\cite{Sankaranarayanan+Tiwari/2011/Relational} for abstracting away the
continuous dynamics in hybrid automata by discrete relations, %using suitable abstractions,
resulting in an infinite state transition system.  Both time
independent~\cite{Sankaranarayanan+Tiwari/2011/Relational} and time
dependent relations~\cite{zutshi2012timed, mover2013time} have been
proposed; the former captures all reachable states over all time,
whereas, the latter explicitly includes time by relating reachable
states to time. Thus it can prove timing properties whereas, time
independent relational abstractions can not. Properties of relational
abstractions can be verified using $k$-induction or falsified
using bounded model checkers.

%For general hybrid dynamical
%systems, algorithmically computing analytical solutions do not exist.


% Similar to timeless relational abstractions, time-aware relational
% abstractions \cite{mover2013time} construct binary relations between
% the current state $\x$ of the system and any future reachable state
% $\x'$. The difference lies with the latter also constructing relations
% between the current time $t$ and any future time $t'$.



% Alternatively, verification approaches have also been used, where a
% formal model (usually hybrid automata) is first constructed, often
% manually and then exhaustively checked for violations. By restricting
% ourselves to falsification

%Model checking of pwa systems.

%\item Fixed time caveats
%\item Effects of refining parameters: time step $\Delta$ and grid size $\epsilon$.

%\mypara{Model Checking PWA Models.} Model checking transition systems

% Mention yan's work



% \subsection{Relational Abstractions}

% For general hybrid dynamical systems, algorithmic ways for computing
% analytical solutions do not exist. Discretization (along with suitable
% abstractions) is often employed to transform the systems into a
% discrete transition system. Relational abstractions is one such idea,
% which abstracts continuous dynamics by discrete relations using
% appropriate \textit{reachability invariants}. Both time independent
% and time dependent relations have been proposed; the former captures
% all reachable states over all time, whereas, the latter explicitly
% includes time by relating reachable states to time. Thus it can prove
% timing properties whereas, time independent relational abstractions
% can not. In both cases, the resulting abstractions can be interpreted
% as discrete transition systems and can be analyzed using model
% checkers. We briefly discuss these two types of abstractions.

% \mypara{Timeless Relational Abstraction}

% Relational abstractions for hybrid systems were proposed in
% \cite{Sankaranarayanan+Tiwari/2011/Relational}. For a hybrid automaton
% model, they can summarize the continuous dynamics of each mode using a
% binary relation over continuous states. The resulting relations are
% timeless (independent) and hence valid for all time as long as the
% mode invariant is satisfied. The relations take the general form of
% $R(\x,\x') \bowtie 0$, where $\bowtie$ represents one of the
% relational operators $=, \ge, \le, <, >$. For example, an abstraction
% that captures the monotonicity with respect to time for the
% differential equation $\dot{x} = 2$ is $x' > x$. The abstraction
% capturing the relation between the set of ODEs: $\dot{x} = 2$,
% $\dot{y} = 5$, is $5(x' - x) = 2(y' - y)$.

% Such relation summaries, discretized the continuous evolution of a
% given system into an infinite state transition system, in which safety
% properties can be verified using $k$-induction, or falsified using
% bounded model checkers. The relationalization procedure involves
% finding suitable invariants in a chosen abstract domain, such as
% affine abstractions, eigen abstractions or box abstractions
% \cite{Sankaranarayanan+Tiwari/2011/Relational}.  For this, different
% techniques like template based invariant generation
% \cite{Gulwani+Tiwari/2008/Constraint,
% Colon+Sankaranarayanan+Sipma/03/Linear} can be used.

% \mypara{Timed and Time-Aware Relational Abstractions}

% As the above discussed relations are timeless, we cannot reason over
% the timing properties of the original system. Moreover, they are not
% suitable for time-triggered systems. Timed relational
% abstractions~\cite{zutshi2012timed} and time-aware relational
% abstractions~\cite{mover2013time} were proposed to overcome these
% shortcomings by explicitly including time in the relations. In
% addition, both can be more precise than their timeless counterpart,
% but unlike it, assume continuous affine dynamics.

% To analyze time-triggered systems like an SDCS modeled by an affine
% hybrid automata~\footnote{An affine hybrid automata is restricted to
% the ODEs, resets and guards being affine.}, timed relational
% abstractions can be computed directly from the solutions of the
% underlying affine ODEs.  Recall that the solution of an affine ODE
% $\dot{\x} = A\x$ can be represented using the matrix exponential as
% $\x(t) = e^{tA}\x(0)$. This gives us the timed relation $\x' =
% e^{tA}\x$.  Clearly, the relation is non-linear with respect to time.
% However, for the case of SDCS with a fixed sampling time period, we
% obtain linear relations in $\x$.  We demonstrated the usefulness of
% timed relational abstractions for the case of linear systems in
% \cite{zutshi2012timed}, and now incorporate the idea to discretized
% black box dynamical systems.

% Similar to timeless relational abstractions, time-aware relational
% abstractions \cite{mover2013time} construct binary relations between
% the current state $\x$ of the system and any future reachable state
% $\x'$. The difference lies with the latter also constructing relations
% between the current time $t$ and any future time $t'$. This is
% achieved by a case by case analysis of the eigen structure of the
% matrix $A$ (of an affine ODE). Separate abstractions are used for the
% case of linear systems with constant rate, real eigen values and
% complex eigenvalues.
