
We have presented another methodology to find falsifications in black
box dynamical systems. Combining the ideas from abstraction based
search \cite{zutshi2014multiple}, with our previous relational
abstractions~\cite{zutshi2012timed}, we proposed $k$-relational
modeling to compute richer abstractions. Simple linear regression can
be used to estimate the local dynamics of the trajectory segments and
compute PWA relational models which can be interpreted as a transition
system.  These can then be checked for safety violations using an
off-the-shelf bounded model checker.  Finally, we implemented the
ideas as a tool S3CAM-R and demonstrated its performance on a few examples.

\todo{memory required}
\todo{extension to LTL/MTL props}

% \subsection{Improvements}

% The approach seems promising as it avoids an expensive refinement
% step. But, before it can be applied to more complex systems, we need
% to overcome its shortcomings.  As a future extension, we are
% investigating ideas to improve the performance of the underlying three
% main sub-processes, namely, abstraction, modeling, and BMC. We detail
% these below.

% \begin{enumerate}

% \item \emph{Abstraction} The abstract domain we have used, has been
%     confined to the interval or rectangular domain. Such a domain is
%     very coarse when compared to more general polyhedral domains, using
%     which, we can gain better precision. Due to polyhedral domains
%     being computational expensive, their integration in S3CAM needs to
%     be explored thoroughly.

% \item \emph{Modeling} We have used $k$-relational modeling to increase
%     the precision of the enriched abstractions, however, there are
%     several other refinement strategies that remain to be explored.
%     This includes refining the abstract states by using guard
%     predicates (similar to predicate abstraction), using non-linear
%     templates in parametric regression to compute more precise
%     quadratic relations, and using an adaptive time step instead of a
%     fixed one to address the non-linearities due to a long time
%     horizon.

% \item \emph{Model Checking} As of now, we use SMT solvers to search
%     for concrete counter-examples. Although they are very efficient,
%     owing to extensive engineering effort, reachability of transition
%     systems resulting from dynamical systems remains a difficult
%     problem. As the time horizon of the safety property increases, the
%     possible combinations of discrete transitions increase
%     exponentially. Hence, to find a counterexample which is a sequence
%     of discrete transitions over a `long' time horizon is not
%     tractable for most, but the simplest of dynamical systems.
%     Instead, we can use linear programming solvers, by enumerating
%     each path in the graph $G^R$. Note that the constraints are all
%     linear (conjunctions) along an abstract path. However, in the
%     worst case, the number of paths in a graph can be of the order
%     $n!$ where $n$ is the number of vertices of a graph. Hence, this
%     will not be feasible, unless, we can prioritize paths by using a
%     triage process similar to one used by S3CAM and using a budget on
%     the maximum number of paths.

%     Another approach to address the issue would be to use an adaptive
%     time discretization technique, where relations over both shorter
%     and longer time steps are computed, and the SMT solver can
%     `select' the time step precise enough to find a counter-example.

% \item \emph{SMT Solvers} One major impediment to our approach is the
%     fact that SMT solvers use the theory of reals with \emph{exact
%     precision}. This is important for verification approaches, but can
%     be relaxed for the problem of falsification. An SMT solver which
%     uses approximate reasoning but returns robust counter-examples
%     will be as useful, and perhaps more efficient.

% \end{enumerate}

% \subsection{Data Driven Analysis}

% Finally, we would like to mention that our approach is `simulation'
% driven. It can be easily modified to be `data driven', by working with
% a fixed set of data. Given a data set, we then need to automatically
% find a good abstraction and a high fidelity PWA transition system
% using which, we can summarize the black box hybrid dynamical system.
% Apart from model checking the transition system, one can extend it to
% the analysis of SDCS. More specifically, by combining the transition
% system model of a plant with the control software, a model checker
% like CBMC \cite{kroening2014cbmc} can be used to do a closed loop
% symbolic analysis of the SDCS. This can be an alternative to S3CAM-X.

%which will be
%very useful to for analyzing data for potential property violations.
