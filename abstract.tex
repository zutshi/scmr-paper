Sampling based falsification techniques for black box hybrid dynamical
systems use numerical simulations to search for violations of
properties (specified in temporal logic). This constraints them to a
`best effort' category unlike rigorous verification techniques, which
however require a white box model. This work tries to bridge the gap
by presenting an automatic falsification technique which models the
black box system before searching for a violation. It proceeds by (a)
modeling the behavior of the system using a piecewise affine (PWA)
discrete time model and, (b) encoding the search for a violation as a
standard bounded model checking problem. The resulting problem is a
disjunction of linear programs and can be solved directly using
off-the-shelf SMT solvers or by multiple runs of a linear programming
solver. If a counter-example is found, the existence of a
corresponding violation in the original system is checked for
reproducibility.

% In this work we address the problem of finding unsafe behaviors with
% given a safety property for a hybrid dynamical system. We treat the
% system as a black box and approach the problem in two distinct steps.
% First, we model the behavior of the system using a piecewise affine
% discrete time model. Such a model is incomplete in nature and is
% computed with respect to the given property. Next, we encode the
% search for falsification as a bounded model checking query and use an
% SMT solver to find a counter example. If found, we check the existence
% of the violation in the original system to ensure reproducibility.

%Time bounded LTL property in
%Models from verification
%model without data
%black box model/data -> structured model
