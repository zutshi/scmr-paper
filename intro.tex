
% Testing of hybrid systems is a complex task, due to the interaction
% between discrete and continuous dynamics. Industrially developed
% hybrid systems, with embedded software interacting with physical
% components, are typically safety-critical. Thus, it is important to
% guarantee that the systems adhere to functional and safety
% specifications. Since the reachability problem for hybrid systems is
% known to in general be undecidable, exhaustive exploration of the
% state-space is intractable, and thus testing is an approach to attempt
% to achieve system correctness. However, to guarantee safety and other
% requirements, the testing procedures need to rely on a rigorous
% theoretical foundation. In this special session aims to collect and
% connect researchers, within academia as well as industry, working on
% model-based testing of hybrid systems. The session calls for papers
% proposing new theoretical research directions in the field, as well as
% papers describing practical applications of model-based testing of
% hybrid systems.


Model based design has been quite successful in the industry.
Tools like \SIMULINK are often used to create models describing hybrid
dynamical systems. Testing such systems for safety properties remains
a hard problem, and the current approaches consist of either guided
testing using numerical simulations or translating to hybrid automata.
We now suggest an approach which lies in the middle.

In this work we propose a two step process for searching violations of
the properties. This involves (a) learning a piece-wise affine model
and (b) exhaustively searching it for violations. It must be noted
that the violation found in the model might not be valid in the
original system, depending on the accuracy of the model.
% concretizing the violation in the original system.

The piece-wise affine model is the aggregation of affine maps, each
approximating observed local behavior of the system.  The model is
discrete time and `relational'. It relates the states reachable in a
fixed time step $\Delta$ to a given state.

In our previous work~\cite{zutshi2014multiple}, we analyzed systems by
restricting ourselves to their black box semantics. This enabled us to
reason about the state-space reachability of the system without a
direct analysis of its structure. Using a coarse abstraction which we
searched on-the-fly, we could observe the local dynamics as required.
This gave us an efficient procedure to find abstract counter-examples.
However, to concretize the counter-examples, a `closer look' or
refinement of the abstraction is required. The grid based state-space
abstraction was refined by splitting the relevant cells (abstract
states) into smaller ones. As noted previously, due to the curse of
dimensionality, such a uniform splitting is an expensive operation,
and can not scale to higher dimensions.

We further our exploration of trajectory segment based methods; and
explore an alternative approach to overcome the explosion in abstract
states. Instead of selectively refining the abstraction, we compute a
model of the black box system and use bounded model checking to find a
concrete counter-example in the model. Due to modeling errors, this
might not be reproducible in the original black-box system. We then
use this counter-example to guide the search towards a counter-example
in the original system.

More specifically, we use regression to quantitatively estimate the
discovered relations, which were witnessed by trajectory segments, by
affine maps. These maps approximate locally observed behaviors, which
are incorporated into the sampled reachability graph as edge
annotations.  The resulting graph or the Piece-Wise Affine (PWA)
relational model can be interpreted as an infinite state discrete
transition system and model checked for time bounded safety
properties. A counter-example if found, can indicate the presence of a
violation in the system.
%Furthermore, using linear programming, we can over-approximate the
%neighbhourhood of the counter-example in the model.

% In the conclusion, we discuss extensions to data driven
% approaches (instead of simulator driven), where instead of a
% simulator, a fixed set of data is provided as the behavioral
% description of the system. Using a combination of relational modeling,
% and program analysis we outline the future work for falsifying
% properties of SDCS.




% mention discrete time, cont. state space
% Bounded time behvaiors are modeled
% only one violation is being searched for
% mention explicitly the relation between older cegar work
