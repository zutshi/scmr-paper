
% Testing of hybrid systems is a complex task, due to the interaction
% between discrete and continuous dynamics. Industrially developed
% hybrid systems, with embedded software interacting with physical
% components, are typically safety-critical. Thus, it is important to
% guarantee that the systems adhere to functional and safety
% specifications. Since the reachability problem for hybrid systems is
% known to in general be undecidable, exhaustive exploration of the
% state-space is intractable, and thus testing is an approach to attempt
% to achieve system correctness. However, to guarantee safety and other
% requirements, the testing procedures need to rely on a rigorous
% theoretical foundation. In this special session aims to collect and
% connect researchers, within academia as well as industry, working on
% model-based testing of hybrid systems. The session calls for papers
% proposing new theoretical research directions in the field, as well as
% papers describing practical applications of model-based testing of
% hybrid systems.


Model-based development is rapidly becoming the preferred paradigm for
developing embedded control software in an industrial context.  Tools
like \SIMULINK are often used to create models of closed-loop
dynamical systems, \ie, a plant model of the physical systems that we
wish to control, and a model for the embedded controllers.  Automatic
approaches for testing the safety and performance of such closed-loop
systems is challenging, and current approaches focus mostly on guided
testing using numerical simulations
\cite{annpureddy2011s,donze2010breach,deshmukh2015stochastic,dreossi2015efficient,akazaki}.
On the other hand, theoretical research on hybrid and timed systems
focusses on formal guarantees for restricted classes of dynamical
systems such as hybrid automata where the system state evolves
according to a vector field that is either constant, affine
\cite{frehse2011spaceex}, or polynomial \cite{chen2015reachability}.
In this paper, we propose an approach that leverages useful features
from both kinds of techniques.

We propose a two step process for finding violations of safety
properties. In the first step, we learn a piece-wise affine (PWA) {\em
relational} model from simulations of the dynamical system. In the
second step, we leverage symbolic techniques to exhaustively analyze
the model learned in the first phase. We remark that the approach we
propose is neither sound nor complete: a violation found by our
technique may not exist in the original system, and our technique may
miss violations that exist in the original system. A fair question is:
why is such a technique useful? There are two perspectives from which
we can answer this question.

In many settings, we may not have a physics-based model derived from
first principles, but may just have time-series data of the system
behavior. In such a setting, a data-driven model allows us to
generalize individual time-series behaviors. But, there are several
approaches in the literature to perform data-driven modeling; e.g.,
system identification techniques that learn dynamic models from data
\cite{ljung1999system}, auto-regressive models for forecasting
\cite{wei1994time}, bounded-error hybrid systems identification
methods \cite{bemporad2003greedy,bemporad2005bounded}, clustering
based methods \cite{ferrari2003clustering}, methods to identify
piecewise affine dynamics \cite{paoletti2007identification}, and
machine learning techniques that learn static or dynamic models from
data
\cite{narendra1990identification,lu2009linear,juloski2005bayesian}.
Why then do we need a new modeling technique? The answer to this
question lies in our goal for the second step, \ie, symbolic analysis
of the learned model. For this step, we need to have models that can
be digested by existing symbolic tools, which is harder to do with
some of the aforementioned heavier data-driven modeling methods.
Hence, we use a lighter flavor of data-driven modeling such as the one
proposed for learning relational models from data
\cite{zutshi2012timed,sankaranarayanan2011relational}. By learning a
{\em PWA relational model}, \ie, essentially a PWA discrete-time
dynamical system, we can use software analysis approaches such as
bounded model checking assisted by SMT solvers and linear programming
solvers.

The other perspective for our technique is the context when we have an
effectively black-box model, \ie, a model that has features such as
nonlinearities, delays or look-up tables that make symbolic analysis
rather challenging.  We note that the ultimate goal for our technique
is to find violations of safety properties in the {\em actual embedded
control system}, and not, per se, in the model. The aphorism
attributed to mathematician George E.P. Box is quite apt for our
context, and bears worth repeating,  ``all models are wrong, some
models are useful.'' We can sidestep analyzing a highly complex model
by approximating its behavior with a simpler relational model, which
in turn can give us promising tests to run on the actual system. This
gives our technique intrinsic value as a debugging tool.

% In our previous work~\cite{zutshi2014multiple}, we analyzed systems by
% restricting ourselves to their black box semantics. This enabled us to
% reason about the state-space reachability of the system without a
% direct analysis of its structure. Using a coarse abstraction which we
% searched on-the-fly, we could observe the local dynamics as required.
% This gave us an efficient procedure to find abstract counter-examples.
% However, to concretized the counter-examples, a `closer look' or
% refinement of the abstraction is required. The grid based state-space
% abstraction was refined by splitting the relevant cells (abstract
% states) into smaller ones. As noted previously, due to the curse of
% dimensionality, such a uniform splitting is an expensive operation,
% and can not scale to higher dimensions.

In \cite{zutshi2014multiple}, the authors investigate an on-the-fly
abstraction technique that is close to the approach we propose.  Both
approaches use an existential abstraction of the state-space of the
hybrid dynamical system. The abstract model learned is a directed
acyclic graph, where nodes represent regions of the state-space, and
an edge between two regions indicates a transition between states in
the regions.  A key difference in this work is that the model being
learned in addition to being an existential graph-based abstraction is
also a {\em relational} abstraction.  In other words, the approach
presented here improves the existential abstraction by annotating the
edges in the graph-based abstraction with affine relations between the
source and destination states.  More specifically, we use linear
regression to quantitatively estimate the discovered edges in the
graph-based abstraction of the state-space. These linear maps
approximate locally observed behaviors, and the resulting graph or the
Piece-Wise Affine (PWA) relational model can be interpreted as an
infinite state discrete transition system and model checked for time
bounded safety properties. A counter-example if found, can indicate
the presence of a violation in the system.  We then use this
counter-example to guide the search towards a counter-example in the
original system. 

We remark that our work focusses on building discrete PWA models, and
is unrelated to work on analyzing dynamical systems that have PWA
vector fields. Techniques to analyze PWA dynamical systems using
bisimulation relations translate the system into a non-deterministic
infinite state transition system, and use model checking approaches to
verify them \cite{asarin2000approximate}. Related approaches using
bisimulation relations \cite{pappas2003bisimilar,tabuada2006linear,
yordanov2007model} have interesting parallels with the work presented
here.

%Furthermore, using linear programming, we can over-approximate the
%neighbhourhood of the counter-example in the model.

% In the conclusion, we discuss extensions to data driven
% approaches (instead of simulator driven), where instead of a
% simulator, a fixed set of data is provided as the behavioral
% description of the system. Using a combination of relational modeling,
% and program analysis we outline the future work for falsifying
% properties of SDCS.
% Bounded time behaviors are modeled
% only one violation is being searched for
% mention explicitly the relation between older cegar work

The layout of the paper is as follows: In Section~\ref{sec:prelims},
we introduce the basic steps to obtain a PWA relational model. In
Section~\ref{sec:pwa-rel}, we explore the connection between the PWA
relational model and previous graph-based abstraction techniques such
as those in \cite{zutshi2014multiple}. We present extensions of
relational modeling in Section~\ref{sec:rel-mod}, and present
experimental validation of our technique on a few textbook examples of
dynamical systems with hybrid and polynomial dynamics in
Section~\ref{sec:res}.
