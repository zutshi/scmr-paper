
% Testing of hybrid systems is a complex task, due to the interaction
% between discrete and continuous dynamics. Industrially developed
% hybrid systems, with embedded software interacting with physical
% components, are typically safety-critical. Thus, it is important to
% guarantee that the systems adhere to functional and safety
% specifications. Since the reachability problem for hybrid systems is
% known to in general be undecidable, exhaustive exploration of the
% state-space is intractable, and thus testing is an approach to attempt
% to achieve system correctness. However, to guarantee safety and other
% requirements, the testing procedures need to rely on a rigorous
% theoretical foundation. In this special session aims to collect and
% connect researchers, within academia as well as industry, working on
% model-based testing of hybrid systems. The session calls for papers
% proposing new theoretical research directions in the field, as well as
% papers describing practical applications of model-based testing of
% hybrid systems.


Model-based development is rapidly becoming the preferred paradigm for
developing embedded control software in an industrial context.  Tools
like \SIMULINK are often used to create models of closed-loop
dynamical systems, \ie, a plant model of the physical systems that we
wish to control, and a model for the embedded controllers.  Automatic
approaches for testing the safety and performance of such closed-loop
systems is challenging, and current approaches focus mostly on guided
testing using numerical simulations
\cite{annpureddy2011s,donze2010breach,deshmukh2015stochastic,dreossi2015efficient,akazaki}.
On the other hand, theoretical research on hybrid and timed systems
focusses on formal guarantees for restricted classes of dynamical
systems such as hybrid automata where the system state evolves
according to a vector field that is either constant, affine
\cite{frehse2011spaceex}, or polynomial \cite{chen2015reachability}.
In this paper, we propose an approach that leverages useful features
from both kinds of techniques.

We propose a two step process for finding violations of safety
properties. In the first step, we learn a piece-wise affine (PWA) {\em
relational} model from simulations of the dynamical system. In the
second step, we leverage symbolic techniques to exhaustively analyze
the model learned in the first phase. We remark that the approach we
propose is neither sound nor complete: a violation found by our
technique may not exist in the original system, and our technique may
miss violations that exist in the original system. A fair question is:
why is such a technique useful? There are two perspectives from which
we can answer this question.

In many settings, we may not have a physics-based model derived from
first principles, but may just have time-series data of the system
behavior. In such a setting, a data-driven model allows us to
generalize individual time-series behaviors through an abstract, PWA
relational model. There are several approaches in the literature to
perform data-driven modeling, system identification techniques that
learn dynamic models from data \cite{ljung1999system}, auto-regressive
models for forecasting \cite{wei1994time}, machine learning techniques
that learn static or dynamic models from data
\cite{narendra1990identification,lu2009linear}. However, in the second
step of our technique, \ie, symbolic analysis of the learned mode, we
need to have models that can be digested by existing symbolic tools,
which is harder to do with some of the aforementioned heavier
data-driven modeling methods. Hence, we use a lighter flavor of
data-driven modeling such as the one proposed for learning relational
models from data
\cite{zutshi2012timed,sankaranarayanan2011relational}. By learning
what is essentially a discrete-time dynamical system with PWA updates,
our technique permits us to use software analysis approaches such as
bounded model checking assisted by SMT and linear programming solvers.

The other perspective is when we have a model that is highly complex.
We note that the ultimate goal for our technique is to find violations
of safety properties in the {\em actual embedded control system}, and
not, per se, in the model. The aphorism attributed to mathematician
George E.P. Box is worth repeating,  ``all models are wrong, some
models are useful.'' Thus, we can sidestep analyzing a highly complex
model by approximating its behavior with a simpler relational model,
which in turn gives us promising tests to run on the actual system,
then our technique has intrinsic value.

% In our previous work~\cite{zutshi2014multiple}, we analyzed systems by
% restricting ourselves to their black box semantics. This enabled us to
% reason about the state-space reachability of the system without a
% direct analysis of its structure. Using a coarse abstraction which we
% searched on-the-fly, we could observe the local dynamics as required.
% This gave us an efficient procedure to find abstract counter-examples.
% However, to concretize the counter-examples, a `closer look' or
% refinement of the abstraction is required. The grid based state-space
% abstraction was refined by splitting the relevant cells (abstract
% states) into smaller ones. As noted previously, due to the curse of
% dimensionality, such a uniform splitting is an expensive operation,
% and can not scale to higher dimensions.

In \cite{zutshi2014multiple}, the authors investigate an on-the-fly
abstraction technique that is close to the approach we propose. A key
difference is that the abstraction explored in this work is an {\em
existential} abstraction: the approximate model learned is a directed
acyclic graph, where nodes represent regions of the state-space, and
an edge between two regions indicates that there is a transition
between some states in the regions. Our approach can be viewed as a
further improvement of the existential abstraction by annotating an
edge in the graph-based abstraction with an affine relation between
the source and destination states connected by the edge.

% We further our exploration of trajectory segment based methods; and
% explore an alternative approach to overcome the explosion in abstract
% states. Instead of selectively refining the abstraction, we compute a
% model of the black box system and use bounded model checking to find a
% concrete counter-example in the model. Due to modeling errors, this
% might not be reproducible in the original black-box system. 


More specifically, we use linear regression to quantitatively estimate
the discovered edges in the graph-based abstraction of the
state-space. These linear maps approximate locally observed behaviors,
and the resulting graph or the Piece-Wise Affine (PWA) relational
model can be interpreted as an infinite state discrete transition
system and model checked for time bounded safety properties. A
counter-example if found, can indicate the presence of a violation in
the system.  We then use this counter-example to guide the search
towards a counter-example in the original system.


%Furthermore, using linear programming, we can over-approximate the
%neighbhourhood of the counter-example in the model.

% In the conclusion, we discuss extensions to data driven
% approaches (instead of simulator driven), where instead of a
% simulator, a fixed set of data is provided as the behavioral
% description of the system. Using a combination of relational modeling,
% and program analysis we outline the future work for falsifying
% properties of SDCS.
% Bounded time behvaiors are modeled
% only one violation is being searched for
% mention explicitly the relation between older cegar work

The layout of the paper is as follows: In Section~\ref{sec:prelims},
we introduce the basic steps to obtain a PWA relational model. In
Section~\ref{sec:pwa-rel}, we explore the connection between the PWA
relational model and previous graph-based abstraction techniques such
as those in \cite{zutshi2014multiple}. We present extensions of
relational modeling in Section~\ref{sec:rel-mod}, and present
experimental validation of our technique on a few textbook examples of
dynamical systems with hybrid and polynomial dynamics in
Section~\ref{sec:res}.
