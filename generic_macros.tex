% Generic
%\DeclareMathAlphabet{\mathpzc}{OT1}{pzc}{m}{it}

% English
\newcommand{\ie}{{i.e.}\xspace}
\newcommand{\Ie}{{I.e.}\xspace}
\newcommand{\eg}{{e.g.}\xspace}
\newcommand{\etc}{{etc.}\xspace}
\newcommand{\viz}{{viz.\xspace}}
\newcommand{\etal}{{et al.}\xspace}

\renewcommand\vec[1]{\mathbf{#1}}
% Generic refs
\newcommand{\lemref}[1]{Lemma~\ref{lem:#1}}
\newcommand{\secref}[1]{Sec.~\ref{sec:#1}}
\newcommand{\figref}[1]{Fig.~\ref{fig:#1}}
\newcommand{\exref}[1]{Example~\ref{ex:#1}}
\newcommand{\thmref}[1]{Theorem~\ref{thm:#1}}
\newcommand{\tabref}[1]{Table~\ref{tab:#1}}
\newcommand{\algoref}[1]{Algorithm~\ref{algo:#1}}
\newcommand{\lstref}[1]{Listing~\ref{lst:#1}}
\newcommand{\chapref}[1]{Chapter~\ref{chap:#1}}
\newcommand{\defref}[1]{Definition~\ref{def:#1}}

% Comments, Reviewing, Formatting
\newcommand{\ignore}[1]{}
\newcommand\todo[1]{[[\textcolor{red}{\textsf{#1}}]]}
% General Math
\newcommand{\setof}[1]{\ensuremath{\{#1\}}}
\newcommand\tupleof[1]{\ensuremath\left\langle #1 \right \rangle}
\newcommand{\sublist}[2]{\ensuremath{#1_{1},\ldots,#1_{#2}}}
\newcommand{\suplist}[2]{\ensuremath{#1^{1},\ldots,#1^{#2}}}
\newcommand \card[1] {\left| #1 \right|}
\newcommand \floor[1] {\left\lfloor #1 \right\rfloor}
\newcommand \ceil[1] {\left\lceil #1 \right\rceil}

% Complexity
\newcommand{\npcomplete}{\textsc{Np}-\textsc{complete}}
\newcommand{\nphard}{\textsc{Np}-\textsc{Hard}}
\newcommand{\nlogspace}{\textsc{NLogspace}\xspace}
\newcommand{\pspace}{\textsc{PSpace}\xspace}
\newcommand{\expspace}{\textsc{ExpSpace}\xspace}

\newcommand{\mcl}[1]{\multicolumn{1}{l}{#1}}
\newcommand{\mcc}[1]{\multicolumn{1}{c}{#1}}

\newcommand{\mypara}[1]{\vspace{0.6em} \noindent{\bf #1}}
\newcommand{\myipara}[1]{\vspace{0.6em} {\em #1}\xspace}
%\newcommand{\myipara}[1]{\vspace{0.6em} \noindent{\em #1}\xspace}
\newcommand{\myiparaCompact}[1]{ {\em #1}\xspace}

% Sets
\newcommand{\Reals}{\ensuremath{\mathbb{R}}}
\newcommand{\reals}{\Reals}
\newcommand{\real}{\Reals}
\newcommand{\Nats}{\ensuremath{\mathbb{N}}}
\newcommand{\Integers}{\ensuremath{\mathbb{Z}}}
\newcommand{\Rationals}{\ensuremath{\mathbb{Q}}}

%%\newcommand{\mathsf}[1]{\mbox{\textsc{#1}}}
\newcommand{\mathsc}[1]{\mbox{\sc #1}}
\newcommand\scr[1]{\ensuremath\mathcal{#1}}
\newcommand\Gradient{\ensuremath\nabla}
\newcommand\pdiff[2]{\partial_{#1}{#2}}
\newcommand\jth[2]{ #1^{(#2)} }

% Logic
% \newcommand{\land}{\ensuremath\wedge}
% \newcommand{\lor}{\ensuremath\vee}

% Software
\newcommand\flowstar{FLOW*}
\newcommand\MATLAB{Matlab\textsuperscript{\textregistered}}
\newcommand\SIMULINK{Simulink\textsuperscript{\textregistered}}
\newcommand\STATEFLOW{Stateflow\textsuperscript{\textregistered}}
\newcommand\EMCODER{Embedded Coder\textsuperscript{\textregistered}}

%\newcommand\vs{\mathbf{s}}
%\newcommand\vw{\mathbf{w}}
%\newcommand\vx{\mathbf{x}}
%\newcommand\vy{\mathbf{y}}
%\newcommand\vz{\mathbf{z}}
%\newcommand\vu{\mathbf{u}}
%\newcommand\vj{\mathbf{j}}

\newcommand \vx {\vec{x}}
\newcommand \vy {\vec{y}}
\newcommand \vz {\vec{z}}
\newcommand \vu {\vec{u}}
\newcommand \vb{\vec{b}}

% long squiggily arow
\newcounter{sarrow}
\newcommand\xrsquigarrow[1]{%
\stepcounter{sarrow}%
\begin{tikzpicture}[decoration=snake]
\node (\thesarrow) {\strut#1};
\draw[->,decorate] (\thesarrow.south west) -- (\thesarrow.south east);
\end{tikzpicture}
}

% misc
\newcommand\ii{i+1}
\newcommand\denotation[1]{ \left\llbracket #1 \right\rrbracket}



% The \munepsfig command is used to insert a new EPS figure
% into our document.  Usage is:
%
%       \munepsfig[args]{filename}{caption}
%
% where:
%       - the optional 'args' argument is passed to the
%         embedded \includegraphics command, this can be used
%         to scale the figure or rotate it.
%       - 'filename' is the name of the EPS file in the 'figures'
%         directory that is to be inserted (note that 'filename'
%         should not include the '.eps' extension).
%       - 'filename' also serves as the label for the figure.
%         with the text 'fig:' prepended.
%
% Sample Usage:
%       \munepsfig[scale=0.5,angle=90]{barchart}{Population over time}

% inserts the EPS file 'figures/barchart.eps' reduced in size by 50%
% rotated 90 degrees and with the caption "Popuation over Time."
% We can refer to that figure as Figure~\ref{fig:barchart} in the text.
%
\newcommand{\inclfig}[3][scale=1.0]{%
        \begin{figure}[h]
                \centering
                \vspace{2mm}
%               \includegraphics[#1]{figures/#2.eps}
                \includegraphics[#1]{figs/#2}
                \caption{#3}
                \label{fig:#2}
        \end{figure}
}

% Units
%\newcommand{\degree}{^{\circ}}
\newcommand{\degreeC}{^{\circ}{\rm C}}
%\newcommand{\degreeF}{^{\circ}{\rm F}}

%\newcommand{\CC}{C\nolinebreak\hspace{-.05em}\raisebox{.4ex}{\tiny\bf +}\nolinebreak\hspace{-.10em}\raisebox{.4ex}{\tiny\bf +}}
\def\Cpp{{C\nolinebreak[4]\hspace{-.05em}\raisebox{.4ex}{\tiny\bf ++}}}
\def\CC{{C\nolinebreak[4]\hspace{-.05em}\raisebox{.4ex}{}}\;}


