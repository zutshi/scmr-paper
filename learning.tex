The $\mathsf{ScatterAndSimulate}$ procedure helps in constructing the
$\epsilon$-TS abstraction using a user-specified budget of trajectory
fragments (obtained using simulation).  In general, the budget
required to adequately explore the transition space requires exploring
a number of states that is exponential in the dimension of
$\HybridStates$.  This can cause undue burden in terms of computation
time as well as storage.  The central idea in this paper is to {\em
summarize} enrich the transitions in the $\epsilon$-TS abstraction
using relations to obtain an $\epsilon$-RTS. 

In other words, suppose there is an edge from cell $C$ to $C'$, and if
$\vx \in C$ and $\vx' \in C'$, then we can consider all concrete
trajectories that go from $C$ to $C'$, and use these to replace
several ``proximal'' tuples $(\vx,\vxp)$ by a map $F$ of the form
$\vxp = F(\vx, \vu) \pm \delta$, where $\delta$ is a worst-case error
estimate. The form of $F$ depends on the basis function used for
regression. We focus on linear regression with different basis
functions such as affine and polynomial. For the special case of
linear regression, the resulting update relation can be represented by
matrices $A$ and $B$, \ie, $r(\vx,\vxp) = \setof{(\vx,\vxp) \mid |\vxp
- (A\vx + B\vu)| < \delta}$.

In statistics, parametric regression is the problem of finding the
paremters of a function or \textit{predictor}, which can `best'
summarize the relationship between the given set of observed input
$\x$ and output $\x'$ vectors. The basis functions used for the
predictor are fixed beforehand and the paramters are searched for.
The notion of `best' is formally captured using a loss function which
can be non-linear. We use the commonly used loss function
\textit{ordinary least squares} (OLS) to fit affine and polynomial
kernels of fixed degree.  ~\footnote{The choice of loss function and
regression analysis affects the quality of the learnt models. As the
procss of selecting an appropriate heuristic is not straightforward,
we just use the simplest one.} OLS defines its loss function as the
sum of squares of the errors in prediction. The task is then to
determine the parameters, such that the least square error of the
predictor is minimized for the given data set.  The details on
performing linear regressoin can be found in standard texts on
learning and statistics~\cite{friedman2001elements}.

% Suppose we have $N$ tuples of the
% form $(\vx,\vxp)$ in the reachability relation that we have explored,
% where $\x\in\reals^n$ and $\x'\in\reals^n$. Then, we wish to learn a
% function described as $\x' = F(\vx, \vu)$.  


% In this fashion, we can we can use OLS to approximate its trajectories
% of system $\System$ at fixed time step $\Delta$ by a discrete map $\x'
% = F(\hat{\x})$.

We note that regression tools are often able to provide an
$n\times 1$ vector $\vec{\delta}$ of error intervals, where the
$i^{th}$ entry of the form $[{\delta_i}_{\min},{\delta_i}_{\max}]$
indicates the best and worst possible error in computation of the
in $\vxp_i$.

The choice of basis functions for constructing an $\epsilon$-RTS
involves a tradeoff between computing precise local models and
efficiency of the analysis of the resulting relational abstraction.
Affine basis functions can be poor approximations for complex dynamics
and can result in large errors, but they can be very efficiently
analyzed in practice.  On the other hand, analysis of arbitrary
non-linear functions is not generally tractable. We restrict ourselves
to affine and quadratic polynomial basis functions.  Moreover, as we
use a collection of relations to approximate the local behaviors (in
state-space) of the system $\System$, we can often do so with simple
basis functions. 

\begin{example}
    For the~\exref{vdp}, for every abstract relation between two cells
    $C, C'$, linear regression analysis is performed on the respective
    set of trajectory segments, and the affine relation is computed.
    \figref{vdp-abs-paths} shows the cells and the trajectory
    segments, which are part of the data sets constructed using the
    $1$-relational modeling. Against each cell, its unique identifier
    (integer co-ordinate) is mentioned.
\end{example}
