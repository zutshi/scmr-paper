We describe the the procedure $\mathsf{BoundedModelCheck}$ used at
Line~\ref{algoline:bmc} in Algorithm~\ref{algo:s3camBMC}
that checks if the $\epsilon$-precision Relational Transition System
$\RTSAbstraction{\epsilon}{\System}$ violates the safety property
$\psi(\vx)$ in at most $k$ steps.

We describe two possible implementations:
(1) a symbolic Bounded Model Checking~\cite{Biere+Others/99/Symbolic} encoding of
the paths of $\RTSAbstraction{\epsilon}{\System}$ of length $k$ that 
violates $\psi$, and (2) an algorithm that explicitly enumerates the
``discrete'' paths formed cells $\Cells$ and the transition
$\CellEdges$, and then checks if the relations of the variables $\vx$
on the visited edges hold for for each path (this is in similar in
spirit to~\cite{DBLP:conf/fmcad/BuLWL08}).


%% BMC macro
\newcommand{\bvars}[1]{\ensuremath{{B}^{#1}}}
\newcommand{\rvars}[1]{\ensuremath{{\vx}^{#1}}}
\newcommand{\vars}[1]{\ensuremath{{V}^{#1}}}
\newcommand{\cellvar}[1]{\ensuremath{{cell}^{#1}}}
\newcommand{\initbmc}{\ensuremath{INIT}}
\newcommand{\encbmc}{\ensuremath{BMC}}
\newcommand{\errorbmc}{\ensuremath{ERROR}}
\newcommand{\transbmc}{\ensuremath{TRANS}}
\newcommand{\edgebmc}{\ensuremath{EDGE}}
\newcommand{\subs}[3]{\ensuremath{{#1}[{#2}/{#3}]}}

Given a set of variables $X$, we denote with $X^i=\{x^i | x \in X\}$
the copy of the variables $X$ renamed with the $^i$ superscript.
Given a Boolean formula $\exprs{V}$
%% $\subs{\exprs{V}}{z}{x}$ is the formula $\exprs{V}$ where all the
%% occurrences of the variable $x$ are replaced with $z$.
We use the shorthand $\subs{\exprs{V}}{X^i}{X}$ to denote the formula
where every variable $x$ is substituted to $x^+$

We encode the finite set of cells $\Cells$ with a set of Boolean
variables $\bvars{}$. For clarity, given the cell $C \in \Cells$ we
write $\cellvar{} = C$, instead of showing its encoding with $\bvars{}$.
%
In the following, we interpret each variable $x \in \vx$ as a Real
valued variable and we refer to all the Boolean and Real variables
with the set $\vars{} = \rvars{} \vee \bvars{}$
%
We encode the set of paths of length $k$ of
$\RTSAbstraction{\epsilon}{\System}$ that violates the property $\psi$
as the quantifier first-order logic formula:
\begin{eqnarray*}
\encbmc(\RTSAbstraction{\epsilon}{\System}, \psi, k) & := &
\initbmc(\vars{0}) \wedge
\bigwedge_{i=1}^{k}{\transbmc(\vars{i-1}, \vars{i})} \wedge
\neg \subs{\psi}{\rvars{i}}{\rvars{}}\\
% init
\initbmc(\vars{i}) & := &
\bigwedge_{C \in \InitCells}{(\cellvar{i} = C) \wedge
\subs{\CellLabelingFunction(C)}{\rvars{i}}{\rvars{}}
}
\\
% trans
\transbmc(\vars{i}, \vars{i+1}) & := &
  \bigvee_{(C,C^+) \in \CellLabelingFunction}{
    \edgebmc_{(C,C^+)}(\vars{i}, \vars{i+1})
  } \\
\edgebmc_{(C,C^+)}(\vars{i}, \vars{i+1}) & := &
(\cellvar{i} = C) \wedge (\cellvar{i+1} = C^+) \wedge  \\
&& \subs{\CellLabelingFunction(C)}{\rvars{i}}{\rvars{}} \wedge
   \subs{\CellLabelingFunction(C^+)}{\rvars{i+1}}{\rvars{}} \wedge \\
&& \subs{\subs{\EdgeLabelingFunction((C,C^+))}{\rvars{i}}{\rvars{}}}{\rvars{i+1}}{{\rvars{+}}} \\
\end{eqnarray*}

The formula
$\encbmc(\RTSAbstraction{\epsilon}{\System}, \psi, k)$ is satisfiable
if and only if there exists a path of length $k$.
The procedure
$\mathsf{BoundedModelCheck}(\RTSAbstraction{\epsilon}{\System}, \psi, k)$
checks the satisfiability of the sequence of formulas 
$\encbmc(\RTSAbstraction{\epsilon}{\System}, \psi, 0)$, 
$\ldots$,
$\encbmc(\RTSAbstraction{\epsilon}{\System}, \psi, k)$
with a SMT solver, returning as soon as one of the formula is
satisfiable (i.e. as soon as it found a path in
$\RTSAbstraction{\epsilon}{\System}$).









The PWA transition system is a finite location, infinite state
transition system. As all the guards and transitions are affine, a
bounded reachability query is equivalent to checking all
combinations of discrete locations, where each combination can be
summarized as a linear program. As the time is bounded, the number of
combinations are finite, but, exponential in number.

It should be noted that an explicit execution of the transition system
represents a directed acyclic graph, with each location being a node,
and a directed edge from each node to all nodes reachable in forward
time. A path on this graph is then a linear program, with each
branching node corresponding to a logical disjunction.

Temporal properties of PWA transitions systems can be checked using
off-the-shelf model checkers like SAL~\cite{SAL-SRI}, which can reason
over infinite state transitions systems using SMT solvers.
Furthermore, lazy SMT solvers can be employed for this specific
problem instance to achieve better efficiency~\cite{shoukry2017smc}.
We now show how relational PWA abstractions can be viewed as PWA
transition systems, and how the problem of falsification can
be answered by a BMC query.


% To clarify:
%  - is there a maximum k that we look at? Where do we guarnatee that?
%    
