%% BMC macro
\newcommand{\unsafebmc}{\ensuremath{{\psi}_{\UnsafeStates}}}
\newcommand{\bvars}[1]{\ensuremath{{\vec{B}}^{#1}}}
\newcommand{\rvars}[1]{\ensuremath{{\vx}^{#1}}}
\newcommand{\vars}[1]{\ensuremath{{\vec{V}}^{#1}}}
\newcommand{\cellvar}[1]{\ensuremath{{cell}^{#1}}}
\newcommand{\initbmc}{\ensuremath{INIT}}
\newcommand{\encbmc}{\ensuremath{BMC}}
\newcommand{\transbmc}{\ensuremath{TRANS}}
\newcommand{\edgebmc}{\ensuremath{EDGE}}
\newcommand{\subs}[3]{\ensuremath{{#1}[{#2}/{#3}]}}

\newcommand{\abspathbmc}[0]{\ensuremath{\Pi_{abs}}}
\newcommand{\concbmc}[1]{\ensuremath{CONC_{{#1}}}}
%
We describe the implementation of the procedure
$\mathsf{BoundedModelCheck}$ used at Line~\ref{algoline:bmc} in
Algorithm~\ref{algo:s3camBMC},
that checks if the $\epsilon$-precision Relational Transition System
$\RTSAbstraction{\epsilon}{\System}$ reaches an error condition
$\UnsafeStates$ in at most $k$ steps returning, in that case, a path
$\Pi$. In the following we assume that the set of unsafe states
$\UnsafeStates$ can be represented with a Boolean formula
$\unsafebmc(\vx)$.

We describe two implementations:
(1) a symbolic Bounded Model Checking~\cite{Biere+Others/99/Symbolic}
(BMC) algorithm and (2) an algorithm (similarly
to e.g.~\cite{DBLP:conf/fmcad/BuLWL08}) that explicitly enumerates the
abstract paths formed by cells $\Cells$ and edges $\CellEdges$,
checking for their feasibility in $\RTSAbstraction{\epsilon}{\System}$.


\paragraph{Symbolic path exploration.}
%
%
Given a set of variables $X$, we denote with $X^i=\{x^i | x \in X\}$
the copy of the variables $X$ renamed with the ``$^i$'' superscript.
Given a Boolean formula $\exprs{\vars{}}$,
%% $\subs{\exprs{V}}{z}{x}$ is the formula $\exprs{V}$ where all the
%% occurrences of the variable $x$ are replaced with $z$.
$\subs{\exprs{\vars{}}}{\vars{i}}{\vars{}}$ is the formula $\exprs{\vars{}}$
where every variable $v \in \vars{}$ is substituted with $v^+ \in \vars{+}$.

We encode the finite set of cells $\Cells$ with a set of Boolean
variables $\bvars{}$. For clarity, given the cell $C \in \Cells$ we
write $\cellvar{} = C$, instead of showing its encoding with $\bvars{}$.
%
In the following, we interpret each variable $x \in \vx$ as a Real
valued variable and we refer to all the Boolean and Real variables
with the set $\vars{} = \rvars{} \vee \bvars{}$
%
We encode the set of paths of length $k$ of
$\RTSAbstraction{\epsilon}{\System}$ that reaches $\unsafebmc$
as the quantifier-free first-order logic formula:
\begin{eqnarray*}
\encbmc(\RTSAbstraction{\epsilon}{\System}, \unsafebmc, k) & := &
\initbmc(\vars{0}) \wedge
\bigwedge_{i=1}^{k}{\transbmc(\vars{i-1}, \vars{i})} \wedge
\subs{\unsafebmc}{\rvars{k}}{\rvars{}}\\
% init
\initbmc(\vars{i}) & := &
\bigwedge_{C \in \InitCells}{(\cellvar{i} = C) \wedge
\subs{\CellLabelingFunction(C)}{\rvars{i}}{\rvars{}}
}
\\
% trans
\transbmc(\vars{i}, \vars{i+1}) & := &
  \bigvee_{(C,C^+) \in \CellLabelingFunction}{
    \edgebmc_{(C,C^+)}(\vars{i}, \vars{i+1})
  } \\
\edgebmc_{(C,C^+)}(\vars{i}, \vars{i+1}) & := &
(\cellvar{i} = C) \wedge (\cellvar{i+1} = C^+) \wedge  \\
&& \subs{\CellLabelingFunction(C)}{\rvars{i}}{\rvars{}} \wedge
   \subs{\CellLabelingFunction(C^+)}{\rvars{i+1}}{\rvars{}} \wedge \\
&& \subs{\subs{\EdgeLabelingFunction((C,C^+))}{\rvars{i}}{\rvars{}}}{\rvars{i+1}}{{\rvars{+}}}
\end{eqnarray*}

The formula
$\encbmc(\RTSAbstraction{\epsilon}{\System}, \unsafebmc, k)$ is satisfiable
if and only if there exists a path of length $k$ in
$\RTSAbstraction{\epsilon}{\System}$ that reaches a state in $\unsafebmc$.

The procedure
$\mathsf{BoundedModelCheck}(\RTSAbstraction{\epsilon}{\System})$
checks the satisfiability of the sequence of formulas
$\encbmc(\RTSAbstraction{\epsilon}{\System}, \unsafebmc, 0)$,
$\ldots$,
$\encbmc(\RTSAbstraction{\epsilon}{\System}, \unsafebmc, k)$
with an SMT solver, returning a path $\Pi$ of
$\RTSAbstraction{\epsilon}{\System}$ as soon as one of the formula is
satisfiable, or no paths otherwise.

\paragraph{Explicit path exploration.}
The second algorithm explicitly enumerates
the (abstract) paths in the graph that has $\Cells$
as nodes, and $\CellEdges$ as edges, and may end in an unsafe state
$\UnsafeStates$.
%
Such paths may not be real paths of
$\RTSAbstraction{\epsilon}{\System}$, since they may not satisfy the
conditions imposed by the labeling function $\CellLabelingFunction$
and the relation $\EdgeLabelingFunction$, and hence the algorithm must
further check, for each abstract path $\abspathbmc$, if there exists a
concrete path $\Pi$ of
$\RTSAbstraction{\epsilon}{\System}$.

$\abspathbmc := C_0, \ldots, C_l$ is an abstract path of
 $\RTSAbstraction{\epsilon}{\System}$ where:
\begin{itemize}[label=--,leftmargin=1em,labelsep=*]
\item for $i \in [0,l]$, $C_i \in \Cells$,
\item $C_0 \in \InitCells$,
\item for $i \in [0,l-1]$, $(C_i, C_{i+1}) \in \EdgeLabelingFunction$,
\item $C_l$ is such that
$\CellLabelingFunction(C_l) \land \unsafebmc(\vx)$ is satisfiable
(i.e. the cell $C_l$ contains an unsafe state).
\end{itemize}
%
There exists a concrete path $\Pi$ of
$\RTSAbstraction{\epsilon}{\System}$ that traverses the sequences of
cells of an abstract path $\abspathbmc$ if and only if the formula
\begin{eqnarray}
\label{eq:simulateabs}
\concbmc{\abspathbmc} & := &
\bigg( \bigwedge_{i=0}^{l}{
  \subs{\CellLabelingFunction(C_i)}{\rvars{i}}{\rvars{}}
} \bigg)
\wedge
\subs{\unsafebmc}{\rvars{l}}{\rvars{}} \wedge \\
&&
\bigwedge_{i=1}^{l}{
  \subs{
  \subs{\CellEdges((C_i,C_{i+1}))}{\rvars{i}}{\rvars{}}
  }{\rvars{i+1}}{\rvars{+}}
}
\end{eqnarray}
is satisfiable\footnote{Formula~\ref{eq:simulateabs} is a conjunction
of linear predicates and hence its satisfiability can be casted as the
feasibility of a linear program.}.
%
This can be proved by showing that each model of the formula
$\concbmc{\abspathbmc}$  ``corresponds'' to a path of
$\RTSAbstraction{\epsilon}{\System}$, and each path of
$\RTSAbstraction{\epsilon}{\System}$ that traverses the sequence of
cells of $\abspathbmc$ ``corresponds'' to a model of
$\concbmc{\abspathbmc}$.

The explicit path exploration enumerates all the abstract
paths $\abspathbmc$ of length at most $k$. The enumeration can either
be a depth-first search (DFS) or a breadth-first search (BFS). As soon
as a $\abspathbmc$ is found, the algorithm checks the satisfiability
of $\concbmc{\abspathbmc}$. If  $\concbmc{\abspathbmc}$ is
satisfiable, the algorithm returns a concrete path $\Pi$, otherwise
the search continues trying to enumerate another abstract path.
The search is guaranteed to terminate, since the length of the
abstract paths is at most $k$.

On the one hand, a possible advantage of the explicit enumeration (2)
over the symbolic BMC (1) is that it can explore the paths of
$\RTSAbstraction{\epsilon}{\System}$ using different heuristics
(e.g. a DFS instead of a BFS), and
that each concretization check is ``simpler'' than a single symbolic
BMC query.
%
On the other hand, the symbolic BMC implicitly exploits SMT solvers,
that automatically prune the search space (and intuitively, also the number
of abstract paths that must be visited) by learning conflict clauses.



\begin{example}
\mypara{\emph{Model Check}}: The graph $\scr{H}(0.1)$ can now be viewed as a
    transition system $\tupleof{C,\x,\scrT,C_0,\HybridStates_0}$,
    where $C \in vertices(\scr{H}(0.1))$ and $\x \in
    \HybridStates$. A transition $\tau \in \scrT$ is of the form
    $\tupleof{C, C',\rho_{\tau}}$, where $\rho_{\tau}(\x,\x')
    \subseteq \setof{R_{(C,C')}(\x,\x') |( C,C') \in
    edge(\scr{H}(0.1))}$.
\mypara{\emph{Check Counter-example}}: The infinite state (but finite location) transition system can
    be model checked to find a concrete counter-example, which if
    exists, can indicate the existence of a similar trace in the
    original black-box system $\System$. The latter check is carried
    out as before, using the numerical simulation function
        $\simulate$~\footnote{For the given example, the model checker is unable to
        find a counter-example.}.

\inclfig[width=0.50\linewidth]{vdp_graph.pdf}{Enriched
graph $G^R$. The affine maps $f$ for the transition relations are show
in \tabref{vdp-pwa}.}{vdp-graph}
\end{example}

%% Temporal properties of PWA transitions systems can be checked using
%% off-the-shelf model checkers like SAL~\cite{SAL-SRI}, which can reason
%% over infinite state transitions systems using SMT solvers.
%% Furthermore, lazy SMT solvers can be employed for this specific
%% problem instance to achieve better efficiency~\cite{shoukry2017smc}.
%% We now show how relational PWA abstractions can be viewed as PWA
%% transition systems, and how the problem of falsification can
%% be answered by a BMC query.

% To clarify:
%  - is there a maximum k that we look at? Where do we guarnatee that?
%
