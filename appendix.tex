Plan of action:
---------------------------------------------

- What is the paper saying/should be saying?
- Simplify notations....pwa relational and k-relational



- Modify ordering of sections: Abstraction -> PWA enrichment
- Simplify notations
- Fold multiple PWA relational/transitions system into one and use the
  transitions system as a semantic interpretation for model checking.
  Can be eliminated if automatons/kripke structures are used instead.
- Fix the `convex-hull' in the definition 2.2 PWA Relational Model.
  Showcase clustering using a benchmark. Add it to future work
  and cite Nimit's  paper. Or, retract it entirely?
- Clearly state that the approach is not a generic data-drive system
  identification technique, but focused on the falsification aspect.
  We only model selective state-space regions: abstract violations.
- De-emphasize OLS (and the process and the cost functions used for
  modeling). Add that the idea hinges on selectively identifying the
  state-space for modeling and not how the models might be created.
  Emphasize again that the requirement is only on the linearity of
  the individual models and hence, OLS is one way. Trade-off between
  efficiency and complexity can be achieved by selecting the `right'
  modeling template.
- Add benchmarks: hybrid with multiple modes.

Summary of the reviews
---------------------------------------------

R1: Like the idea but the presentation and results need improvement

- Clarify that the plant is inherently a hybrid dynamical system
  and in the future (or if time permits) a controller can be added and
  analyzed together as a infinite state transition system.
- Clarify the clustering done.


R2: Seems the most positive, but focuses on PWA identification.

- Clarify the purpose of the approach is to use sub-optimal
  identification techniques but still provide a `reasonable' and
  scalable falsification technique.

R3: Poor benchmarks: use automotive control examples, try
polynomial kernels (with Z3), compare with simulation based
falsification (S-Taliro?).

- Try polynomial kernels on a benchmark?
- Clarify S3CAM was already compared with S-Taliro.

R4: Probably the most negative review.
Sloppy presentation; high level concepts are not well presented
and contributions are not clear. Benchmarks are too simple. Why OLS?
Contributions are not significant/novel.

- Spell out the contributions clearly.
- Use Simulink models.
- Cite papers which learn hybrid automata from i/o traces, and clarify
  the difference.
  \cite{medhat2015framework} and many others
- Clarify that OLS is just one choice out of many.

R5: Probably a borderline.

- Add a motivation section showcasing the contributions.

--
============================================================================
EMSOFT 2017 Reviews for Submission 159
============================================================================

Title: Piece-Wise Relational Models for Falsifying Safety Properties of Hybrid Systems

Authors: Aditya Zutshi, Sergio Mover, Sriram Sankaranarayanan and Jyotirmoy Deshmukh
============================================================================
                            REVIEWER 1
============================================================================


---------------------------------------------------------------------------
Reviewer's Scores
---------------------------------------------------------------------------

               Relevance to EMSOFT (1-4): 3
          Presentation/Readability (1-4): 2
                       Originality (1-4): 4
               Technical Soundness (1-4): 2
Account of prior work and references (1-3): 2


---------------------------------------------------------------------------
Comments
---------------------------------------------------------------------------

This paper is about checking safety properties on continuous models of an
embedded controller's environment. The continuous model is provided as a
black-box function capable of simulating Δ seconds from a given state with
constant input values. From what I can understand: (i) the state-space of
the system is explored by running multiple simulations; (ii) the resulting
states are clustered based on a quantization function; 

(iii) an affine
relation (the convex hull) is calculated to characterize each cluster; 

[TODO: Fix this]

(iv)
linear regression analysis is used to characterize the transitions between
clusters as affine relations, (iv) A finer abstraction is produced by
distinguish trajectories that initially pass through the same (k - 1)
clusters but then end up in different clusters, and (v) safety properties of
the resulting discrete "relational" abstraction are analyzed using SMT-based
bounded model checking. The work does not aim at soundness or completeness,
but rather to find problems in real embedded systems. The technique is a
variation of one proposed at EMSOFT in 2014 ([28]). It is demonstrated on
the academic examples from [28] with mixed results.

   ============
   [TODO: Add an explicit controller+plant benchmark: fuzzy control?]
   ============

I am not an expert in this area, but it seems to me that this paper presents
some novel and interesting ideas. I found, however, that after a great
introduction, the presentation becomes quite frustrating. There are a
succession of different formalisms and notations, which often lack precision
and explanations, and the links between them are not clearly described (I
had to battle to write the above summary and I am not certain that it is
correct). And, even after a careful reading of the text, I am still unsure
about one or two fundamental details:

1. The title states that the paper addresses "hybrid systems" and the system
   model includes discrete states, but thereafter the paper only treats
   continuous systems. Apparently this is reasonable because of an
   assumption that the model is a "switched-mode dynamical system".

   It would be nice to define exactly what you mean by "switched-mode
   dynamical system". Does it mean that only the controller model makes
   discrete transitions and that the environment is a continuous model with
   no discrete events or discontinuities?

   ============
   The systems under test are assumed to be switched-mode to ease the
   presentation, but it is not a limitation of the presented
   technique. Black box systems which have hybrid dynamics due to
   multiple modes, are dealt in the same fashion.
   [TODO: $K$-refinement to differentiate the dynamics of different modes.
   We have added a benchmark to showcase this.
   [TODO: which one?]
   [TODO: clarify the absence of any controllers]
   [TODO: remove or clarify switched-mode systems]
   [TODO: add a hybrid benchmark]
   ============

   I imagine that the idea is to combine the generated discrete abstraction
   with a discrete controller model to verify both together. It would be
   good to state this clearly and to treat the resulting issues throughout
   the paper. I note that the experimental results do not treat this issue.
   Otherwise, why not change the title and focus the text on what you
   actually did?

   ============
   We will explore this in the future, but the current presentation
   only works with the plant, which is itself a hybrid dynamical
   system.
   [TODO: Add this in the conclusion/future work]
   ============


   (Similar remarks hold at a smaller scale. The discussion often describes
    what "can be" done, e.g., in sections 3.2 and 4.1. I would rather read
    about what "has been" done and what the result was!)
  
   ============
   [TODO]
   ============

2. Why do you need a bounded model checker to check safety invariants when
   you already enumerate and characterize the reachable states? That is, why
   not just test the invariant on the convex hulls as they are generated?

   My guess is that you later want to restrict the state-space by composing
   your abstraction with a controller model. Is that right? This probably
   seems obvious to you, but please put yourself in the place of someone who
   is new to the work and not an expert in the domain! In any case, this
   idea is not clearly stated, not addressed by the text, and not considered
   in the experimental evaluation.

   ============
   [TODO: I don't understand this comment, but it is related to
   point (iii) of the summary]
   Although we enumerate the abstract space, we require the model
   checker to find a concrete counter-example.
   ============

   Is this why the PWA transition system characterizes states by a valuation
   function v and not simply as tuples? If yes, how will this work? I do not
   understand why you have a valuation function when the system only has one
   variable (x).
   ============
   [TODO]
   ============

Other comments and suggestions

* The paper puts a certain emphasis on "learning" a model. Would it not be
  more accurate to talk about "approximating" or "abstracting"?
   ============
   [TODO: Seems reasonable to me. Any suggestions??]
   ============

* It would be great to include an "overview figure" that lists all of the
  different formalisms and explains how they fit together. The enumeration
  in section 5 is very helpful, but it introduces even more notation and
  does not completely describe how all of the other graphs and transition
  systems are related.
   ============
   [TODO]
   ============

INTRODUCTION

* "Automatic approaches for testing the safety and performance of such
closed-loop systems is challenging," ->
  "Developing automatic approaches for testing the safety and performance of
such closed-loop systems is challenging," ?

* "Related approaches using bisimulation relations [20, 25, 27] have
interesting parallels with the work presented here.": What does this sentence
mean exactly?

LEARNING PWA RELATIONAL MODELS

* "The system statespace (denoted X) is a subset of $Q × R^n"$
-> "The system statespace, denoted X, is a subset of $Q × R^n"$

* "An input to S is assumed to be a time-bounded signal, i.e., a function
  from [0,T] to U, with T as the time horizon, and $U \subseteq R^k$.": Why not just
  use functions from [0,T] to $R^k$? What do you gain by introduction U (it is
  not exploited in the rest of the paper)?

* "we view S as effectively black-box"
-> "we view S as a black-box"

* "equipped with a forward simulation map, in a simulation time-step"
-> "equipped with a forward simulation map, which in a simulation time-step"

* "allows computing the unique reachable state x ∈ X"
-> "allows computing the unique reachable state x' ∈ X" (?)

* "the forward simulation map $SIMS^S_Δ : X × U × R^+ ↦ X": why R^+$?

* Why is Δ sometimes subscripted with t? Do you use a constant step-size or
  not? The text could be clearer.

* "Also, to ease the presentation, we conflate the inputs with the state and
  denote them by x.": This sentence could be clearer. I think you mean to
  say that you consider that the input values are included in the continuous
  state x. That said, you keep using u at the start of section 2.1.

* "several “proximal” tuples": "proximate"?

* "In statistics, linear regression is the problem of finding"
-> "Linear regression is the problem of finding"

* "For rest of the section, we elide" -> "For the rest of the paper, we
  elide" ?

* "The equation 2 now becomes"
-> "Equation 2 now becomes"

* "In this fashion, we can we can use OLS to approximate its trajectories of
  system S at fixed time step"
-> "In this fashion, we can use OLS to approximate the trajectories of
system S at a fixed time step"

* The definition of a "PWA transition system" is not a numbered definition
  (while the definition of a PWA relation model is). The "move function"
  seems to define the semantics of the transition systems. This could be
  more clear. You use G here for a set of conjunctions of affine predicates,
  and later it is used for graphs.

* "and use the name a variable as a"
-> "and use the name of the variable as a"

* "We how we can use the formal machinery of a PWA transition"?

* "S defined by a simulator SIMS∆~\footnote{...}."
-> "S defined by a simulator SIMS∆.\footnote{...}"

* Definition 2.2 is not precise enough. Why not systematically state the
  values of $L, x, T, G, U, l_0, Θ$? This would aid the poor reader by
  specifying the initial state and the set of locations. What is the set of
  locations?

* "the the valuation function for variable x"
-> "the valuation function for variable x"

* "A PWA model will be usually non-deterministic,"
-> "A PWA model will usually be non-deterministic,"

* "Abusing the notation,"
-> "Abusing notation,"

* "a single step in ρ by x' = ρ(x)": this abuse of notation is regrettable.
  It would be clearer and more precise, I believe, to avoid it.

* "iff for every state $x ∈ X^n": Why "^n"$?

* "In what follows, we use the terms PWA relational model and PWA
   transition system interchangeably to mean PWA relational model.": Please
   choose one term and use it consistently.

* "As all the guards and transitions are affine, a bounded reachability
  query is equivalent to checking all combinations of discrete locations,
  where each combination can be summarized as a linear program.": I did not
  understand what you are trying to say here.

* "Temporal properties of pwa transitions systems"
  -> "Temporal properties of PWA transitions systems" (and elsewhere).

CONNECTION WITH GRAPH-BASED ABSTRACTION

* I found the transition from the previous section very abrupt and it was
  unclear where all this was going.

* "As an alternative, we introduce data driven ‘enrichment’ of abstractions
  to approximate the concrete relations underlying the abstract relations.
  Informally, we use linear regression to compute affine maps associated
  with the relations between two abstract states.": I would say that the
  first statement is informal and the second is formal. I prefer the second
  as it says precisely what you do. "data driven enrichment of abstractions"
  is a great sounding phrase, but I think it is essentially empty.

* "The abstraction of the hybrid dynamical system is a a partitioning"
-> "The abstraction of a hybrid dynamical system is a partitioning"

* "This further defines an equivalence relations"
-> "This further defines an equivalence relation"

* "We call it a time parameterized PWA relational abstraction": Is there
  a real need to introduce more terminology? This name is never used again.

RELATIONAL MODELING EXTENSIONS

* "such a relation can be rarely represented"
-> "such a relation can rarely be represented"

* "Using a tool like FLOW*": missing citation or URL.

* "we rely on statistical methods like simple linear": which other
  statistical methods did you actually use?

* "with the below relation"
-> "with the following relation"

* "C ~ C' ⟺  ∃x ∈ C. ∃x' ∈ C'. x' ∈ Ax + b ± δ": Why do you start
using "±
  δ"? Figure 1, Definition 2.2, and §2.1 just use "+ δ".

* Figure 2 is unnecessary. It does not really show anything.

* "Form a state x ∈ C,"
-> "From a state x ∈ C,"

* "be taken as long as x' ∈ C’ is satisfied."
-> "be taken as long as x' ∈ C' is satisfied."

* "split them in to multiple sets"
-> "split them into multiple sets"

* "PWA transition system which only defines the evolution of states x,"
-> "PWA transition system that only defines the evolution of states x,"

* "trajectory segments form a cell C."
-> "trajectory segments from a cell C."

* "D = {(x, x')|x ∈ C}": Do you not need an extra clause to relate x and x'
  to the model transitions?

* "Figure 5: The data gets split into two sets"
-> "Figure 5: The data is split into two sets"

AN EXAMPLE: VAN DER POL OSCILLATOR

* "The corresponding reachability graph is H(∆ = 0.1).": more notation and
  without a precise definition!

* "which if exists, can indicate the existence of a"
-> "which if it exists, may indicate the existence of a"

* "numerical simulation function SIM ∆~\footnote{...}."
-> "numerical simulation function SIM ∆.\footnote{...}"

EXPERIMENTAL RESULTS

* "Its different parts comprised of the"
-> "Its different parts comprise the"

* "Finally, the The bounded model checking"
-> "Finally, the bounded model checking"

* "due to the differences in which they operate"
-> "due to the differences in how they operate"

* "requires a finer abstraction to begin with in order to minimize the"
-> "modeling errors. On the other hand, a coarser abstraction as the"

* Is table 1 really useful? How is it to be interpreted? What information is
  being conveyed?

* "As apparent form the results,"
-> "As apparent from the results,"

* "but coarse enough to be manageable by the current bmc"
-> "but coarse enough to be manageable by current BMC"

* "during the k-relational modeling process till"
-> "during the k-relational modeling process until"

* "Similarly, small time length trajectory"
-> "Similarly, small time-length trajectory"

REFERENCES

* [6] "Ph.D. Dissertation. Ph. D. thesis," -> "Ph. D. thesis,"

* [22] "(????)": ????

============================================================================
                            REVIEWER 2
============================================================================


---------------------------------------------------------------------------
Reviewer's Scores
---------------------------------------------------------------------------

               Relevance to EMSOFT (1-4): 4
          Presentation/Readability (1-4): 3
                       Originality (1-4): 4
               Technical Soundness (1-4): 3
Account of prior work and references (1-3): 2


---------------------------------------------------------------------------
Comments
---------------------------------------------------------------------------

This submission presents a model based methodology to find falsifications in
black box dynamical systems. After describing approaches based on data driven
modeling, arguing the pros and cons, the authors put forward a framework that
combines ideas from abstraction based search and relational abstractions and
defines the k-relational modeling to compute PWA relational abstractions. To
estimate the local dynamics of the trajectory segments and compute relational
models, which were then interpreted as infinite state discrete transition
system, the proposed framework utilizes simple regression for an
over-determined case. These infinite state discrete transition system
constructions were model checked for safety violations using an off-the-shelf
bounded model checkers and a naive linear programming based specialized model
checker for comparison. The framework was implemented as a tool chain, namely
the PWA-Rel and its performance was demonstrated on a few benchmarks.

The paper is well organized and written and should be accepted. I list next a
few issues that could improve the readability and the overall manuscript:
- The submission is missing an abstract.
-  The following sentence “For this step, we need to have models that can be
digested by existing symbolic tools, which is harder to do with some of the
aforementioned heavier data-driven modeling methods” requires a more in-depth
discussion. Please enrich the discussion especially because the paper starts on
such a negative tone without providing concrete examples. While I do not
disagree with the authors I feel that things could be made clearer with a few
explanatory sentences.

   ============
   [TODO: Can not use/scale generic neural-networks, non-linear
   kernels for regression,...]
   ============
- In section 2.1. the authors state that when N>n “a single affine function
cannot be found”. Should the problem be stated in finding the best choice of
a and b or should the problem considering determining the optimal number of
piece wise affine approximations? Can the magnitude N-n give a hint of what is
a good number of PWA / affine approximations? I feel that this could lead to a
nice results which the authors can easily cast in optimization (2).

   ============
   In general, finding the `optimal' number of PWA models can be
   framed as a multiobjective optimization problem, where a small
   number of clusters and an accurate affine map is desired. This is a
   hard problem for arbitrary non-linear systems~\cite{[TODO]}.
   Hence, in this paper, we presented a scalable falsification approach
   with a sub-optimal PWA identification approach.

   [TODO: OLS is not the main focus of the paper. We should state that
   and mention that any method for learning affine functions can be
   used.]
   ============

- In section 4.1 the authors state that “in the presence of complex
non-linear behavior, using a single affine map can lead to poor
approximation”. In this context and in relation with recent effort in CPS on
modeling complex dynamics (see “Constructing compact causal mathematical
models for complex dynamics” ICCPS2017) can the authors comment or speculate
on how the complexity of the dynamics might influence the required number of
affine maps? How does the non-smoothness and non-differentiability influence
the computation of R(C,C’)? Is there any relationship between the number of
nondifferentiable points and the number of required affine maps? It will also
be enrich the discussion of data-driven modeling techniques with recent efforts
in modeling non-smooth complex dynamics with fractal properties and hint
towards potential extensions.


   ============
   [TODO: ...]
   ============

- Another critical parameter in the framework seems to be the time steps Delta
used in section 4. Can the authors perform a sensitivity analysis to gauge its
impact on overall quality of the results?

   ============
   [TODO: ...]
   ============

- Since the main goal of this technique is to find violations of safety
properties in the embedded control system, it would prove useful for the reader
if the authors could comment which of the evaluations are considering this
specific goal. Again, this is not a criticism, but I feel that the
investigation in section 6 could be described in a few more sentences to
connect with embedded control.
   ============
   [TODO: ...]
   ============
- Lastly, the paper has a few misspellings and grammar issues. I summarize some
I found: “apparent form the results…”, “We how we can use”,
“Finally, the The bounded…”

============================================================================
                            REVIEWER 3
============================================================================


---------------------------------------------------------------------------
Reviewer's Scores
---------------------------------------------------------------------------

               Relevance to EMSOFT (1-4): 3
          Presentation/Readability (1-4): 3
                       Originality (1-4): 2
               Technical Soundness (1-4): 3
Account of prior work and references (1-3): 2


---------------------------------------------------------------------------
Comments
---------------------------------------------------------------------------

The paper presents a method for falsifying safety properties of hybrid systems.
The method first constructs a relational abstraction of the system (which is
considered as a black box) from a set of simulation traces. The relations in
the abstraction are affine with uncertainty. The approach is quite interesting,
however the examples used for demonstration are unfortunately not black boxes,
although these examples are treated as black boxes, since it would be of great
interest to show its usefulness for complex models for which traditional
verification tools cannot be applied. A comparison with S3CAM is interesting,
but is less informative than a comparison with simulation-based validation
tools.
I also recommend using relations more complex than affine, such as polynomial,
for which Z3 can be used (nonlinear relations may reduce the number of
hyper-rectangles).

The method is of heuristic nature and thus convincing experimental results
would be needed. The paper compares with previous work (S3CAM) using a similar
approach. This approach has been tested on more real-life examples from
automative control. I am wondering why there are no comparisons on these
examples.

============================================================================
                            REVIEWER 4
============================================================================


---------------------------------------------------------------------------
Reviewer's Scores
---------------------------------------------------------------------------

               Relevance to EMSOFT (1-4): 3
          Presentation/Readability (1-4): 2
                       Originality (1-4): 3
               Technical Soundness (1-4): 2
Account of prior work and references (1-3): 2


---------------------------------------------------------------------------
Comments
---------------------------------------------------------------------------

The paper presents a model-based methodology to do falsification for hybrid
systems. The technique combines the abstraction based search and relation
abstraction concept to learn a PWA abstract model from the simulations of a
system. The transitions between states of the abstract model characterized by
the affine relations between source and target states. A model checking
technique is then applied to find a counter-example for the abstract model.
This counter-example will be used to guide the search for a counter-example of
the original system. Experimental results of several academic benchmarks are
reported that show the capability of proposed technique in finding false
behavior.

It's clear to the reviewer that the authors may require more time to complete
the paper (e.g., the abstract is missing in the paper. The paper's organization
also needs to be improved. It's not clear for the reviewer to see the
high-level concept and contributions of the paper from the beginning.

Although the author claims that the proposed technique can effectively find a
counter-example for a black-box model, however, the experiment result does not
reflect that statement. Most of the benchmarks are nonlinear continuous systems
that are not black-box. I strongly suggest the author evaluate the proposed
approach on some black-box systems such as industrial Simulink models.

Also, using linear regression to infer affine dynamics and discrete structures
of the abstractions of hybrid systems has been applied in the past. Please
check out the related work on mining hybrid automata from input/output traces

Medhat, Ramy, et al. "A framework for mining hybrid automata from input/output
traces." Embedded Software (EMSOFT), 2015 International Conference on. IEEE,
2015.

Using OLS does not always result in finding the best fit. For example, if some
data points are excessively small or large compared to the rest of data, they
will have disproportionately large effects on the resulting constant. In this
case, using other techniques like ridge regression or lasso regression may
yield a better result.

As the approach is neither sound nor complete and the application perspective
for falsifying black-box system is not adequately demonstrated, it's hard to
assert that the contributions of the proposed approach are novel.

   ============
   [TODO: ...]
   ============


The paper contains some minor typos such as:

Page 2, OLS paragraph: the best choice for a and b, consistency of
capitalization

Page 3, definition 2.2, second bullet: the the valuation function -> delete
one the

Page 3, section 2.2: pwa -> PWA
Page 4, section 3.1, at the first line: a a partitioning of -> delete one
a
Page 7, section 6.1, at the first line: the The bounded model checking ->
delete The
Page 8, section 6.2: bmc -> BMC

============================================================================
                            REVIEWER 5
============================================================================


---------------------------------------------------------------------------
Reviewer's Scores
---------------------------------------------------------------------------

               Relevance to EMSOFT (1-4): 3
          Presentation/Readability (1-4): 3
                       Originality (1-4): 3
               Technical Soundness (1-4): 3
Account of prior work and references (1-3): 3


---------------------------------------------------------------------------
Comments
---------------------------------------------------------------------------

This is sophisticated stuff that, sadly, went over my head. Your motivation is
good and overall, this seems reasonable, but I couldn't easily wrap my head
around the sorts of system for which this technique would work. I suggest
leading with the Van Der Pol oscillator to help ignorant readers such as myself
get a handle on what it is you're actually able to do, then launch into the
formalism. As it stands, I think you need to be fairly expert in hybrid
dynamical systems to even begin to grasp what's going on here.

