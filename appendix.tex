\obeyspaces
\obeylines
======================================
EMSOFT 2017 Reviews for Submission 159
======================================

Title: Piece-Wise Relational Models for Falsifying Safety Properties of Hybrid Systems

=====================================
                            REVIEWER 1
=====================================


----------------------------------
Reviewer's Scores
---------------------------------

               Relevance to EMSOFT (1-4): 3
          Presentation/Readability (1-4): 2
                       Originality (1-4): 4
               Technical Soundness (1-4): 2
Account of prior work and references (1-3): 2


------------------------------------
Comments
-----------------------------------


1. The title states that the paper addresses "hybrid systems" and the system
   model includes discrete states, but thereafter the paper only treats
   continuous systems. Apparently this is reasonable because of an
   assumption that the model is a "switched-mode dynamical system".

   It would be nice to define exactly what you mean by "switched-mode
   dynamical system". Does it mean that only the controller model makes
   discrete transitions and that the environment is a continuous model with
   no discrete events or discontinuities?

>> The systems under test are assumed to be switched-mode to ease the
>> presentation, but it is not a limitation of the presented
>> technique. Black box systems which have hybrid dynamics due to
>> multiple modes, are dealt in the same fashion.

2. Why do you need a bounded model checker to check safety invariants when
   you already enumerate and characterize the reachable states? That is, why
   not just test the invariant on the convex hulls as they are generated?

>> BMC is required to find a concrete counter-example. The invariants
>> found through clustering (which has been since removed due to the
>> impending confusion) can not be used to conclude the presence of a
>> concrete coutner-examples by themselves as they are just an
>> approximation.

   My guess is that you later want to restrict the state-space by composing
   your abstraction with a controller model. Is that right?

>> This is a very important idea which we plan to explore in future
>> work. But it is not being presented in the paper as of now.

======================================
                            REVIEWER 2
======================================


--------------------------------------
Reviewer's Scores
--------------------------------------

               Relevance to EMSOFT (1-4): 4
          Presentation/Readability (1-4): 3
                       Originality (1-4): 4
               Technical Soundness (1-4): 3
Account of prior work and references (1-3): 2


--------------------------------------
Comments
--------------------------------------

- In section 2.1. the authors state that when N>n “a single affine function
cannot be found”. Should the problem be stated in finding the best choice of
a and b or should the problem considering determining the optimal number of
piece wise affine approximations? Can the magnitude N-n give a hint of what is
a good number of PWA / affine approximations? I feel that this could lead to a
nice results which the authors can easily cast in optimization (2).

>> In general, finding the `optimal' number of PWA models can be
>> framed as a multi-objective optimization problem, where a small
>> number of clusters and an accurate affine map is desired. This is a
>> hard problem for arbitrary non-linear systems.
>> Hence, in this paper, we presented a scalable falsification approach
>> with a sub-optimal PWA identification approach.


Can the authors comment or speculate on how the complexity of the
dynamics might influence the required number of affine maps? How does
the non-smoothness and non-differentiability influence the computation
of R(C,C’)? Is there any relationship between the number of
nondifferentiable points and the number of required affine maps?


- Another critical parameter in the framework seems to be the time steps Delta
used in section 4. Can the authors perform a sensitivity analysis to gauge its
impact on overall quality of the results?

>> It is a good idea, and it can have a big affect on the quality of
the results. However, selecting a `good' Delta is not straightforward.

- Since the main goal of this technique is to find violations of safety
properties in the embedded control system, it would prove useful for the reader
if the authors could comment which of the evaluations are considering this
specific goal. Again, this is not a criticism, but I feel that the
investigation in section 6 could be described in a few more sentences to
connect with embedded control.

>> Current examples include generic dynamical systems. But the same
>> procedure can be used to find falsifications in closed loop control
>> systems. We plan to do extensive case studies in the future.


======================================
                            REVIEWER 3
======================================


--------------------------------------
Reviewer's Scores
--------------------------------------

               Relevance to EMSOFT (1-4): 3
          Presentation/Readability (1-4): 3
                       Originality (1-4): 2
               Technical Soundness (1-4): 3
Account of prior work and references (1-3): 2


--------------------------------------
Comments
--------------------------------------

A comparison with S3CAM is interesting, but is less informative than a
comparison with simulation-based validation tools.

>> In earlier works, S3CAM was shown to be better than S-Taliro.
>> Hence, for now only S3CAM was compared. In the future we plan to have
>> extensive case-studies and comparison with other tools.

I also recommend using relations more complex than affine, such as
polynomial, for which Z3 can be used (nonlinear relations may reduce
the number of hyper-rectangles).

>> We tried polynomial kernels but Z3 can not scale even on small
>> number of polynomial constraints. Instead we used non-linear
>> optimization solvers. But, they were very sensitive to the initial
>> guess.


======================================
                            REVIEWER 4
======================================


--------------------------------------
Reviewer's Scores
--------------------------------------

               Relevance to EMSOFT (1-4): 3
          Presentation/Readability (1-4): 2
                       Originality (1-4): 3
               Technical Soundness (1-4): 2
Account of prior work and references (1-3): 2


--------------------------------------
Comments
--------------------------------------

Although the author claims that the proposed technique can effectively find a
counter-example for a black-box model, however, the experiment result does not
reflect that statement. Most of the benchmarks are nonlinear continuous systems
that are not black-box. I strongly suggest the author evaluate the proposed
approach on some black-box systems such as industrial Simulink models.

Using OLS does not always result in finding the best fit. For example, if some
data points are excessively small or large compared to the rest of data, they
will have disproportionately large effects on the resulting constant. In this
case, using other techniques like ridge regression or lasso regression may
yield a better result.

>> As the choice of loss function and type of regression analysis
>> depends on the problem at hand, we show our methodology using the
>> simplest one. In practice, sophisticated statistical tests can be used
>> to find the a good analysis.


======================================
                            REVIEWER 5
======================================


--------------------------------------
Reviewer's Scores
--------------------------------------

               Relevance to EMSOFT (1-4): 3
          Presentation/Readability (1-4): 3
                       Originality (1-4): 3
               Technical Soundness (1-4): 3
Account of prior work and references (1-3): 3


--------------------------------------
Comments
--------------------------------------

I suggest leading with the Van Der Pol oscillator to help ignorant
readers such as myself get a handle on what it is you're actually able
to do, then launch into the formalism.

>> We have made the Van Der Pol oscillator as a running example to improve
>> the presentation.
