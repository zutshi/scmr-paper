In this section, we describe a {\em relational modeling} approach to
abstract a system. The technique described in this section can be
viewed as an enrichment of the procedure to construct an
$\epsilon$-precision transition system from the previous section.  The
central idea of this paper is that a discrete abstraction based on a
transition system can be augmented with additional information which
can greatly improve the accuracy of the abstraction. The refined
discrete abstraction takes the form of a $\epsilon$-precision {\em
relational} transition system $\RTSAbstraction{\epsilon}{\System}$.

\subsection{$\epsilon$-precision Relational Transition System}

Let $\exprs{\vx}$ denote a set of Boolean-valued expressions over the
variable $\vx$ and their Boolean combinations.  For example, if $\vx =
\setof{x_1,x_2}$, then $\exprs{\vx}$ could be $\setof{ x_1 > 0, x^2_1
+ x_2 > 0, x_2 < 1 \wedge x_1 + x_2 > -1}$.  An $\epsilon$-precision
Relational Transition System $\RTSAbstraction{\epsilon}{\System}$ is a
labeled transition system described by the tuple:
$(\Cells,\vx,\CellEdges,\InitCells,\CellLabelingFunction,
\EdgeLabelingFunction,\UnsafeStates)$, where:

\begin{itemize}[label=--,leftmargin=1em,labelsep=*]
\item
$\Cells$ is a set of cells,
\item
$\vx$ represents a variable that takes values in $\ContStates$, 
\item
$\CellEdges$ is a finite subset of $\Cells \times \Cells$, 
\item
$\InitCells$ is a set of intial cells, 
\item
$\CellLabelingFunction$ is a function mapping a cell $C$ to a
an expression $e(\vx) \in \exprs{\vx}$,
\item
$\EdgeLabelingFunction$ maps an edge $(C,C')$ to a pair $(g(\vx),
r(\vx,\vxp))$, where the {\em guard relation} $g(\vx) \in \exprs{\vx}$
and the {\em reset relation} $r(\vx,\vxp) \in \exprs{\vx,\vxp}$, and
$\vxp$ is a special variable.
\item
$\UnsafeStates \subset \ContStates$ is a set of unsafe states.
\end{itemize}


The operation of $\RTSAbstraction{\epsilon}{\System}$ is described in
terms of its configuration and its moves. A configuration is a pair
$(C, \nu)$, where $C \in \Cells$, and $\nu$ is a valuation function
assigning values in $\ContStates$ to the variable $\vx$.  The
transition system starts in some cell $C \in \InitCells$.  The move
function $\mu$ maps a configuration $(C,\nu)$ to the next
configuration $(C',\nu')$, iff there is a transition $(C,C') \in
\CellEdges$ such that:
\begin{enumerate}
\item
There exists $\contState \in \ContStates$ such that
$\quant_\epsilon(\contState) = C$ and exists $\contState' \in
\ContStates$ such that $\quant_\epsilon(\contState') = C'$,
and $\contState' = \simmap_\Delta(\contState, t, u)$,
\item
If $\CellLabelingFunction(C) =  e(\vx)$, and if
$\CellLabelingFunction(C') = e'(\vx)$, then for $\nu(\vx) =
\contState$ and $\nu(\vx') = \contState'$, the expressions
$e(\nu(\vx))$ and $e'(\nu'(\vx))$ evaluate to true, 
\item
If $\EdgeLabelingFunction((C,C'))$ = $(g(\vx), r(\vx,\vxp))$, 
for $\nu(\vx) = \contState$ and $\nu'(\vxp) = \contState'$, 
$g(\nu(\vx))$ and $r(\nu(\vx), \nu'(\vx))$ evaluate to true.
\end{enumerate}

\begin{example}
Consider a system with the following two reachability relations
discovered using simulations. The values of the inputs are irrelevant
to this example, so we just assume it is some fixed constant $u$.
Also, let $\Delta = 0.1$.
\begin{eqnarray}
\label{eq:transitions}
(0.05, 0.1) & = & \simmap_\Delta((0.1, 0.1), 0, u), \\
(1.35,1.15) & = & \simmap_\Delta((0.9, 0.8), 0.1, u).
\end{eqnarray}

Consider the following $\RTSAbstraction{\epsilon}{\System}$ where,
$\Cells = \setof{C_1,C_2}$, $\vx = (x_1,x_2)$, $\InitCells = \setof{C_1}$,
$\CellEdges = \setof{(C_1,C_1), (C_1, C_2)}$, and,
\[ 
\begin{array}{l}
\CellLabelingFunction(C_1) = (0 \le x_1 < 1) \wedge (0 \le x_2 < 1), \\
\CellLabelingFunction(C_2) = (1 \le x_1 < 2) \wedge (1 \le x_2 < 2), \\
\EdgeLabelingFunction((C_1,C_1)) = (g_1,r_1), \text{ where, } \\
\qquad g_1 = (0 \le x_1 < 0.5) \wedge (0 \le x_2 < 1), \\
\qquad r_1 = (0 < x^+_1 - 0.5x_1 < 0.1) \wedge (0 < x^+_2 - x_2 < 0.001), \\
\EdgeLabelingFunction((C_1,C_2)) = (g_2,r_2), \text{ where, } \\
\qquad g_2 = (0.5 \le x_1 < 1) \wedge (0 \le x_2 < 1), \\
\qquad r_2 = (0.4 < x^+_1 - x_1 < 0.5) \wedge (0.3 < x^+_2 - x_2 < 0.4). \\
\end{array}
\]

Observe that the $\epsilon$-precision relation transition system is a
valid abstraction of the given trajectory fragments in
Eq.~\ref{eq:transitions}.
\end{example}




% We define the PWA transition system as a discrete transition system
% $\rho:\tupleof{\pLocs,\vx,\scrT, \pLocI,\Theta}$, where $\pLocs$ is
% a nonempty set of locations, $\vx$ is a set of variables, $\scrT$
% is a finite set of transitions of the form $\tupleof{l, \tau,
% l'}$, where $l\in\pLocs$ is the pre-location, $l'\in\pLocs$ the
% post-location and $\tau \in \scrT$ is the transition. $\tau$
% is labeled by the transition relation $\rho_{\tau}$
% over the current states and the next states. The transition relation
% is defined in terms of affine predicates on the current and next
% states and an affine update between them.
% \[
%     \rho_{\tau}(\vx,\vx') \subseteq \setof{(\vx,\vx')) \;|\;
%     g_\tau(\vx) \land g'_\tau(\vx') \land f_{\tau}(\vx, \vx')}
% \]
% where $f_{\tau}$ is an affine relation on the pre and post states
% $\vx,\vx'$, $g_{\tau}$ and $g'_{\tau}$ are affine guards (pre and
% post conditions) on the states. Finally, $\pLocI$ is an initial
% location $\Theta$ is an affine assertion characterizing the initial
% states included in $\pLocI$.


% A configuration of the PWATS is
% a pair (l,\nu(v)), where \nu is a valuation function assigning values to
% variables.

% A run of the PWA TS is as follows: the PWA TS starts in the initial
% location from a valuation \nu(v) \in theta.

% A move of the PWA TS is described by the function mu: from (L,R^n) to
% (L,R^n), and mu follows the rules that
% mu(l,\nu(v)) = (l',\nu'(v)), iff (l,g,g',f,l') in T, and
% \nu(v) satisfies g, \nu(v') satisfies g', and \nu(v') = f(\nu(v)).

% We how we can use the formal machinery of a PWA transition system $A$
% to approximate the behavior of a hybrid dynamical system $\System$
% defined by a simulator $\simulate^{\System}$~\footnote{Due to $f_\tau$
% being restricted to an affine form, we can only approximate general
% hybrid systems.}. 


%defines a set valued relation of the form $R:\setof{(\x,\x') \;|\; \x'
%\in f_\tau(\x)}$.


% Given a state vector $\x\in\reals^n$, a guarded affine map
% $\gm:(\guard, \amap)$ defines the discretized consecution rule as a
% pair of an affine guard predicate $\guard:C\x - \vd \le 0$ and an affine map
% $\amap:A\x+\vb$, where $A \in \reals^{n \times n}$, $C \in
% \reals^{m \times n}$ are matrices and $\vb \in \reals^n$, $\vd
% \in \reals^m$ are vectors. A guarded affine map is satisfied if its
% guard is satisfied.

% We now formalize the PWA relational model for a dynamical system.
% \begin{definition}[PWA Relational Model]
% 
% A PWA relational model $A(\System)$ of a hybrid dynamical system
% $\System$ is simply a PWA transition system, where:
% 
% \begin{itemize}[noitemsep, leftmargin= 1.5 em]
% \item
% each transition of $A$ represents a discrete time step $\Delta$ of the
% system $\System$,
% \item
% if the state-space of $\System$ is $\reals^n$, the the valuation
% function for variable $\x$ in $A$ assigns $\x$ some value in
% $\reals^n$,
% \item
% for the $N$ tuples $(v,v')$ discovered by simulation, if $v' = Av
% + B + \vec{\delta}$ represents the affine relation discovered using
% linear regression, and if $h$ is the convex hull of all the $v$
% states across all tuples and $h'$ is the convex hull of all the $v'$
% states across all tuples, then $f_\tau \subseteq
% \setof{(\x,\x') \mid \x' \in A\x + B + \vec{\delta}}$,
% and $g$ states that $\x$ is in $h$, and $g'$ states that
% $\x' \in h'$.
% \end{itemize}
% 
% Note that the guard predicates $g_\tau$ represent a conjunction of $m$
% affine inequalities, then we can represent $g_\tau$ as follows:
% \[
% g_\tau  \equiv C\x - \vd \le 0
% \]
% where $C$ is an $m \times n$ matrix and $\vd$ is $m \times 1$ matrix.


% Given a hybrid dynamical system over a state-space $\reals^n$, a PWA
% relational model $\rho$ is given by the tuple
% $\tupleof{\pLocs,\x,\scrT,\pLocI,\Theta}$, where $\tau \in \scrT$ is a
% collection of affine transitions  and $\Theta$ is an affine predicate
% over $\x\in\real^n$. The transition relation is then defined by
% $\scrT$ with $n$ transition relations as follows
% 
% \begin{equation}
%     \scrT = \left\{
%         \begin{array}{ll}
%             g_1(\x) \land g'_1(\x') \implies f_1(\x,\x') \\
%             \ldots\\
%             g_n(\x) \land g'_n(\x') \implies f_n(\x,\x') \\
%         \end{array}
%     \right.
% \end{equation}
% \end{definition}

% where $g_i$ and $g'_i$ are affine predicates and $f_i$ are affine
% relations and $\x'$ denotes the next state of the system.

% A PWA relational model is \textit{deterministic}, iff for every state
% $\x\in\reals^n$, a unique guarded affine map is satisfied and if all
% entries in the error interval vector $\vec{\delta}$ are singletons.
% However, in practice such a case rarely exists. A PWA model will be
% usually non-deterministic, both due to multiple choices of transitions
% from any location and the reachable states at the $k^{th}$ time step
% (or after $k\Delta$ units of time) being a set.  Abusing the notation,
% we denote the set of states reachable in a single step in $\rho$ by
% $\x' = \rho(\x)$.
% 
% A PWA relational model system is \textit{complete}, iff for every
% state $\x\in\HybridStates^n$, there exists at least one satisfied
% transition relation $\scrT$. If this is not the case, the system can
% deadlock, with no further executions. This usually results from
% modeling errors, and from here on, we do not consider such cases.
% 
% In what follows, we use the terms PWA relational model and PWA
% transition system interchangeably to mean PWA relational model.
% 
\subsection{Bounded Model Checking (BMC)}
The PWA transition system is a finite location, infinite state
transition system. As all the guards and transitions are affine, a
bounded reachability query is equivalent to checking all
combinations of discrete locations, where each combination can be
summarized as a linear program. As the time is bounded, the number of
combinations are finite, but, exponential in number.

It should be noted that an explicit execution of the transition system
represents a directed acyclic graph, with each location being a node,
and a directed edge from each node to all nodes reachable in forward
time. A path on this graph is then a linear program, with each
branching node corresponding to a logical disjunction.

Temporal properties of PWA transitions systems can be checked using
off-the-shelf model checkers like SAL~\cite{SAL-SRI}, which can reason
over infinite state transitions systems using SMT solvers.
Furthermore, lazy SMT solvers can be employed for this specific
problem instance to achieve better efficiency~\cite{shoukry2017smc}.
We now show how relational PWA abstractions can be viewed as PWA
transition systems, and how the problem of falsification can
be answered by a BMC query.

