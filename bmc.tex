\section{Model Checking}

% include an algorithm
We now describe the entire procedure in three steps.
\begin{enumerate}
    \item Given a black box system $\System$ specified by
        $\simulate^\System$, we
        use scatter-and-simulate with ($\Delta$-length) trajectory
        segments to sample a finite reachability graph
        $\scr{H}(\Delta)$.
    \item Using regression, we enrich the graph relations and annotate
        the edges with the estimated affine maps.
    \item Interpreting the directed reachability graph as a transition
        system, we use a bounded model checker to find fixed length
        violations to safety properties.
\end{enumerate}


\section{An Example: Van der Pol Oscillator}

We now describe the above using the Van der Pol oscillator benchmark
presented in \cite{zutshi2014multiple}. Simulations generated using
uniformly random samples are shown in \figref{vdp-cont}. We want to
check the property $P_3$, indicated by the red box, given the initial
set indicated by the green box. The abstraction is defined by
$\quant_{0.2}(\x)$, which results in an evenly gridded state-space,
where each cell is of size $0.2 \times 0.2$ units.
Scatter-and-simulate is then used to construct the abstract graph $G$,
using $2$ samples per cell and the time step $\Delta = 0.1s$. The
complete process follows.

\begin{enumerate}
    \item{\emph{Abstract}} Consider an implicit abstraction induced by the quantization
    function $\quant_\epsilon(\x)$. The corresponding reachability
    graph $\scr{H}(\Delta)$ (with time step $\Delta$) is given by
    $\tupleof{\scr{C},\areach{\Delta}, \scr{C}_0, \scr{C}_u}$.
\item{\emph{Discover}}: Using scatter-and-simulate, enumerate a finite number
    of cells (vertices) and the relations (edges) of the graph
    $\scr{H}(\Delta)$. Associate the set of generated trajectory
    segments with their respective originating cells using a map $D:C
    \mapsto \setof{\pi | start(\pi) \in C}$. The discovered
    abstraction is show in \figref{vdp-abs}. As mentioned, red cells
    are unsafe and green cells are initial cells.
\item{\emph{Relationalize}} For each cell $C$, perform regression analysis on
    the respective set of trajectory segments $D(C)$, and compute a
    set of affine relations $R_{C,C'}(\x,\x')$ between $\x \in C$ and $\x'
    \in C'$ s.t., $\text{edge} (C,C') \in \scr{H}(\Delta)$. Annotate
    each edge in the graph $\scr{H}(\Delta)$ with the respective
    relation.

    \figref{vdp-map} shows the cells and the trajectory segments which
    are part of the data sets constructed using the $1$-relational
    modeling. Against each cell, its unique identifier (integer
    co-ordinate) is mentioned.  Finally, \figref{vdp-graph} and
    \tabref{vdp-pwa} show the enriched graph $G^R$ with its transition
    relations. Note that self loops result when an observed trajectory
    segment has its $start$ and $end$ states in the same cell.

\item{\emph{Model Check}} The graph $\scr{H}(\Delta)$ can now be viewed as a
    transition system $\tupleof{C,\x,\scrT,C_0,\HybridStates_0}$,
    where $C \in vertices(\scr{H}(\Delta))$ and $\x \in
    \HybridStates$. A transition $\tau \in \scrT$ is of the form
    $\tupleof{C, C',\rho_{\tau}}$, where $\rho_{\tau}(\x,\x')
    \subseteq \setof{R_{(C,C')}(\x,\x') |( C,C') \in
    edge(\scr{H}(\Delta))}$.
\item{\emph{Check Counter-example}} The infinite state (but finite location) transition system can
    be model checked to find a concrete counter-example, which if
    exists, can indicate the existence of a similar trace in the
    original black-box system $\System$. The latter check is carried
    out as before, using the numerical simulation function
    $\simulate$. For the given example, the model checker is unable to
    find a counter-example.
\end{enumerate}

\inclfig[width=0.75\linewidth]{vdp_cont_traces.png}{Van der
Pol: continuous trajectories. Red and green boxes indicate unsafe and
initial sets.}
\label{fig:vdp-cont}

\inclfig[width=0.75\linewidth]{vdp_abs.png}{The discovered
abstraction $\scr{H}(0.1)$. Red cells are unsafe cells and green cells are
initial cells.}
\label{fig:vdp-abs}

\inclfig[width=\linewidth]{vdp_disc_map.png}{Cells and
trajectory segments used by $1$-relational modeling.}
\label{fig:vdp-map}

\inclfig[width=0.40\linewidth]{vdp_graph.png}{Enriched
graph $G^R$. The affine maps $f$ for the transition relations are show
in \tabref{vdp-pwa}.}
\label{fig:vdp-graph}

\begin{table}[!htbp]
\centering
\caption{PWA model computed using OLS. The affine model for each edge $(C,C')$ in the graph \figref{vdp-graph} is given by $x'\in Ax + b + \delta$, where $\delta$ is a vector of intervals.}
\label{tab:vdp-pwa}
\begin{tabular}{@{}ccccc@{}}
\toprule
$C$ & $C'$ & $A$ & $b$ & $\delta$\\
\midrule
(0, -1)   & (0, -1)   & $\begin{bmatrix}0.99& 0.12 \\-0.12&1.63\end{bmatrix}$&$\begin{bmatrix}0\\0\end{bmatrix}  $&$\begin{bmatrix}[0, 0]\\ [0, 0]\end{bmatrix}$\\
(0, -1)   & (0, -2)   & $\begin{bmatrix}0.99& 0.12 \\-0.10&1.63\end{bmatrix}$&$\begin{bmatrix}0\\0\end{bmatrix}  $&$\begin{bmatrix}[0, 0]\\ [0, 0]\end{bmatrix}$\\
\midrule
(0, -2)   & (0, -2)   & $\begin{bmatrix}0.99& 0.12 \\-0.09&1.63\end{bmatrix}$&$\begin{bmatrix}0\\0\end{bmatrix}  $&$\begin{bmatrix}[0, 0]\\ [0, 0]\end{bmatrix}$\\
(0, -2)   & (0, -4)   & $\begin{bmatrix}0.99& 0.12 \\-0.07&1.62\end{bmatrix}$&$\begin{bmatrix}0\\0\end{bmatrix}  $&$\begin{bmatrix}[0, 0]\\ [0, 0]\end{bmatrix}$\\
\midrule
(0, -9)   & (-1, -15) & $\begin{bmatrix}0.99& 0.12 \\-0.09&1.61\end{bmatrix}$&$\begin{bmatrix}0\\-0.04\end{bmatrix} $&$\begin{bmatrix}[0, 0]\\ [0, 0.01]\end{bmatrix}$\\
\midrule
(-1, -15) & (-3, -24) & $\begin{bmatrix}0.95& 0.12 \\-1.29&1.47\end{bmatrix}$&$\begin{bmatrix}-0.01\\-0.41\end{bmatrix}$&$\begin{bmatrix}[0, 0]\\ [0, 0.01]\end{bmatrix}$\\
\midrule
(-3, -24) & (-4, -29) & $\begin{bmatrix}1.11&-0.17 \\-1.71&0.66\end{bmatrix}$&$\begin{bmatrix}-1.07\\-3.31\end{bmatrix}$&$\begin{bmatrix}[-0.04, 0.04]\\ [-0.07, 0.09]\end{bmatrix}$\\
\midrule
(-4, -29) & (-4, -29) & $\begin{bmatrix}0.99& 0.01 \\-0.41&1.02\end{bmatrix}$&$\begin{bmatrix}0\\-0.28\end{bmatrix} $&$\begin{bmatrix}[0, 0]\\ [0, 0]\end{bmatrix}$\\
\midrule
(-1, -10) & (-2, -16) & $\begin{bmatrix}0.97& 0.12 \\-0.71&1.56\end{bmatrix}$&$\begin{bmatrix}0\\-0.11\end{bmatrix} $&$\begin{bmatrix}[0, 0]\\ [0, 0.01]\end{bmatrix}$\\
(-1, -10) & (-2, -15) & $\begin{bmatrix}0.97& 0.12 \\-0.70&1.58\end{bmatrix}$&$\begin{bmatrix}0\\-0.08\end{bmatrix} $&$\begin{bmatrix}[0, 0]\\ [0, 0.01]\end{bmatrix}$\\
\midrule
(-2, -16) & (-3, -23) & $\begin{bmatrix}0.92& 0.12 \\-1.84&1.39\end{bmatrix}$&$\begin{bmatrix}-0.02\\-0.73\end{bmatrix}$&$\begin{bmatrix}[0, 0]\\ [0, 0]\end{bmatrix}$\\
\midrule
(-3, -23) & (-5, -29) & $\begin{bmatrix}3.14&-0.81 \\-0.96&0.23\end{bmatrix}$&$\begin{bmatrix}-3.13\\-4.99\end{bmatrix}$&$\begin{bmatrix}[-0.04, 0.03]\\ [-0.03, 0.04]\end{bmatrix}$\\
\midrule
(-2, -15) & (-3, -22) & $\begin{bmatrix}0.93& 0.12 \\-1.75&1.40\end{bmatrix}$&$\begin{bmatrix}-0.02\\-0.68\end{bmatrix}$&$\begin{bmatrix}[0, 0]\\ [0, 0]\end{bmatrix}$\\
\midrule
(-3, -22) & (-5, -29) & $\begin{bmatrix}3.08&-0.72 \\-1.69&0.40\end{bmatrix}$&$\begin{bmatrix}-2.77\\-4.57\end{bmatrix}$&$\begin{bmatrix}[-0.02, 0.02]\\ [-0.01, 0.02]\end{bmatrix}$\\
\bottomrule
\end{tabular}
\end{table}



% (0, -1)   & (0, -1)   &  \begin{bmatrix}0.9944498 & 0.12911959 \\-0.12013468&1.63250531\end{bmatrix} &\begin{bmatrix}-0.00003683\\-0.00085411\end{bmatrix}&\begin{bmatrix}[-0.0, 0.0]\\ [-0.0, 0.0]\end{bmatrix}\\
% (0, -1)   & (0, -2)   &  \begin{bmatrix}0.99523928& 0.1290447  \\-0.10236736&1.6314103 \end{bmatrix} &\begin{bmatrix}-0.00013019\\-0.00289192\end{bmatrix}&\begin{bmatrix}[-0.0, 0.0]\\ [-0.0, 0.0]\end{bmatrix}\\
% \midrule
% (0, -2)   & (0, -2)   &  \begin{bmatrix}0.99565348& 0.12916002 \\-0.09352462&1.6341424 \end{bmatrix} &\begin{bmatrix}-0.00015574\\-0.00339664\end{bmatrix}&\begin{bmatrix}[-0.0, 0.0]\\ [-0.0, 0.0]\end{bmatrix}\\
% (0, -2)   & (0, -4)   &  \begin{bmatrix}0.99668599& 0.12892455 \\-0.0725035 &1.62937613\end{bmatrix} &\begin{bmatrix}-0.00037391\\-0.00796189\end{bmatrix}&\begin{bmatrix}[-0.0, 0.0]\\ [-0.0, 0.0]\end{bmatrix}\\
% \midrule
% (0, -9)   & (-1, -15) &  \begin{bmatrix}0.99985362& 0.12858788 \\-0.09447884&1.61476417\end{bmatrix} &\begin{bmatrix}-0.0018997 \\-0.04200363\end{bmatrix}&\begin{bmatrix}[-0.0, 0.0]\\ [-0.0, 0.01]\end{bmatrix}\\
% \midrule
% (-1, -15) & (-3, -24) &  \begin{bmatrix}0.95473217& 0.12473489 \\-1.29198505&1.47155614\end{bmatrix} &\begin{bmatrix}-0.01248412\\-0.41100129\end{bmatrix}&\begin{bmatrix}[-0.0, 0.0]\\ [-0.0, 0.01]\end{bmatrix}\\
% \midrule
% (-3, -24) & (-4, -29) &  \begin{bmatrix}1.11789749&-0.17261206 \\-1.71017779&0.66006597\end{bmatrix} &\begin{bmatrix}-1.0745344 \\-3.31246863\end{bmatrix}&\begin{bmatrix}[-0.04, 0.04]\\ [-0.07, 0.09]\end{bmatrix}\\
% \midrule
% (-4, -29) & (-4, -29) &  \begin{bmatrix}0.99796588& 0.01012817 \\-0.41529747&1.02432211\end{bmatrix} &\begin{bmatrix}-0.00135199\\-0.28016983\end{bmatrix}&\begin{bmatrix}[-0.0, 0.0]\\ [-0.0, 0.0]\end{bmatrix}\\
% \midrule
% (-1, -10) & (-2, -16) &  \begin{bmatrix}0.97381953& 0.12768367 \\-0.71242741&1.56764585\end{bmatrix} &\begin{bmatrix}-0.00320265\\-0.11964102\end{bmatrix}&\begin{bmatrix}[-0.0, 0.0]\\ [-0.0, 0.01]\end{bmatrix}\\
% (-1, -10) & (-2, -15) &  \begin{bmatrix}0.97373058& 0.12859163 \\-0.70173855&1.58540872\end{bmatrix} &\begin{bmatrix}-0.0015415 \\-0.08616376\end{bmatrix}&\begin{bmatrix}[-0.0, 0.0]\\ [-0.0, 0.01]\end{bmatrix}\\
% \midrule
% (-2, -16) & (-3, -23) &  \begin{bmatrix}0.92850123& 0.12242812 \\-1.84720528&1.39159826\end{bmatrix} &\begin{bmatrix}-0.02334236\\-0.73348865\end{bmatrix}&\begin{bmatrix}[-0.0, 0.0]\\ [-0.0, 0.0]\end{bmatrix}\\
% \midrule
% (-3, -23) & (-5, -29) &  \begin{bmatrix}3.14947585&-0.81150646 \\-0.96159406&0.23543254\end{bmatrix} &\begin{bmatrix}-3.13860326\\-4.99696623\end{bmatrix}&\begin{bmatrix}[-0.04, 0.03]\\ [-0.03, 0.04]\end{bmatrix}\\
% \midrule
% (-2, -15) & (-3, -22) &  \begin{bmatrix}0.93156623& 0.12260264 \\-1.75117219&1.40104348\end{bmatrix} &\begin{bmatrix}-0.02214125\\-0.68423569\end{bmatrix}&\begin{bmatrix}[-0.0, 0.0]\\ [-0.0, 0.0]\end{bmatrix}\\
% \midrule
% (-3, -22) & (-5, -29) &  \begin{bmatrix}3.08806462&-0.72332776 \\-1.69271079&0.40308293\end{bmatrix} &\begin{bmatrix}-2.77908467\\-4.57411339\end{bmatrix}&\begin{bmatrix}[-0.02, 0.02]\\ [-0.01, 0.02]\end{bmatrix}\\






\subsection{Search Parameters}

The search parameters for S3CAM-R include the parameters of S3CAM:
$N$, $\epsilon$ and $\Delta$. Additionally, they also include the
maximum error budget $\delta_{max}$ for OLS and the maximum length of
segmented trajectory for building $k$-relational models.

We have already discussed the effects of $N$, $\epsilon$ and $\Delta$
on S3CAM's performance. However, they also have an effect on
relational modeling. A finer grid with small cells produces more
accurate models than a coarser grid with large cells. Similarly,
small time length trajectory segments result in more accurate
modeling.
%An increase in the number of samples can increases the
%accuracy of regression.

\paragraph{Maximum Model Error ($\boldsymbol{\delta}_{\bold{max}}$) .}
Given a $\delta_{max}$, we keep increasing $k$ during the
$k$-relational modeling process till $\delta_i \le \delta_{max}$ is
satisfied. A $k_{max}$ is introduced to bound the longest
segmented trajectory which can be considered.

% \paragraph{Bounded k relations ($\boldsymbol{k}_{\bold{max}}$.) }

\subsection{Reasons for Failure}
The approach can fail to find a counter-example, even when it exists, in
one of three ways.

\begin{itemize}

    \item \emph{S3CAM Fails} No abstract counter-example is found by
        S3CAM. We can remedy it by increasing search budgets and/or
        restarting.

    \item \emph{BMC Fails} An abstract counter-example is found, but
        the BMC fails to find a concrete counter-example in the PWA
        relational model. The failure can be attributed to either (a)
        a spurious abstract counter-examples or (b) a poorly estimated
        model. The former can be addressed by restarting but the
        latter requires that the maximum model error $\delta_{max}$ be
        decreased.

    \item \emph{Inaccurate PWA Modeling} An abstract counter-example
        is found, and it is successfully concretized in the PWA
        relational model by the BMC. However, it is not reproducible
        in the black box system. This happens when the PWA relational
        model is not precise enough.

\end{itemize}
