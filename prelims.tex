In this section, we present the model of the hybrid dynamical system
under test, and the background for PWA relational abstractions.

\mypara{System under test ($\System$)} 

We assume that $\System$ is a hybrid dynamical system with exogenous
inputs modeling an embedded control system. The system state-space
(denoted $\HybridStates$) is a subset of $\Modes \times \Reals^n$,
where $\Modes$ is a finite set of discrete modes, and $\Reals^n$
represents the continuous-valued state-space. For ease of exposition,
we assume that the hybrid system is effectively a switched-mode
dynamical system, \ie, the mode of the system is completely determined
by its state.  An input to $\System$ is assumed to be a time-bounded
{\em signal}, \ie, a function from $[0,T]$ to $\Inputs$, with $T$ as
the time horizon, and $\Inputs \subseteq \Reals^k$.  In this paper, we
view $\System$ as effectively black-box equipped with a forward
simulation map, in a simulation time-step of $\Delta_t$ from time $t$,
allows computing the unique reachable state $\vx \in \HybridStates$
given state-input pair $(\vx,\vu)$.  

% time and input parameterized reachability relation $\reach{t,
% \vu}_\System \subseteq \HybridStates \times \HybridStates$ as follows:
% $\simulate^{\System}:
% (\vx, \vu, t) \mapsto \vx$, where $\vx \in \HybridStates$, $\vu \in
% \Inputs$ and $t \in [0,T]$.


% Given the hybrid state space $\vx\in\HybridStates$ and input space $u
% \in\Inputs$ and time step $t$, it computes the unique successor state
% $\vy=\simulate^{\System}(\vx,u,t)$ such that $\vx \reach{t,u} \vy$.


\begin{definition}[System Evolution]
    The behavior of system $\System$ at time $t$ is described as an
    evolution in the hybrid state-space $\HybridStates$ under the
    effect of the input $\vu \in \Inputs$ at time $t$. The forward
    simulation map $\simulate^\System : \HybridStates \times \Inputs
    \times \Reals^+ \mapsto \HybridStates$, maps a state $\x$ at time
    $t$ to a new state $\x'$ at time $t+\Delta_t$, under a constant
    input $\vu$.  Finally, we define the time and input parameterized
    reachability relation $\reach{t,u} \subseteq \HybridStates \times
    \HybridStates$ such that $(\vx,\vx') \in \reach{t,u}$ iff
    $\vx'=\simulate^{\System}(\vx,u,t)$.
\end{definition}

\mypara{System Assumptions:} We require that $\System$ can always be
simulated forward in time with deterministic results. This in turn
requires existence and uniqueness of trajectories over a finite time
horizon $[0,T]$, which can be guaranteed by Lipschitz continuity of
the vector field in each underlying hybrid mode, ruling away issues
such as finite escape times~\cite{Meiss/2007/Differential}, and having
well-defined semantics for simulations at switching boundaries. While
these assumptions hold for the dynamical system examples which we
consider in this paper, such assumptions can be restrictive for
general control system models. When analyzing a system that does not
obey theoretical conditions that allow deterministic simulation, we
assume that the underlying tool used for numerical approximations of
system behaviors uses deterministic simulation semantics.  Finally, we
assume full observability of the system state, validate our results
only against the $\simulate$ function (as the analytic closed-form
representation of the system dynamics is assumed to be unavailable).
From here on, we drop the subscript $\System$ whenever it is clear
from the context.

\subsection{Learning Dynamics using Simple Linear Regression}

The approach in \cite{zutshi2014multiple} uses the forward simulation
map to explore a fixed number of tuples in the reachability relation
$\reach{t,\vu}$ specified by a user-provided budget.  In general, the
budget required to adequately explore the transition space requires
exploring an exponential (in the dimension of $\HybridStates$) number
of such tuples. This can cause undue burden in terms of computation
time as well as storage.  The central idea in this paper is to {\em
summarize} several tuples as a single affine map obtained using linear
regression.  In other words, we can replace several ``proximal''
tuples $(\vx,\vx')$ in $\reach{t,u}$ by an affine map of the form
$\vx' = A\vx + B\vu + \delta$, where $A$ and $B$ are matrices and
$\delta$ is a worst-case error estimate.

In statistics, linear regression is the problem of finding an affine
function or \textit{predictor}, which can `best' summarize the
relationship between the given set of observed input $\x$ and output
$\x'$ vectors. The notion of `best' is formally captured using a loss
function which can be non-linear. We use the commonly used loss
function \textit{ordinary least squares} (OLS) for this presentation.

\mypara{Ordinary Least Squares (OLS)}
For rest of the section, we elide the discrete mode from the hybrid
state of the system, and only focus on the continuous state. Note that
we can do this because of our assumption that the system is a
switched-mode dynamical system.  Suppose we have $N$ tuples of the
form $(\x,\x')$ in the reachability relation that we have explored,
where $\x\in\reals^n$ and $\x'\in\reals^n$. Then, we wish to learn an
affine function described as 
\begin{equation}
\label{eq:affinemap}
\x' = A\x + B
\end{equation}
Here, $A$ is a $n\times n$ matrix, and $B$ is a $n\times 1$ matrix.


If $N>n$, which is the case in the current context of \textit{finding
the best fit}, the problem is over-determined: there are more
equations than unknowns. Hence, a single affine function cannot be
found which satisfies the equation \eqref{eq:affinemap}. Instead, we
need to find the `best' choice for $\vec{a}$ and $b$. This choice is
formally defined using a loss function. For the case of simple linear
regression or OLS, the loss function is the sum of squares of the
errors in prediction.  The task is then to determine $A$ and $B$, such
that the least square error of the affine predictor is minimized for
the given data set.

\begin{equation}
    \operatornamewithlimits{argmin}_{A,B}\displaystyle\sum_{i=1}^{N}{\left(\x' - (A\x + B)\right)^2}
\label{reg}
\end{equation}

To ease the presentation, we can rewrite the above as a homogeneous
expression by using $\hat{\x} = [\x^T\ 1]^T$ and replacing $A$ and $B$ by the vector
$\vec{A} = [A\ B]$. The equation~\ref{reg} now becomes
\[
\operatornamewithlimits{argmin}_{\vec{A}}\displaystyle\sum_{i=1}^{N}{(\x' - \vec{A}^T \hat{\x})^2}
\]

The solution of OLS can be analytically computed as the Moore-Penrose
pseudo-inverse: $\vec{A} = (X^TX)^{-1}X^T X'$, where $X$ is the matrix
representing the horizontal stacking of all $\hat{\x}$ and $X'$
represents the stacking of all $\x'$ values.  The details can be found
in standard texts on learning and statistics
\cite{friedman2001elements}. Alternative methods based on gradient
descent can also be used to find a suitable $A$ matrix.

In this fashion, we can we can use OLS to approximate its trajectories
of system $\System$ at fixed time step $\Delta$ by a discrete map $\x'
= \vec{A}\hat{\x}$.

% where
% $A = \begin{bmatrix}\vec{a}_1^T \\ \vdots \\ \vec{a}_n^T
% \end{bmatrix}$ and $\vb = \begin{bmatrix}b_1 \\ \vdots \\ b_n
% \end{bmatrix}$. 

We note that linear regression tools are often able to provide an
$n\times 1$ vector $\vec{\delta}$ of error intervals, where the
$i^{th}$ entry of the form $[{\delta_i}_{\min},{\delta_i}_{\max}]$
indicates the best and worst possible error in computation of the
$i^{th}$ state in $\x'$.


Affine maps are poor approximations for arbitrary non-linear
functions. Hence, we use a collection of affine maps to approximate
the local behaviors (in state-space) of the system $\System$. This
results in a piece wise approximation, as described in the next
section.


% \mypara{Regression Analysis}

% In statistics, regression is the problem of finding a
% \textit{predictor}, which can suitably predict the relationship
% between the given set of observed input $\x$ and output $\y$ vectors.
% In other words, assuming that $\y$ depends on $\x$, regression
% strategies find either a parameterized or a non-parameterized
% prediction function to explain the dependence. We now discuss simple
% linear regression, which is parametric in nature and searches for an
% affine predictor. It is also called \textit{ordinary least squares}.

% \mypara{Ordinary Least Squares (OLS)}

% Let the data set be comprised of $N$ input and output pairs $(\x,
% y)$, where $\x\in\reals^n$ and $y\in\reals$. If $N>n$, which is
% the case in the current context of \textit{finding the best fit}, the
% problem is over-determined; there are more equations than
% unknowns. Hence, a single affine function cannot be found which
% satisfies the equation $\forall i\in\{1\ldots N\}. y_i = \vec{a}^T\x_i + b$. Instead, we
% need to find the `best' choice for $\vec{a}$ and $b$. This is formally
% defined using a loss function. For the case of simple linear
% regression or OLS, the loss function is the sum of squares of the
% errors in prediction.  The task is then to determine the vector of
% coefficients $\vec{a}$ and an offset constant $b$, such that the least
% square error of the affine predictor is minimized for the given data
% set.
% \begin{equation}
%     \operatornamewithlimits{argmin}_{\vec{a}, b}\displaystyle\sum_{i=1}^{N}{\left(\y_i - (\vec{a}^T\x_i + b)\right)^2}
% \label{reg}
% \end{equation}
% To ease the presentation, we can rewrite the above as a homogeneous
% expression by augmenting $\x$ by $\hat{\x} = \begin{bmatrix}\x \\ 1\end{bmatrix}$ and replacing $\vec{a}$ and $b$ by the vector
% $\vec{ab} = \begin{bmatrix}\vec{a} \\ b \end{bmatrix}$. The equation~\ref{reg} now becomes
% \[
%     \operatornamewithlimits{argmin}_{\vec{ab}}\displaystyle\sum_{i=1}^{N}{(\y_i - \vec{ab}^T \hat{\x}_i)^2}
% \]
% The solution of OLS can be analytically computed as
% \[ Ab = (X^TX)^{-1}X^T\y\]
% where $X$ is the matrix representing the horizontal stacking of all
% $\hat{\x}$.
% The details can be found in several texts on learning and statistics
% \cite{friedman2001elements}.

% Given a time invariant dynamical system $\x' = \simulate(\x, \Delta)$,
% where $\x\in\reals^n$ we can use OLS to approximate its trajectories
% at fixed time step $\Delta$ by a discrete map $\x' = A\x + \vb$, where
% $A = \begin{bmatrix}\vec{a}_1^T \\ \vdots \\ \vec{a}_n^T \end{bmatrix}$ and $\vb =
% \begin{bmatrix}b_1 \\ \vdots \\ b_n \end{bmatrix}$. This map is a
%     global relational model for the system. The data set for the
%     regression is a set of pairs $\setof{(\x,\x') | \x' =
%     \simulate(\x, \Delta)}$ and can be generated on demand. Another
%     set can be generated to compute the error $\delta$ as an interval
%     of vectors, where each element is an interval $\delta_i \in
%     [\delta_{min}, \delta_{max}]$.
%
%     Affine maps are poor
%     approximations for arbitrary non-linear functions. Hence, we use a
%     collection of affine maps to approximate the local behaviors (in
%     state-space) of the system $\simulate$. This results in a piece
%     wise approximation, as described in the next section.



% \subsection{Relational Abstractions}
% For a hybrid automaton model, they can summarize the continuous
% dynamics of each mode using a binary relation over continuous states.
% The resulting relations are timeless (independent) and hence valid for
% all time as long as the mode invariant is satisfied. The relations
% take the general form of $R(\x,\x') \bowtie 0$, where $\bowtie$
% represents one of the relational operators $=, \ge, \le, <, >$. For
% example, an abstraction that captures the monotonicity with respect to
% time for the differential equation $\dot{x} = 2$ is $x' > x$. The
% abstraction capturing the relation between the set of ODEs: $\dot{x} =
% 2$, $\dot{y} = 5$, is $5(x' - x) = 2(y' - y)$.

%%  Similar to timeless relational abstractions, time-aware relational
%%  abstractions \cite{mover2013time} construct binary relations between
%%  the current state $\x$ of the system and any future reachable state
%%  $\x'$. The difference lies with the latter also constructing relations
%%  between the current time $t$ and any future time $t'$. This is
%%  achieved by a case by case analysis of the eigen structure of the
%%  matrix $A$ (of an affine ODE). Separate abstractions are used for the
%%  case of linear systems with constant rate, real eigen values and
%%  complex eigenvalues.


\subsection{Piecewise Affine (PWA) Transition System}

We define the PWA transition system as a transition system
$\rho:\tupleof{\pLocs,\pVars,\scrT, \pLocI,\Theta}$, where $\pLocs$ is
a nonempty set of locations, $\pVars$ is a set of variables, $\scrT$
is a finite subset of $\pLocs \times \pLocs$, and each transition
$\tau \in \scrT$ is associated with a transition relation of the form
\[
    \rho_{\tau}(\pVar,\pVar') \subseteq \setof{(\pVar,\pVar')) \;|\;
    g_\tau(\pVar) \land g'_\tau(\pVar') \land f_{\tau}(\pVar, \pVar')}
\]
where $f_{\tau}$ is an affine relation on the pre and post states
$\pVar,\pVar'$, $g_{\tau}$ and $g'_{\tau}$ are affine guards (pre and
post conditions) on the states. Finally, $\pLocI$ is an initial
location $\Theta$ is an affine assertion characterizing the initial
states included in $\pLocI$.

We can now use a PWA transition system $\rho$ to approximate the
behavior of a hybrid dynamical system $\System$ defined by a simulator
$\simulate^{\System}$~\footnote{Due to $f_\tau$ being restricted to an affine
form, we can only approximate general hybrid systems.}. Each
transition of $\rho$ represents a discrete time step $\Delta$. Let the
state-space of $\System$ be given by $\x\in\HybridStates^n$, then the
states of $\rho$ are also given by $\x\in\HybridStates^n$. The affine guard
predicates $g_\tau$ are given by a conjunction of hyper-planes and
hence can be represented in matrix form as a polyhedron (defined by
$m$ constraints) in the state-space
\[
g:C\x - \vd \le 0
\]
where $C$ is an $m \times n$ matrix and $\vd$ is a vector of length
$m$. The affine relation $f_\tau$ can be represented using matrix $A$
and an offset vector $\vb$ as $f_\tau \subseteq \setof{(\x,\x') | \x'
= A\x+\vb}$. Finaly, we incorporte an approximate non-deterministic
error term $\delta$, defined as a vector of intervals
$[\delta_{\min},\delta_{\max}]$.
\[
    f_\tau \subseteq \setof{(\x,\x') | \x' = A\x+\vb}.
\]
%defines a set valued relation of the form $R:\setof{(\x,\x') \;|\; \x'
%\in f_\tau(\x)}$.


% Given a state vector $\x\in\reals^n$, a guarded affine map
% $\gm:(\guard, \amap)$ defines the discretized consecution rule as a
% pair of an affine guard predicate $\guard:C\x - \vd \le 0$ and an affine map
% $\amap:A\x+\vb$, where $A \in \reals^{n \times n}$, $C \in
% \reals^{m \times n}$ are matrices and $\vb \in \reals^n$, $\vd
% \in \reals^m$ are vectors. A guarded affine map is satisfied if its
% guard is satisfied.

We now formalize the PWA relational model for a dynamical system.
\begin{definition}[PWA Relational Model]

Given a hybrid dynamical system over a state-space $\reals^n$, a PWA
relational model $\rho$ is given by the tuple
$\tupleof{\pLocs,\x,\scrT,\pLocI,\Theta}$, where $\tau \in \scrT$ is a
collection of affine transitions  and $\Theta$ is an affine predicate
over $\x\in\real^n$. The transition relation is then defined by
$\scrT$ with $n$ transition relations as follows

\begin{equation}
    \scrT = \left\{
        \begin{array}{ll}
            g_1(\x) \land g'_1(\x') \implies f_1(\x,\x') \\
            \ldots\\
            g_n(\x) \land g'_n(\x') \implies f_n(\x,\x') \\
        \end{array}
    \right.
\end{equation}
\end{definition}

where $g_i$ and $g'_i$ are affine predicates and $f_i$ are affine
relations and $\x'$ denotes the next state of the system.

A PWA model is \textit{deterministic}, iff for every state
$\x\in\reals^n$, a unique guarded affine map is satisfied and if all
entries in the error interval vector $\vec{\delta}$ are singletons.
However, in practice such a case rarely exists. A PWA model will be
usually non-deterministic, both due to multiple choices of transitions
from any location and the reachable states at the $k^{th}$ time step
(or after $k\Delta$ units of time) being a set.  Abusing the notation,
we denote the set of states reachable in a single step in $\rho$ by
$\x' = \rho(\x)$.

A PWA transition system is \textit{complete}, iff for every state
$\x\in\HybridStates^n$, there exists at least one satisfied
transition relation $\scrT$. If this is not the case, the system can
deadlock, with no further executions. This usually results from
modeling errors, and from here on, we do not consider such
cases.

\subsection{Bounded Model Checking (BMC)}
The PWA transition system is a finite location, infinite state
transition system. As all the guards and transitions are affine, a
bounded reachability query is equivalent to checking all
combinaions of discrete locations, where each combination can be
summarized as a linear program. As the time is bounded, the number of
combinations are finite, but, exponential in number.

It should be noted that an explicit execution of the transition system
represents a directed acyclic graph, with each location being a node,
and a directed edge from each node to all nodes reachable in forward
time. A path on this graph is then a linear program, with each
branching node corresponsing to a logical disjunction.

Temporal properties of pwa transitions systems can be checked using
off-the-shelf model checkers like SAL~\cite{SAL-SRI}, which can reason
over infinite state transitions systems using SMT solvers.
Furthermore, lazy SMT solvers can be employed for this specific
problem instance to achieve better efficiency~\cite{shoukry2017smc}.
We now show how relational pwa abstractions can be viewed as pwa
transition systems, and how the problem of falsification can
be answered by a BMC query.

% Given a dynamical system over a continuous state-space
% $\ContStates\in\reals^n$, a PWA model is a map $\ContStates \mapsto
% \setof{\gm_1, \gm_2, \ldots, \gm_n}$ from the state-space of the
% dynamical system to a finite set of guarded affine maps
% $\setof{\gm_1, \gm_2, \ldots, \gm_n}$. It defines the consecution
% rule by the affine map of the satisfied $\gm$.

% \begin{equation}
%     \pwa = \left\{
%         \begin{array}{ll}
%             \gm_1: \amap_1(\x) &\text{if}\; \guard_1(\x) \\
%             \gm_2: \amap_2(\x) &\text{if}\; \guard_2(\x) \\
%             \ldots & \ldots\\
%             \gm_n: \amap_n(\x) &\text{if}\; \guard_n(\x) \\
%         \end{array}
%     \right.
% \end{equation}

% \end{definition}

% where $\amap_i = A_i\x + \vb_i$, and $\guard_i(\x) \equiv C_i\x-\vd_i\le0$.
% Abusing the notation, we denote the next state computed by the PWA
% model as $\x' = \pwa(\x)$. A PWA model is \textit{deterministic}, iff
% for every state $\x\in\ContStates$, a unique guarded affine map is
% satisfied. A PWA model is \textit{complete}, iff for every state
% $\x\in\ContStates$, there exists at least one satisfied guarded affine
% map $\gm$.

