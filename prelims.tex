In this section, we present the model of the hybrid dynamical system
under test and the basics of pwa relational abstraction.

We now use the family of reachability relations presented
in~\cite{zutshi2014multiple} to charecterize the system $\System$
under test. We denote the states of the hybrid system $\System$, by $\HybridStates \subseteq \Modes \times \Reals^n$ be the (infinite)
set of hybrid states of the system $\System$, where $\Modes$ is a
finite set of discrete modes, and $\Inputs$ be the set of input
signals to $\System$ of the form $[0,T] \rightarrow \reals^k$ for some
given time horizon $T$. The family of
relations $\reach{t,u} \subseteq \HybridStates \times \HybridStates$,
parameterized by time $t \geq 0$ and input signal $u \in \Inputs$
defined over time $[0,T]$.  We use the notation $\vx\ \reach{t,u} \vy$
to denote $(\vx,\vy) \in \reach{t,u}$. The relations $\reach{t,u}$
satisfy the following basic properties:


We assume a black-box model
equipped with a forward simulation map $\simulate^{\System}$ that can
be used to compute the discrete time traces of the system. Let
$\reach{}$ Given the
hybrid state space $\vx\in\HybridStates$ and input space $u
\in\Inputs$ and time step $\Delta_t$, it computes the unique successor
state $\vy=\simulate^{\System}(\vx,u,t)$ such that $\vx \reach{t,u}
\vy$.

\begin{definition}

    The system under test $\System$ evolves in the set of infinite
    hybrid state-space $\HybridStates$ under the effect of inputs $\vu
    \in \Inputs$. Its evolution is defined by a simulation function
    $\simulate^\System : \HybridStates \times \Inputs \Reals^+ \mapsto
    \HybridStates$, where $\simulate^\System(\x,\vu,\Delta_t)$ maps a
    current state $\x$ at time $t$ to a new state $\x'$ at time
    $t+\Delta_t$, under a constant input $\vu$.

\end{definition}

The subscript $\System$ will be dropped whenever the system $\System$
being referred to is clear from the context.


\myipara{Black-Box Assumptions:} For above properties of the
$\reach{t,u}$ relation to hold, we require that $\System$ be
deterministic. We also assume that $\System$ can be simulated from a
given initial state using function $\simulate^{\System}$. This
requires existence and uniqueness of trajectories over a finite time
horizon $[0,T]$, which can be guaranteed by Lipschitz continuity of
the vector field in each hybrid mode and ruling away issues such as
finite escape times~\cite{Meiss/2007/Differential}. From a practical
standpoint, these assumptions do not pose significant restrictions for
the type of problems that we wish to address.  Finally, we assume full
observability of the system state, ignore numerical errors due to
simulation and {\em do not} assume that an analytic closed-form
representation of the system dynamics is available.

\subsection{Learning Dynamics using Simple Linear Regression}

In statistics, linear regression is the problem of finding an affine
function or \textit{predictor}, which can `best' summarize the
relationship between the given set of observed input $\x$ and output
$\y$ vectors. The notion of `best' is formally captured using a loss
function which can be non-linear. We use the commonly used loss
function \textit{ordinary least squares} (OLS) for this presentation.

\mypara{Ordinary Least Squares (OLS)}

The
Let the data set be comprised of $N$ input and output pairs $(\x, y)$,
where $\x\in\reals^n$ and $y\in\reals$. If $N>n$, which is the case in
the current context of \textit{finding the best fit}, the problem is
over-determined; there are more equations than unknowns. Hence, a
single affine function cannot be found which satisfies the equation
$\forall i\in\{1\ldots N\}. y_i = \vec{a}^T\x_i + b$. Instead, we need
to find the `best' choice for $\vec{a}$ and $b$. This is formally
defined using a loss function. For the case of simple linear
regression or OLS, the loss function is the sum of squares of the
errors in prediction.  The task is then to determine the vector of
coefficients $\vec{a}$ and an offset constant $b$, such that the least
square error of the affine predictor is minimized for the given data
set.
\begin{equation}
    \operatornamewithlimits{argmin}_{\vec{a}, b}\displaystyle\sum_{i=1}^{N}{\left(\y_i - (\vec{a}^T\x_i + b)\right)^2}
\label{reg}
\end{equation}
To ease the presentation, we can rewrite the above as a homogeneous
expression by augmenting $\x$ by $\hat{\x} = \begin{bmatrix}\x \\ 1\end{bmatrix}$ and replacing $\vec{a}$ and $b$ by the vector
$\vec{ab} = \begin{bmatrix}\vec{a} \\ b \end{bmatrix}$. The equation~\ref{reg} now becomes
\[
    \operatornamewithlimits{argmin}_{\vec{ab}}\displaystyle\sum_{i=1}^{N}{(\y_i - \vec{ab}^T \hat{\x}_i)^2}
\]
The solution of OLS can be analytically computed as
\[ Ab = (X^TX)^{-1}X^T\y\]
where $X$ is the matrix representing the horizontal stacking of all
$\hat{\x}$.
The details can be found in several texts on learning and statistics
\cite{friedman2001elements}.

Given a time invariant dynamical system $\x' = \simulate(\x, \Delta)$,
where $\x\in\reals^n$ we can use OLS to approximate its trajectories
at fixed time step $\Delta$ by a discrete map $\x' = A\x + \vb$, where
$A = \begin{bmatrix}\vec{a}_1^T \\ \vdots \\ \vec{a}_n^T \end{bmatrix}$ and $\vb =
\begin{bmatrix}b_1 \\ \vdots \\ b_n \end{bmatrix}$. This map is a
    global relational model for the system. The data set for the
    regression is a set of pairs $\setof{(\x,\x') | \x' =
    \simulate(\x, \Delta)}$ and can be generated on demand. Another
    set can be generated to compute the error $\delta$ as an interval
    of vectors, where each element is an interval $\delta_i \in
    [\delta_{min}, \delta_{max}]$.

    Affine maps are poor
    approximations for arbitrary non-linear functions. Hence, we use a
    collection of affine maps to approximate the local behaviors (in
    state-space) of the system $\simulate$. This results in a piece
    wise approximation, as described in the next section.


% \mypara{Regression Analysis}

% In statistics, regression is the problem of finding a
% \textit{predictor}, which can suitably predict the relationship
% between the given set of observed input $\x$ and output $\y$ vectors.
% In other words, assuming that $\y$ depends on $\x$, regression
% strategies find either a parameterized or a non-parameterized
% prediction function to explain the dependence. We now discuss simple
% linear regression, which is parametric in nature and searches for an
% affine predictor. It is also called \textit{ordinary least squares}.

% \mypara{Ordinary Least Squares (OLS)}

% Let the data set be comprised of $N$ input and output pairs $(\x,
% y)$, where $\x\in\reals^n$ and $y\in\reals$. If $N>n$, which is
% the case in the current context of \textit{finding the best fit}, the
% problem is over-determined; there are more equations than
% unknowns. Hence, a single affine function cannot be found which
% satisfies the equation $\forall i\in\{1\ldots N\}. y_i = \vec{a}^T\x_i + b$. Instead, we
% need to find the `best' choice for $\vec{a}$ and $b$. This is formally
% defined using a loss function. For the case of simple linear
% regression or OLS, the loss function is the sum of squares of the
% errors in prediction.  The task is then to determine the vector of
% coefficients $\vec{a}$ and an offset constant $b$, such that the least
% square error of the affine predictor is minimized for the given data
% set.
% \begin{equation}
%     \operatornamewithlimits{argmin}_{\vec{a}, b}\displaystyle\sum_{i=1}^{N}{\left(\y_i - (\vec{a}^T\x_i + b)\right)^2}
% \label{reg}
% \end{equation}
% To ease the presentation, we can rewrite the above as a homogeneous
% expression by augmenting $\x$ by $\hat{\x} = \begin{bmatrix}\x \\ 1\end{bmatrix}$ and replacing $\vec{a}$ and $b$ by the vector
% $\vec{ab} = \begin{bmatrix}\vec{a} \\ b \end{bmatrix}$. The equation~\ref{reg} now becomes
% \[
%     \operatornamewithlimits{argmin}_{\vec{ab}}\displaystyle\sum_{i=1}^{N}{(\y_i - \vec{ab}^T \hat{\x}_i)^2}
% \]
% The solution of OLS can be analytically computed as
% \[ Ab = (X^TX)^{-1}X^T\y\]
% where $X$ is the matrix representing the horizontal stacking of all
% $\hat{\x}$.
% The details can be found in several texts on learning and statistics
% \cite{friedman2001elements}.

% Given a time invariant dynamical system $\x' = \simulate(\x, \Delta)$,
% where $\x\in\reals^n$ we can use OLS to approximate its trajectories
% at fixed time step $\Delta$ by a discrete map $\x' = A\x + \vb$, where
% $A = \begin{bmatrix}\vec{a}_1^T \\ \vdots \\ \vec{a}_n^T \end{bmatrix}$ and $\vb =
% \begin{bmatrix}b_1 \\ \vdots \\ b_n \end{bmatrix}$. This map is a
%     global relational model for the system. The data set for the
%     regression is a set of pairs $\setof{(\x,\x') | \x' =
%     \simulate(\x, \Delta)}$ and can be generated on demand. Another
%     set can be generated to compute the error $\delta$ as an interval
%     of vectors, where each element is an interval $\delta_i \in
%     [\delta_{min}, \delta_{max}]$.
%
%     Affine maps are poor
%     approximations for arbitrary non-linear functions. Hence, we use a
%     collection of affine maps to approximate the local behaviors (in
%     state-space) of the system $\simulate$. This results in a piece
%     wise approximation, as described in the next section.



\subsection{Relational Abstractions}
For a hybrid automaton model, they can summarize the continuous
dynamics of each mode using a binary relation over continuous states.
The resulting relations are timeless (independent) and hence valid for
all time as long as the mode invariant is satisfied. The relations
take the general form of $R(\x,\x') \bowtie 0$, where $\bowtie$
represents one of the relational operators $=, \ge, \le, <, >$. For
example, an abstraction that captures the monotonicity with respect to
time for the differential equation $\dot{x} = 2$ is $x' > x$. The
abstraction capturing the relation between the set of ODEs: $\dot{x} =
2$, $\dot{y} = 5$, is $5(x' - x) = 2(y' - y)$.

%%  Similar to timeless relational abstractions, time-aware relational
%%  abstractions \cite{mover2013time} construct binary relations between
%%  the current state $\x$ of the system and any future reachable state
%%  $\x'$. The difference lies with the latter also constructing relations
%%  between the current time $t$ and any future time $t'$. This is
%%  achieved by a case by case analysis of the eigen structure of the
%%  matrix $A$ (of an affine ODE). Separate abstractions are used for the
%%  case of linear systems with constant rate, real eigen values and
%%  complex eigenvalues.


\subsection{Piecewise Affine (PWA) Transition System}

We define the PWA transition system as a transition system
%(ref.~\secref{DiscSys})
$\rho:\tupleof{\pLocs,\pVars,\scrT, \pLocI,\Theta}$
where each transition $\tau \in \scrT$, is associated with a
transition relation of the form
\[
    \rho_{\tau}(\pVar,\pVar') \subseteq \setof{(\pVar,f_{\tau}(\pVar)) \;|\; g_\tau(\pVar) \land g'_\tau(f_{\tau}(\pVar))}
\]
where $f_{\tau}$ is an affine map relating the states and $g_{\tau}$,
$g'_{\tau}$ are affine guards (pre and post conditions) on the states.
The assertion on the initial states $\Theta$ is also affine.

We can now use a PWA transition system $\rho$ to approximate the
behavior of a hybrid dynamical system $\System$ defined by a simulator
$\simulate^{\System}$~\footnote{Due to $f_\tau$ being restricted to an affine
form, we can only approximate general hybrid systems.}. Each
transition of $\rho$ represents a discrete time step $\Delta$. Let the
state-space of $\System$ be given by $\x\in\HybridStates^n$, then the
states of $\rho$ are also given by $\x\in\HybridStates^n$. The affine guard
predicates $g_\tau$ are given by a conjunction of hyper-planes and
hence can be represented in matrix form as a polyhedron (defined by
$m$ constraints) in the state-space
\[
g:C\x - \vd \le 0
\]
where $C$ is an $m \times n$ matrix and $\vd$ is a vector of length
$m$. The affine maps $f_\tau$ can be represented in matrix form as
$f:\x \mapsto A\x+\vb$. To make $f_\tau$ `richer', we incorporate
an error term $\delta$ which is defined by a vector of intervals
$[\delta_{\min},\delta_{\max}]$. Hence,
\[
f:\x \mapsto A\x + \vb + \delta
\]
defines a set valued relation of the form $R:\setof{(\x,\x') \;|\; \x'
\in f_\tau(\x)}$.


% Given a state vector $\x\in\reals^n$, a guarded affine map
% $\gm:(\guard, \amap)$ defines the discretized consecution rule as a
% pair of an affine guard predicate $\guard:C\x - \vd \le 0$ and an affine map
% $\amap:A\x+\vb$, where $A \in \reals^{n \times n}$, $C \in
% \reals^{m \times n}$ are matrices and $\vb \in \reals^n$, $\vd
% \in \reals^m$ are vectors. A guarded affine map is satisfied if its
% guard is satisfied.

We now formalize the PWA affine transition system for a dynamical system.
\begin{definition}[PWA Transition System]

Given a hybrid dynamical system over a state-space
$\HybridStates^n$, a PWA transition system $\rho$ is given by the
tuple $\tupleof{\pLocs,\x,\scrT,\pLocI,\Theta}$, where $\tau \in
\scrT$ are affine transition relations and $\Theta$ is an affine
predicate over $\x$ and the states $\x\in\HybridStates^n$. The
transition relation is then defined by $\scrT$ with $n$ transition
relations as follows

\begin{equation}
    \scrT = \left\{
        \begin{array}{ll}
            g_1(\x) \land g'_1(\x') \implies \x' \in f_1(\x) \\
            \ldots\\
            g_n(\x) \land g'_n(\x') \implies \x' \in f_n(\x) \\
        \end{array}
    \right.
\end{equation}

\end{definition}

where $f_i:\x \mapsto A_i\x + \vb_i + \delta_i$,
$g_i(\x):C_i\x-\vd_i\le0$, $g'_i(\x'):C'_i\x'-\vd'_i\le0$ and $\x'$
denotes the next state of the system.

A PWA model is \textit{deterministic}, iff for every state
$\x\in\HybridStates^n$, a unique guarded affine map is satisfied and
all errors $\delta$ are singletons. However, in practice such a case
rarely exists. A PWA model will be usually non-deterministic, both due
to multiple choice of transition relations and the reachable states at
the $k^{th}$ time step (or after $k\Delta$ units of time) being a set.
Abusing the notation, we denote the set of states reachable in a
single step in $\rho$ by $\x' = \rho(\x)$.

A PWA transition system is \textit{complete}, iff for every state
$\x\in\HybridStates^n$, there exists at least one satisfied
transition relation $\scrT$. If this is not the case, the system can
deadlock, with no further executions. This usually results from
incorrect modeling of SDCS, and from here on, we do not consider such
cases.

% Given a dynamical system over a continuous state-space
% $\ContStates\in\reals^n$, a PWA model is a map $\ContStates \mapsto
% \setof{\gm_1, \gm_2, \ldots, \gm_n}$ from the state-space of the
% dynamical system to a finite set of guarded affine maps
% $\setof{\gm_1, \gm_2, \ldots, \gm_n}$. It defines the consecution
% rule by the affine map of the satisfied $\gm$.

% \begin{equation}
%     \pwa = \left\{
%         \begin{array}{ll}
%             \gm_1: \amap_1(\x) &\text{if}\; \guard_1(\x) \\
%             \gm_2: \amap_2(\x) &\text{if}\; \guard_2(\x) \\
%             \ldots & \ldots\\
%             \gm_n: \amap_n(\x) &\text{if}\; \guard_n(\x) \\
%         \end{array}
%     \right.
% \end{equation}

% \end{definition}

% where $\amap_i = A_i\x + \vb_i$, and $\guard_i(\x) \equiv C_i\x-\vd_i\le0$.
% Abusing the notation, we denote the next state computed by the PWA
% model as $\x' = \pwa(\x)$. A PWA model is \textit{deterministic}, iff
% for every state $\x\in\ContStates$, a unique guarded affine map is
% satisfied. A PWA model is \textit{complete}, iff for every state
% $\x\in\ContStates$, there exists at least one satisfied guarded affine
% map $\gm$.

\subsection{Learning Affine Function from Trajectory Segments}

The $\simulate$ function can be used to generate a data set of
relations parameterized by time $\Delta_t$, for different values of
states $\x_i$ and inputs $\vu_i$ as $\ds_{\Delta_t} =
\setof{(\x_i,\x'_i,\vu_i) | \x'_i = \simulate(\x_i,\vu_i)}$.
Moreover, by repeatedly applying $\simulate$, a discrete time series
of a finite length $N$ can be generated from a given state $\x_1$
and a sequence of inputs $\tupleof{\vu_1, \vu_2,\ldots,\vu_n}$ as shown
below.
\[\tupleof{\x_1, \x_2, \ldots, \x_n} \;\text{where}\; \x_{i>1} =
\simulate(\x_{i-1}, \vu_{i-1}, \Delta_t)\]

We now introduce a refinement of the data set $\ds \subseteq \ds$ by
incrementally observing the future states in the time series.



\[\ds_{\Delta_t} = \setof{(\x_i,\x'_i,\vu_i) | \x'_i =
\simulate(\x_i,\vu_i)}\]

\[\ds^{C_i} \subseteq \ds_t\]
\[\ds^{C_i}\]


$k$-Refinement of the data set $\ds^k$.
$\ds_k = \setof{(\x,\x',\vu) | \x' =  \simulate(\x,\vu,k\Delta_t)}$

state-input pairs of the form
$(\x,\x')$
data set equivalence traj segment
$D_c, $
