\documentclass[9pt,sigconf]{acmart}

\usepackage{balance}
% \usepackage{cite}

\usepackage{tikz}
\usetikzlibrary{arrows,backgrounds,decorations,decorations.pathmorphing,positioning,fit,automata,shapes,snakes,patterns}


\makeatletter
\newif\if@restonecol
\makeatother
\let\algorithm\relax
\let\endalgorithm\relax

\usepackage{enumitem}
\usepackage{times}
\usepackage{graphicx}
\usepackage{subfig}
\usepackage{amsmath,amssymb,amsfonts}
\usepackage{fancyhdr}
\usepackage{balance}
% \usepackage{cite}
\usepackage{color}
\usepackage{wrapfig}
\usepackage[ruled,vlined,linesnumbered]{algorithm2e}
\usepackage{booktabs}
\usepackage{xcolor,colortbl}
\usepackage{times}
\usepackage{microtype}
\usepackage{tikz}
\usepackage{url}
%\usepackage{tablefootnote}

% \usepackage{hyperref}
% \hypersetup{
%     colorlinks=false,
%     pdfborder={0 0 0},
% }
\usetikzlibrary{arrows,backgrounds,positioning,fit,automata,shapes,snakes,patterns}
%%\usepackage{abstract}

% \newtheorem{example}{Example}[section]
% \newtheorem{definition}{Definition}[section]
% \newtheorem{lemma}{Lemma}[section]
% \newtheorem{theorem}{Theorem}[section]

\begin{document}

%\conferenceinfo{...}{BD}

% \title{Multiple Shooting, CEGAR-based \\ Falsification for Hybrid
% Systems}


\title{Relational Models for Falsifying Safety Properties of Hybrid Systems}


% Generic
%\DeclareMathAlphabet{\mathpzc}{OT1}{pzc}{m}{it}

% English
\newcommand{\ie}{{i.e.}\xspace}
\newcommand{\Ie}{{I.e.}\xspace}
\newcommand{\eg}{{e.g.}\xspace}
\newcommand{\etc}{{etc.}\xspace}
\newcommand{\viz}{{viz.\xspace}}
\newcommand{\etal}{{et al.}\xspace}

\renewcommand\vec[1]{\mathbf{#1}}
% Generic refs
\newcommand{\lemref}[1]{Lemma~\ref{lem:#1}}
\newcommand{\secref}[1]{Sec.~\ref{sec:#1}}
\newcommand{\figref}[1]{Fig.~\ref{fig:#1}}
\newcommand{\exref}[1]{Example~\ref{ex:#1}}
\newcommand{\thmref}[1]{Theorem~\ref{thm:#1}}
\newcommand{\tabref}[1]{Table~\ref{tab:#1}}
\newcommand{\algoref}[1]{Algorithm~\ref{algo:#1}}
\newcommand{\lstref}[1]{Listing~\ref{lst:#1}}
\newcommand{\chapref}[1]{Chapter~\ref{chap:#1}}
\newcommand{\defref}[1]{Definition~\ref{def:#1}}

% Comments, Reviewing, Formatting
\newcommand{\ignore}[1]{}
\newcommand\todo[1]{[[\textcolor{red}{\textsf{#1}}]]}
% General Math
\newcommand{\setof}[1]{\ensuremath{\{#1\}}}
\newcommand\tupleof[1]{\ensuremath\left\langle #1 \right \rangle}
\newcommand{\sublist}[2]{\ensuremath{#1_{1},\ldots,#1_{#2}}}
\newcommand{\suplist}[2]{\ensuremath{#1^{1},\ldots,#1^{#2}}}
\newcommand \card[1] {\left| #1 \right|}
\newcommand \floor[1] {\left\lfloor #1 \right\rfloor}
\newcommand \ceil[1] {\left\lceil #1 \right\rceil}

% Complexity
\newcommand{\npcomplete}{\textsc{Np}-\textsc{complete}}
\newcommand{\nphard}{\textsc{Np}-\textsc{Hard}}
\newcommand{\nlogspace}{\textsc{NLogspace}\xspace}
\newcommand{\pspace}{\textsc{PSpace}\xspace}
\newcommand{\expspace}{\textsc{ExpSpace}\xspace}

\newcommand{\mcl}[1]{\multicolumn{1}{l}{#1}}
\newcommand{\mcc}[1]{\multicolumn{1}{c}{#1}}

\newcommand{\mypara}[1]{\vspace{0.6em} \noindent{\bf #1}}
\newcommand{\myipara}[1]{\vspace{0.6em} {\em #1}\xspace}
%\newcommand{\myipara}[1]{\vspace{0.6em} \noindent{\em #1}\xspace}
\newcommand{\myiparaCompact}[1]{ {\em #1}\xspace}

% Sets
\newcommand{\Reals}{\ensuremath{\mathbb{R}}}
\newcommand{\reals}{\Reals}
\newcommand{\real}{\Reals}
\newcommand{\Nats}{\ensuremath{\mathbb{N}}}
\newcommand{\Integers}{\ensuremath{\mathbb{Z}}}
\newcommand{\Rationals}{\ensuremath{\mathbb{Q}}}

%%\newcommand{\mathsf}[1]{\mbox{\textsc{#1}}}
\newcommand{\mathsc}[1]{\mbox{\sc #1}}
\newcommand\scr[1]{\ensuremath\mathcal{#1}}
\newcommand\Gradient{\ensuremath\nabla}
\newcommand\pdiff[2]{\partial_{#1}{#2}}
\newcommand\jth[2]{ #1^{(#2)} }

% Logic
% \newcommand{\land}{\ensuremath\wedge}
% \newcommand{\lor}{\ensuremath\vee}

% Software
\newcommand\flowstar{FLOW*}
\newcommand\MATLAB{Matlab\textsuperscript{\textregistered}}
\newcommand\SIMULINK{Simulink\textsuperscript{\textregistered}}
\newcommand\STATEFLOW{Stateflow\textsuperscript{\textregistered}}
\newcommand\EMCODER{Embedded Coder\textsuperscript{\textregistered}}

%\newcommand\vs{\mathbf{s}}
%\newcommand\vw{\mathbf{w}}
%\newcommand\vx{\mathbf{x}}
%\newcommand\vy{\mathbf{y}}
%\newcommand\vz{\mathbf{z}}
%\newcommand\vu{\mathbf{u}}
%\newcommand\vj{\mathbf{j}}

\newcommand \vx {\vec{x}}
\newcommand \vy {\vec{y}}
\newcommand \vz {\vec{z}}
\newcommand \vu {\vec{u}}
\newcommand \vb{\vec{b}}

% long squiggily arow
\newcounter{sarrow}
\newcommand\xrsquigarrow[1]{%
\stepcounter{sarrow}%
\begin{tikzpicture}[decoration=snake]
\node (\thesarrow) {\strut#1};
\draw[->,decorate] (\thesarrow.south west) -- (\thesarrow.south east);
\end{tikzpicture}
}

% misc
\newcommand\ii{i+1}
\newcommand\denotation[1]{ \left\llbracket #1 \right\rrbracket}



% The \munepsfig command is used to insert a new EPS figure
% into our document.  Usage is:
%
%       \munepsfig[args]{filename}{caption}
%
% where:
%       - the optional 'args' argument is passed to the
%         embedded \includegraphics command, this can be used
%         to scale the figure or rotate it.
%       - 'filename' is the name of the EPS file in the 'figures'
%         directory that is to be inserted (note that 'filename'
%         should not include the '.eps' extension).
%       - 'filename' also serves as the label for the figure.
%         with the text 'fig:' prepended.
%
% Sample Usage:
%       \munepsfig[scale=0.5,angle=90]{barchart}{Population over time}

% inserts the EPS file 'figures/barchart.eps' reduced in size by 50%
% rotated 90 degrees and with the caption "Popuation over Time."
% We can refer to that figure as Figure~\ref{fig:barchart} in the text.
%
\newcommand{\inclfig}[4][scale=1.0]{%
        \begin{figure}[t]
                \centering
                \vspace{2mm}
%               \includegraphics[#1]{figures/#2.eps}
                \includegraphics[#1]{figs/#2}
                \caption{#3}
                \label{fig:#4}
        \end{figure}
}

% Units
%\newcommand{\degree}{^{\circ}}
\newcommand{\degreeC}{^{\circ}{\rm C}}
%\newcommand{\degreeF}{^{\circ}{\rm F}}

%\newcommand{\CC}{C\nolinebreak\hspace{-.05em}\raisebox{.4ex}{\tiny\bf +}\nolinebreak\hspace{-.10em}\raisebox{.4ex}{\tiny\bf +}}
\def\Cpp{{C\nolinebreak[4]\hspace{-.05em}\raisebox{.4ex}{\tiny\bf ++}}}
\def\CC{{C\nolinebreak[4]\hspace{-.05em}\raisebox{.4ex}{}}\;}



% RRT
\newcommand{\RRT}{\mathcal{RRT}}
\newcommand{\RRTgoal}{\x_{goal}}
\newcommand{\RRTinit}{\x_{init}}
\newcommand{\RRTv}[1]{v_{#1}}
\newcommand{\RRTe}{e}
\newcommand{\RRTV}{V}
\newcommand{\RRTE}{E}
\newcommand{\RRTsample}{x_{sample}}
\newcommand{\RRTnear}{x_{near}}
\newcommand{\RRTtoEx}{x_e}
\newcommand{\RRTxNew}{x_{new}}
\newcommand{\RRTuNew}{u_{new}}
\newcommand{\RRTdedge}[3]{(#1 \xrightarrow{#3} #2)}

% Graph
\newcommand{\Graph}{\mathbf{G}}
\newcommand{\graph}{\mathbf{G}}
\newcommand{\vertex}[1]{v_{#1}}
%\newcommand{\vertex}[1]{#1}
\newcommand{\dedge}[2]{e_{(\vertex{#1}\rightarrow\vertex{#2})}}
\newcommand{\edge}{e}
\newcommand{\vertexSet}{V}
\newcommand{\edgeSet}{E}
%\newcommand{\vertexSetD}{V^\delta}
%\newcommand{\edgeSetD}{E^\delta}
%\newcommand{\graphVE}[1]{\mathbf{G_{#1}(\vertexSet,\edgeSet)}}
%\newcommand{\graphD}[1]{\mathbf{G^\delta_{#1}(\vertexSet^\delta,\edgeSet^\delta)}}
\newcommand{\kPaths}{k\_paths}
\newcommand{\trajToNode}{\Gamma}
\newcommand{\Path}{\mathbf{P}}
\newcommand{\weight}{\mathbf{W}}

% Hybrid Automata
\newcommand{\HA}{\ensuremath{\mathcal{A}}}

\newcommand{\System}{\ensuremath{S}}
\newcommand{\Inputs}{\ensuremath{\mathcal{U}}}
\newcommand{\inputsignal}{\mathbf{u}}
\newcommand{\Flow}{\ensuremath{\mathcal{F}}}
\newcommand{\Modes}{\ensuremath{\mathcal{M}}}
\newcommand{\ContStates}{\ensuremath{X}}
\newcommand{\UnsafeStates}{\ensuremath{U}}
\newcommand{\InitStates}{\ensuremath{X}_\init}
\newcommand{\InitModes}{\ensuremath{M}_\init}
\newcommand{\Inv}{\ensuremath{\mathcal{I}}}
%\newcommand{\Transitions}{\ensuremath{\Delta}}
\newcommand{\Transitions}{\scr{T}}
\newcommand{\HybridStates}{\ensuremath{\mathcal{X}}}
\newcommand{\HybridStateSet}{\ensuremath{X}}
\newcommand{\discreteMode}{q}
\newcommand{\contState}{x}
\newcommand{\ResetMap}{\mathcal{R}}
\newcommand{\Guards}{\mathcal{G}}
\newcommand{\Init}{\mathcal{X}_{0}}
\newcommand{\Err}{\ContStates_{f}}
\newcommand{\initmode}{q_\init}
\newcommand{\cInit}{\ensuremath{X}_0}
\newcommand{\Unsafe}{\mathcal{X}_f}
%\newcommand{\reachSet}[2]{R^{#1}_{\HA}({#2})}
%\newcommand{\Hflow}{\mathcal{H}_{\HA}}
\newcommand{\Hflow}{\mathcal{H}}
\newcommand{\reachSet}{R}

% Trajectories
\newcommand{\discTraj}{q_h}
\newcommand{\hybridTraj}{\tau_h}
\newcommand{\traj}{\pi}
\newcommand{\trajSeg}[2]{\pi^{\mode_{#1}}_{\tau_{#2}}}
\newcommand{\SegTraj}{\mathbf{S}_\pi}
\newcommand{\dwell}{\tau}
\newcommand{\tran}{\delta}
\newcommand{\tbegin}{b}
\newcommand{\tend}{e}
%\newcommand{\graph}{G}
\newcommand{\trajStore}{TS}
\newcommand{\trajSet}{TS}
\newcommand{\candTraj}{CT}
\newcommand{\candTrajSet}{CTS}
\newcommand\Cost{\mathsc{Cost}}


% State variables
\newcommand{\x}{\mathbf{x}}
\newcommand{\y}{\mathbf{y}}
\newcommand{\z}{\mathbf{z}}
%\newcommand{\u}{\mathbf{x}}
\newcommand{\w}{\mathbf{w}}
%\newcommand{\vec}[1]{\mathbf{#1}}
\newcommand{\dvx}{\mathbf{\dot{x}}}
\newcommand{\inp}{\mathbf{u}}
\newcommand{\mode}{\ensuremath{q}}
\newcommand{\dx}{\ensuremath{\dot{x}}}

% Metrics
\newcommand{\lmetric}[2]{{\Vert #2 \Vert}_#1}

% ODE
\newcommand{\odesolution}{\Phi}
\newcommand{\matA}{\mathbf{A}}
\newcommand{\matB}{\mathbf{B}}
\newcommand{\mata}{\mathbf{a}}


\newcommand \maximize{\mathbf{max.}\ }
\newcommand \minimize{\mathbf{min.}\ }

\newcommand \numsolve{\mathtt{NumericSolver}}
\newcommand\Flowmap{\mathsc{Flow}}
%%\newcommand\myipara[1]{\par\noindent\textit{#1:}}

% Arrows
%\newcommand{\contArrow}[2]{\underset{#2}{\overset{#1}{\leadsto}}}
\newcommand{\contArrow}[1]{\leadsto_{#1}}
\newcommand{\jumpArrow}[1]{\xrightarrow{#1}}


\newcommand\dist{\mathsf{d}}
\newcommand\src{\mathsf{src}}
\newcommand\dest{\mathsf{dest}}
\newcommand\timeElapse{\mathcal{T}}
\newcommand\simulate{\mathsc{sim}_\Delta}
%%\newcommand\sim{\mathsc{sim}}
\newcommand\cost{\mathsc{cost}}
\newcommand\reach[1]{\xrightarrow{#1}}
\newcommand\areach[1]{\overset{#1}{\rightsquigarrow}}
%\newcommand\areach[1]{\xrsquigarrow{#1}}

%\newcommand\areach[2]{\overset{#1}{\underset{#2}\rightsquigarrow}}

\newcommand\crel[1]{\xrightarrow{#1}}
\newcommand\drel[1]{\overset{#1}{\rightsquigarrow}}
\newcommand\rel[1]{\overset{#1}{\rightsquigarrow}}


\newcommand\intr{\mathsf{interior}}
\newcommand\assign{:= }
\newcommand\worklist{\mathsf{workList}}
\newcommand\exploredCells{\mathsf{V}}
\newcommand\exploredEdges{\mathsf{E}}
\newcommand\unsafeCells{\mathsf{V}_u}
\newcommand\initialCells{\mathsf{V}_0}
\newcommand\prb{\mathbb{P}}
\newcommand{\abstracteps}{\epsilon}
\newcommand{\refinedeps}{\delta}
\newcommand{\absgraph}{\scr{H}_{\epsilon}(\Delta)}
\newcommand{\refgraph}{\scr{H}_{\delta}(\Delta)}
\newcommand{\abscells}{{\scr{C}}}
\newcommand{\refcells}{{\scr{D}}}






%%%%%%%% from Sriram sty, rearrange!

\def\mathsc#1{\mbox{\sc #1}}

\newcommand\trp[1] {#1^{\scriptscriptstyle T}}
\newcommand\scrS{\mathcal{S}}
\newcommand\scrT{\mathcal{T}}
\newcommand\scrF{\mathcal{F}}
\newcommand\scrG{\mathcal{G}}
\newcommand\scrH{\mathcal{H}}
\newcommand \ints {\ensuremath \mathbb{Z}}
%%\newcommand \extreals {\ensuremath \mathcal{R}^{+}}
\newcommand \lin[1]{\mathit{Lin}(#1)}
\newcommand \conic[1]{\mathit{Cone}(#1)}
\newcommand \conv[1]{\mathit{Convex}(#1)}
\newcommand \false {\mathit{false}}
\newcommand \true  {\mathit{true}}
%\newcommand\pre{\preceq}
\newcommand \pres{\mathbf{pres}}
\newcommand \F {\mathfrak{F}}

\newcommand \abstractF {\F_A}
\newcommand \abstractDomain {\Sigma_A}
\newcommand \templateDomain {\Sigma_T}
\newcommand \dropped {\mathsc{x}}
\newcommand \ctop {\vec{c}_\top}
\newcommand \cbot {\vec{c}_\bot}
\newcommand \concT {\gamma_T}
\newcommand \abstractorder {\leq_A}
\newcommand \abstractLeq{\sqsubseteq}
\newcommand \abstractor {\sqcup}
\newcommand \abstractand {\sqcap}
\newcommand \bigabstractor {\bigsqcup}
\newcommand \setdef[2] { \left\{ #1 \mid #2 \right\}}

\newcommand \CH {\mathcal{CH}}


%%\newcommand \mathsc[1]{\mbox{\textsc{#1}}}
%%transpose



\newcommand \T {\mathcal{T}}
\newcommand \post {\ensuremath\mathit{post}}
\newcommand \dpost {\widehat{\mathit{post}}}
\newcommand \homg[1] {\mathsc{hom}(#1)}
\newcommand \cons{\mathsc{cons}}
\newcommand \widen {\nabla}
\newcommand \dual[1]{\widehat{#1}}
\newcommand \narrow {\/ \bigtriangleup \/ }
\newcommand \refine {\partial}
\newcommand \deta{\pi}

\newcommand \latcap {\sqcap}
\newcommand \latcup {\sqcup}
%\newcommand \guard {\xi}
\newcommand \templ{\gamma}
\newcommand \sizeof[1] { |#1| }
\newcommand \uset[2]{\underset{#1}{\underbrace{#2}}}
\newcommand \ith[2]{ {#1}^{(#2)}}
\newcommand \Cells {\ensuremath \mathcal{C}}
\newcommand \InitCells {\ensuremath \mathcal{C}_\init}
\newcommand \CellEdges {\ensuremath \mathcal{E}}
\newcommand \CellLabelingFunction{\ensuremath \mathcal{L}_\Cells}
\newcommand \EdgeLabelingFunction{\ensuremath \mathcal{L}_\CellEdges}
\newcommand \exprs[1] {\mathsf{EXP}(#1)}
\newcommand \complex {\ensuremath \mathcal{C}}
\newcommand \grb {Gr\"obner\xspace}
\newcommand \hi[1]{}
\newcommand \hihenny[1]{}
\newcommand \newhenny[1]{#1}
\newcommand\markmargin[2]
{[\marginpar[\hfill \mbox{#1}$\rightarrow$]{$\leftarrow$\mbox{#1}} 
{\sf #2]}}

\newcommand\highl[1]{\psframebox[fillstyle=solid,fillcolor=lightgray,linewidth=0pt]{\textbf{#1}}}


\newcommand \locs{\mathbf{L}}
\newcommand\en{\mathbf{en}}




\newcommand{\chapterheading}[1]
{\vfill  
\hfill \fbox{\begin{minipage}{5.5in}
\textsl{#1} \end{minipage} } \newpage}

\newcommand \Z{ \mathbf{Z}}

\newcommand \mand {\ \mathit{and}\ }
\newcommand \matb{\vec{b}}
\newcommand \matl{\vec{\lambda}}
\newcommand \vg {\vec{g}}
\newcommand \va{\vec{a}}
\newcommand\ve{\vec{e}}
\newcommand \vc {\vec{c}}
\newcommand \vf {\vec{f}}
\newcommand \vh {\vec{h}}
\newcommand \vv{\vec{v}}
\newcommand \vzero {\vec{0}}
\newcommand \vq {\vec{q}}
\newcommand \rank {\mathit{rn}}
\newcommand\vbeta{\vec{\rho}}
\newcommand \Petri {\ensuremath \mathcal{P}}
\newcommand \maxrank {\mathit{maxrank}}
\newcommand \lists{\mathit{list}}
\newcommand \nullsp{\mathit{null}} 
\newcommand \vl[1] {\vec{\lambda_{#1}}}
\newcommand\vlam{\vec{\lambda}}
\newcommand \st{\mathbf{s.t.}\ }


\newcommand \D {\mathit{D}}
\newcommand \I {\mathit{I}}
\newcommand \Loc{\mathit{Loc}}
\newcommand \lie{\mathcal{L}}
\newcommand \grad{\nabla}
\newcommand \cn {\mathsf{Cn}}

%\newcommand \diff[2] {\ensuremath \frac{d #1}{d #2}}

\renewcommand\paragraph[1] {\smallskip\par\noindent\textbf{#1}\ \ }
\newcommand \cplus {\uplus}

\newcommand\relop{\mathop{\bowtie}}
\newcommand \vt{\vec{t}}
\newcommand \vd{\vec{d}}
\newcommand\e{\mathsf{e}}
\newcommand\yl{\mathsf{l}}
\newcommand\yu{\mathsf{u}}
\newcommand\yb{\mathsf{b}}


\newcommand\init{\mathsf{init}}
\newcommand\unsafe{\mathsf{unsafe}}

\newcommand\ol[1]{\overline{#1}}
% \newcommand\state[1]{\vec{{\texttt{#1}}}}

\newcommand\scrP{\mathcal{P}}
\newcommand\scrX{\mathcal{X}}
\newcommand\scrC{\mathcal{C}}
\newcommand\interVal[1]{[\underline{#1}, \overline{#1}]}
\newcommand\HIDE[1]{}


\newcommand{\HErr}{\mathcal{X}_{f}}
\newcommand\Eq{\mathsf{Eq}}
\newcommand\Ineq{\mathsf{Ineq}}
%\newcommand \setof[1] { \left\{ #1 \right \}}
%\newcommand \pdiff[2] { \ensuremath \frac{\partial #1}{\partial #2}}
%\newcommand \reach[1]{\mathsf{Reach}(#1)}


\newcommand{\vertexSetD}{V^\delta}
\newcommand{\edgeSetD}{E^\delta}
\newcommand{\graphVE}[1]{\mathbf{G_{#1}(\vertexSet,\edgeSet)}}
\newcommand{\graphD}[1]{\mathbf{G^\delta_{#1}(\vertexSet^\delta,\edgeSet^\delta)}}



% plant and controller

\newcommand\pgm{\rho}
\newcommand\Outputs{Y}
%\newcommand\locs{L}
%\newcommand\sloc{l_i}
%\newcommand\eloc{l_o}
%\newcommand\op{op}
%\newcommand\expr{E}


\newcommand\symMem{\mu}
\newcommand\CFG{\Pi}
\newcommand\pLocs{L}
\newcommand\pEdges{E}
\newcommand\pLabels{\Phi}
\newcommand\pEdge[1]{edge(#1)}
\newcommand\pVars{\mathcal{V}}
\newcommand\pVar{v}
\newcommand\pTransRel{\rho}
\newcommand\pLoc{l}
\newcommand\pLocI{l_0}
\newcommand\pLocF{l_f}
\newcommand\pVarI{V_0}
\newcommand\pPaths{\mathcal{P}}
\newcommand\pPath{p}
\newcommand\pPathCons{\kappa}
\newcommand\pCons{\xi}
\newcommand\pSymMem{\sigma}
\newcommand{\domain}{\mathcal{D}}
\newcommand{\pre}{\mathit{pre}}
\newcommand{\update}{\mathit{update}}
%TODO: redef?? fixit!
%\newcommand{\C}{\mathtt{C}}

\newcommand{\sampletime}{{\tau_{s}}}

\newcommand\pgmF{\rho}
\newcommand\pgmA{\hat{\rho}}
\newcommand\plantStates{x}
\newcommand\abstractPlantStates{X}
\newcommand\controllerOutputs{u}
\newcommand\controllerStates{s}
\newcommand\PathCons{\kappa}
\newcommand\constraints{\psi}
%TODO: redinfe path...conflicting with prev def
%\newcommand\Path{\mathcal{P}}
\newcommand\absState{\mathcal{A}}

\newcommand\ts{\tau_s}

\newcommand\Cix{C_i^x}
\newcommand\Ciy{C_i^y}
\newcommand\Ciix{C_{i+1}^x}
\newcommand\Ciiy{C_{i+1}^y}
% \newcommand\si{s_i}    %conflicts with package siuntix
\newcommand\sii{s_{i+1}}
\newcommand\ui{u_i}
\newcommand\uii{u_{i+1}}
\newcommand\sip[1]{s_{i}^{p_{#1}}}
\newcommand\uip[1]{u_{i}^{p_{#1}}}
\newcommand\siip[1]{s_{i+1}^{p_{#1}}}
\newcommand\uiip[1]{u_{i+1}^{p_{#1}}}
\newcommand\kpi{\kappa^{p_i}}


% \renewcommand{\Si}{S_i}
\newcommand\Si{S_i}
\newcommand\Sii{S_{i+1}}
\newcommand\Ui{U_i}
\newcommand\Uii{U_{i+1}}
\newcommand\Sip[1]{S_{i}^{p_{#1}}}
\newcommand\Uip[1]{U_{i}^{p_{#1}}}
\newcommand\Siip[1]{S_{i+1}^{p_{#1}}}
\newcommand\Uiip[1]{U_{i+1}^{p_{#1}}}

\newcommand \ctrl{\mathsc{ctrl}}
\newcommand\sample{\mathsc{Sample}}

\newcommand\TrajOpt{TrajOpt\;}

\newcommand{\cellequivalence}{\equiv_{\scr{C}}}
\newcommand{\vk}{\mathbf{k}}



\newcommand{\Traj}{\tau}
\newcommand{\cstates}{\scr{X}}
\newcommand{\csys}{S}
\newcommand{\cell}{C}
% \newcommand{\quant}{Q}
%\newcommand\quant{\mathsc{Quant}}

\newcommand\fmincon{\texttt{fmincon}\xspace}


\newcommand{\pwa}{\rho}
\newcommand{\map}{f}
\newcommand{\amap}{f}
\newcommand{\gm}{\scr{T}}
\newcommand{\guard}{g}
\newcommand{\ds}{\scr{D}}
\newcommand{\prop}{\scr{P}}

\newcommand{\simmap}{\mathsc{sim}}
% \newcommand{\initmode}{q_0}
\newcommand{\initstate}{x_0}

\newcommand{\quant}{h}

\newcommand{\TSAbstraction}[2]{\mathcal{T}^{#1}(#2)}
\newcommand{\RTSAbstraction}[2]{\mathcal{R}^{#1}(#2)}

\newcommand{\epsilonmin}{\epsilon_\mathrm{min}}

\newcommand \vzr {\mathbf{0}}
\newcommand \vxp {\vx^{+}}



\begin{abstract}
Models of embedded control systems are often so complex that
falsification and testing approaches can scale only by numerically
simulating them, and hence treating the systems as black-boxes. Common
approaches like sampling-based falsification, use numerical
simulations to search for violations of safety properties. However,
this relegates them to the `best effort' category. On the other hand,
\emph{rigorous} verification (and falsification) techniques require
symbolic models and are comparatively less scalable. We try to bridge
this gap by incorporating data-driven models in a sampling based
search. Using an efficient, on-the-fly discrete abstraction
search procedure, we compute models only for selective regions of the state
space, thus `enriching' the abstraction. We then show how the
`enriched' abstraction can be used to guide the search for violations
without refinement.
%We use grid-based discrete abstractions to partition
%the state-space of the black-box hybrid dynamical system. We then use
%an efficient search procedure to find abstract violations of safety
%properties
%Assuming a grid-based implicit abstraction over the
%state-space of the black-box hybrid dynamical system, we enumerate
%abstract violations of the safety property. We then model the system
%behavior along these violations and search for concrete violations.
The automatic falsification proceeds by (a) randomly exploring the
state-space abstraction using numerically computed trajectories and
enumerating abstract violations, (b) modeling the behavior of the
system along these abstract violations by using discrete-time
piecewise affine (PWA) \emph{relational} models, and finally, (c)
encoding the search for a concrete violation as a bounded model
checking (BMC) problem.  The BMC query is then a disjunction of
conjunctions of linear constraints (i.e. combinations of linear
programs). It can be solved using off-the-shelf SMT solvers or by
multiple runs of a linear programming solver. For any counter-example
found in this `enriched' abstraction, we check the existence of a
corresponding violation in the original system. We demonstrate the
efficacy of our technique on a few benchmarks.

% Models of embedded control systems are often so complex that they can
% be treated as effectively black-box systems. Sampling-based methods
% for testing such systems use numerical simulations to search for
% violations of properties (specified in a suitable requirement
% formalism, such as temporal logic). This constrains them to a `best
% effort' category unlike rigorous verification techniques, which
% however require a model amenable to symbolic analysis. This work tries
% to bridge the gap by presenting an automatic falsification technique
% which learns a data-driven abstraction of the black box system before
% searching for a violation. It proceeds by (a) modeling the behavior of
% the system using a piecewise affine (PWA) discrete-time {\em
% relational} model and, (b) encoding the search for a violation in the
% abstract model as a bounded model checking problem. The bounded model
% checking query in this instance can be viewed as a disjunction of
% conjunctions of linear constraints (i.e. linear programs) and can be
% solved directly using off-the-shelf SMT solvers or by multiple runs of
% a linear programming solver. For any counter-example found in the
% abstraction, we check the existence of a corresponding violation in
% the original system. We demonstrate the efficacy of our technique on a
% few dynamical systems examples.








% =========================================================================
% Sampling based falsification techniques for black box hybrid dynamical
% systems use numerical simulations to search for violations of
% properties (specified in temporal logic). This constraints them to a
% `best effort' category unlike rigorous verification techniques, which
% however require a white box model. This work tries to bridge the gap
% by presenting an automatic falsification technique which models the
% black box system before searching for a violation. It proceeds by (a)
% modeling the behavior of the system using a piecewise affine (PWA)
% discrete time model and, (b) encoding the search for a violation as a
% standard bounded model checking problem. The resulting problem is a
% disjunction of linear programs and can be solved directly using
% off-the-shelf SMT solvers or by multiple runs of a linear programming
% solver. If a counter-example is found, the existence of a
% corresponding violation in the original system is checked for
% reproducibility.



% In this work we address the problem of finding unsafe behaviors with
% given a safety property for a hybrid dynamical system. We treat the
% system as a black box and approach the problem in two distinct steps.
% First, we model the behavior of the system using a piecewise affine
% discrete time model. Such a model is incomplete in nature and is
% computed with respect to the given property. Next, we encode the
% search for falsification as a bounded model checking query and use an
% SMT solver to find a counter example. If found, we check the existence
% of the violation in the original system to ensure reproducibility.

%Time bounded LTL property in
%Models from verification
%model without data
%black box model/data -> structured model

\end{abstract}

\maketitle

%%%%%%%%%%%%%%%%%%%%%%%%%%%%%%
\section{Introduction}
\label{sec:intro}

% Testing of hybrid systems is a complex task, due to the interaction
% between discrete and continuous dynamics. Industrially developed
% hybrid systems, with embedded software interacting with physical
% components, are typically safety-critical. Thus, it is important to
% guarantee that the systems adhere to functional and safety
% specifications. Since the reachability problem for hybrid systems is
% known to in general be undecidable, exhaustive exploration of the
% state-space is intractable, and thus testing is an approach to attempt
% to achieve system correctness. However, to guarantee safety and other
% requirements, the testing procedures need to rely on a rigorous
% theoretical foundation. In this special session aims to collect and
% connect researchers, within academia as well as industry, working on
% model-based testing of hybrid systems. The session calls for papers
% proposing new theoretical research directions in the field, as well as
% papers describing practical applications of model-based testing of
% hybrid systems.


Model-based development is rapidly becoming the preferred paradigm for
developing embedded control software in an industrial context.  Tools
like \SIMULINK are often used to create models of closed-loop
dynamical systems, \ie, a plant model of the physical systems that we
wish to control, and a model for the embedded controllers.  Automatic
approaches for testing the safety and performance of such closed-loop
systems is challenging, and current approaches focus mostly on guided
testing using numerical simulations
\cite{annpureddy2011s,donze2010breach,deshmukh2015stochastic,dreossi2015efficient,akazaki}.
On the other hand, theoretical research on hybrid and timed systems
focusses on formal guarantees for restricted classes of dynamical
systems such as hybrid automata where the system state evolves
according to a vector field that is either constant, affine
\cite{frehse2011spaceex}, or polynomial \cite{chen2015reachability}.
In this paper, we propose an approach that leverages useful features
from both kinds of techniques.

We propose a two step process for finding violations of safety
properties. In the first step, we learn a piece-wise affine (PWA) {\em
relational} model from simulations of the dynamical system. In the
second step, we leverage symbolic techniques to exhaustively analyze
the model learned in the first phase. We remark that the approach we
propose is neither sound nor complete: a violation found by our
technique may not exist in the original system, and our technique may
miss violations that exist in the original system. A fair question is:
why is such a technique useful? There are two perspectives from which
we can answer this question.

In many settings, we may not have a physics-based model derived from
first principles, but may just have time-series data of the system
behavior. In such a setting, a data-driven model allows us to
generalize individual time-series behaviors. But, there are several
approaches in the literature to perform data-driven modeling; e.g.,
system identification techniques that learn dynamic models from data
\cite{ljung1999system}, auto-regressive models for forecasting
\cite{wei1994time}, machine learning techniques that learn static or
dynamic models from data
\cite{narendra1990identification,lu2009linear}.  Why then do we need a
new modeling technique? The answer to this question lies in our goal
for the second step, \ie, symbolic analysis of the learned model. For
this step, we need to have models that can be digested by existing
symbolic tools, which is harder to do with some of the aforementioned
heavier data-driven modeling methods. Hence, we use a lighter flavor
of data-driven modeling such as the one proposed for learning
relational models from data
\cite{zutshi2012timed,sankaranarayanan2011relational}. By learning a
{\em PWA relational model}, \ie, essentially a PWA discrete-time
dynamical system, we can use software analysis approaches such as
bounded model checking assisted by SMT solvers and linear programming
solvers.

The other perspective for our technique is the context when we have an
effectively black-box model, \ie, a model that has features such as
nonlinearities, delays or look-up tables that make symbolic analysis
rather challenging.  We note that the ultimate goal for our technique
is to find violations of safety properties in the {\em actual embedded
control system}, and not, per se, in the model. The aphorism
attributed to mathematician George E.P. Box is quite apt for our
context, and bears worth repeating,  ``all models are wrong, some
models are useful.'' We can sidestep analyzing a highly complex model
by approximating its behavior with a simpler relational model, which
in turn can give us promising tests to run on the actual system. This
gives our technique intrinsic value as a debugging tool.

% In our previous work~\cite{zutshi2014multiple}, we analyzed systems by
% restricting ourselves to their black box semantics. This enabled us to
% reason about the state-space reachability of the system without a
% direct analysis of its structure. Using a coarse abstraction which we
% searched on-the-fly, we could observe the local dynamics as required.
% This gave us an efficient procedure to find abstract counter-examples.
% However, to concretize the counter-examples, a `closer look' or
% refinement of the abstraction is required. The grid based state-space
% abstraction was refined by splitting the relevant cells (abstract
% states) into smaller ones. As noted previously, due to the curse of
% dimensionality, such a uniform splitting is an expensive operation,
% and can not scale to higher dimensions.

In \cite{zutshi2014multiple}, the authors investigate an on-the-fly
abstraction technique that is close to the approach we propose. A key
difference is that the abstraction explored in this work is an
existential {\em relational} abstraction. The approximate model
learned is a directed acyclic graph, where nodes represent regions of
the state-space, and an edge between two regions indicates a
transition between states in the regions. Our approach further
improves the existential abstraction by annotating the edges in the
graph-based abstraction with affine relations between the source and
destination states.

% We further our exploration of trajectory segment based methods; and
% explore an alternative approach to overcome the explosion in abstract
% states. Instead of selectively refining the abstraction, we compute a
% model of the black box system and use bounded model checking to find a
% concrete counter-example in the model. Due to modeling errors, this
% might not be reproducible in the original black-box system. 


More specifically, we use linear regression to quantitatively estimate
the discovered edges in the graph-based abstraction of the
state-space. These linear maps approximate locally observed behaviors,
and the resulting graph or the Piece-Wise Affine (PWA) relational
model can be interpreted as an infinite state discrete transition
system and model checked for time bounded safety properties. A
counter-example if found, can indicate the presence of a violation in
the system.  We then use this counter-example to guide the search
towards a counter-example in the original system.


%Furthermore, using linear programming, we can over-approximate the
%neighbhourhood of the counter-example in the model.

% In the conclusion, we discuss extensions to data driven
% approaches (instead of simulator driven), where instead of a
% simulator, a fixed set of data is provided as the behavioral
% description of the system. Using a combination of relational modeling,
% and program analysis we outline the future work for falsifying
% properties of SDCS.
% Bounded time behvaiors are modeled
% only one violation is being searched for
% mention explicitly the relation between older cegar work

The layout of the paper is as follows: In Section~\ref{sec:prelims},
we introduce the basic steps to obtain a PWA relational model. In
Section~\ref{sec:pwa-rel}, we explore the connection between the PWA
relational model and previous graph-based abstraction techniques such
as those in \cite{zutshi2014multiple}. We present extensions of
relational modeling in Section~\ref{sec:rel-mod}, and present
experimental validation of our technique on a few textbook examples of
dynamical systems with hybrid and polynomial dynamics in
Section~\ref{sec:res}.

%%%%%%%%%%%%%%%%%%%%%%%%%%%%%%

% %%%%%%%%%%%%%%%%%%%%%%%%%%%%%%
% %\subsection{Motivation}
% %\label{sec:mot}
% %\input{motivation.tex}
% %%%%%%%%%%%%%%%%%%%%%%%%%%%%%%

%%%%%%%%%%%%%%%%%%%%%%%%%%%%%%
\subsection{Related Work}
\label{sec:rel}
Piece-Wise Affine (PWA) models are quite popular in the modeling of
both continuous and hybrid dynamical systems. They are very
expressive, and can model non-linear continuous dynamics with
arbitrary accuracy~\cite{wen2008basis} and a large family of hybrid
systems~\cite{heemels2001equivalence}.

\mypara{Falsification of Properties of Hybrid Systems.} The current
state-of-the-art falsification approaches are based on numerical
simulations and can be grouped into robustness guided
S-Taliro~\cite{annpureddy2011s}, Breach~\cite{donze2010breach} and RRT
(Rapidly-exploring Random Tree)
based~\cite{nahhal_test_2007,Dang09,dreossi2015efficient}.  An
alternative approach, based on multiple shooting and CEGAR
(Counterexample Guided Abstraction Refinement) was proposed by the
authors in ~\cite{zutshi2014multiple}.  The alternative approach is
the verification approach, where a formal model (usually hybrid
automata) is first constructed (often manually) and then exhaustively
checked for violations.

\mypara{Identification of hybrid systems} using Piece-Wise affine
models is surveyed in~\cite{paoletti2007identification}.
\begin{enumerate}
\item Expressiveness/power of PWA models?
\item Fixed time caveats
\item Effects of refining parameters: time step $\Delta$ and grid size $\epsilon$.
\end{enumerate}

%\mypara{Model Checking PWA Models.} Model checking transition systems

%%%%%%%%%%%%%%%%%%%%%%%%%%%%%%

%%%%%%%%%%%%%%%%%%%%%%%%%%%%%%
%\section{Graph-Based Abstractions for Black-Box Dynamical Systems}
\section{Transition System Abstraction for Dynamical Systems}
\label{sec:scam}
In this section, we present the model of the hybrid dynamical system
under test, and the background for PWA relational abstractions.


\subsection{System-under-test ($\System$)}

We assume that the system-under-test is a hybrid dynamical system, \ie, a
system which has discrete modes and a continuous state-space, and which
evolvution is either (1) continuous in time according to the dynamics
associated with its current discrete mode (typically described using
differential equations over its continuous state), or (2) through
discrete jumps that possibly change the discrete mode of the system and
possibly reset its continuous state to some value.  However, we assume
that the system is {\em effectively black-box}. This means that we do not
assume any knowledge of the symbolic dynamical equations describing the
system behavior. We do assume some knowledge of the system: an {\em
interface} view, in which we are allowed to stimulate the system with
inputs and observe its outputs.

We formally define a system-under-test $\System$ as a tuple $(\Modes,
\ContStates, \InitModes, \InitStates, \Delta, \Inputs, \Outputs)$.
Here, system state-space (denoted $\HybridStates$) is a subset of
$\Modes \times \ContStates$, where $\Modes$ is a finite set of
discrete modes, and $\ContStates$ is a compact subset of $\Reals^n$
representing the set of continuous states.  The sets $\InitModes
\subseteq \Modes$ and $\InitStates \subseteq \ContStates$ represent
initial modes and states respectively.  We assume that the system is
{\em initialized} to some initial mode $\initmode \in \InitModes$ and
an initial state $\initstate \in \InitStates$ at time $t=0$.  We
assume that the system has $m$ exogenuous inputs which take values
from the set $\Inputs$ which is a compact subset of $\Reals^m$. We
assume that the system has $\ell \le n$ outputs, where $\Outputs
\subseteq \HybridStates$. In this paper, we assume that the entire
system state is observable.  In other words, $\Outputs =
\HybridStates$.

Finally, we restrict our attention to switched-mode systems, \ie, those in
which there is a partition imposed on $\ContStates$, and there is a
bijective map between each element $\ContStates_i$ in the partition and
each mode in $\Modes$. Thus, we can omit the discrete modes from the
system description, and set $\HybridStates$ = $\ContStates$.

The operational semantics of $\System$ is defined in terms of a {\em
forward simulation map}\footnote{We require that $\System$ can always
be simulated forward in time with deterministic results. A few
technical conditions are required on the system dynamics are required
to ensure that this assumption is valid, see for
example~\cite{Meiss/2007/Differential} and \cite{goebel2012hybrid} for
further details. In practice, such conditions are usually impossible
to check for a black-box model. We then assume that the black-box
model is described as a model in a deterministic simulation framework
such as Simulink\textregistered.} $\simmap_\Delta$, which is a
function that maps a time $t$, the current state of the system, and an
input value in $\Inputs$ to the state of the system at time
$t+\Delta$, \ie, $\simmap_\Delta$ is a function from $\ContStates
\times \Inputs \times \Reals^{\ge 0}$ to $\ContStates$.

\subsection{$\epsilon$-precision Transition System Abstraction}

We now show how we can abstract a hybrid dynamical system $\System$
which has trajectories in dense-time by a discrete transition system.
A discrete transition system $\mathcal{T}$ can be defined as the tuple
$(\Cells, \CellEdges, \InitCells, \CellLabelingFunction)$, where
$\Cells$ is a finite set of cells, $\CellEdges \subseteq \Cells \times
\Cells$ is an unlabeled transition relation, and
$\CellLabelingFunction$ is a function that assigns one or more labels
to the cells from a finite set $L$ of labels. The operational
semantics of the transition system are as follows: the system starts
in some cell in $\Cells_\init$. At each step, if the transition system
is in cell $C$, it transitions to cell $C'$ if $(C,C') \in
\CellEdges$. A finite-time trace property $\varphi$ of the system is
some subset of the free monoid $(2^L)^*$. We say that $\mathcal{T}
\not\models \varphi$ if there exists some trace of $\mathcal{T}$ such
that the sequence of labels in that trace does not belong to
$\varphi$.

We now give an example of a specific kind of transition system
abstraction that we can build given $\System$ and the simulation map
$\simmap_\Delta$.  We call the transition system an
$\epsilon$-precision transition system ($\epsilon$-TS for short) and
denote it by $\TSAbstraction{\epsilon}{\System}$.  Let $\epsilon  =
10^{-k}$, for some positive integer $k$.  We define a quantization
function $\quant_\epsilon$ that maps a state $\contState$ in
$\ContStates$ to $10^{-k}*\lfloor \contState*10^k \rfloor$. This
transformation essentially truncates the decimal representation of the
numeric value of the state $\state$ to at most $k$ digits after the
decimal point.  This function induces an equivalence relation
$\contState_i \equiv_\epsilon \contState_j$ iff
$\quant_\epsilon(\contState_i) = \quant_\epsilon(\contState_j)$.

Using the equivalence relation $\equiv_\epsilon$, we can now define
the tuple $(\Cells, \CellEdges, \CellLabelingFunction)$ for
$\TSAbstraction{\epsilon}{\System}$ by considering the quotient
structure $\ContStates/\equiv_\epsilon$.  Essentially, each
equivalence class in the quotient structure is a cell in $\Cells$.
For cells $C$ and $C'$ in $\TSAbstraction{\epsilon}{\System}$, the
edge $(C,C')$ can be an element of $\CellEdges$ iff:
$\simmap_\Delta(\contState, u, t)$ = $\contState'$,
$\quant_\epsilon(\contState)$ = $C$, and,
$\quant_\epsilon(\contState') = C'$.

We assume that there is a specified set of states $\UnsafeStates
\subseteq \ContStates$ that are unsafe\footnote{Our framework is
general enough to reason about richer class of correctness properties
for trajectories such as bounded-time LTL properties. In this case, we
assume the LTL property is defined over Boolean-valued predicates over
$\ContStates$. The labeling function $\CellLabelingFunction$ would
then label cells with the predicates that evaluate to $\mathit{true}$
for some state in the cell.}, and thus restrict our attention to
simple safety properties. Thus, our label set has only two elements:
$\setof{\init,\unsafe}$. We then define $\CellLabelingFunction$ as
$\CellLabelingFunction(C) = \init$ if $C$ contains some initial state,
$\CellLabelingFunction(C) = \unsafe$ if $C$ contains some unsafe
state, and empty otherwise.

In what follows, we review the procedure
$\mathsf{ScatterAndSimulate}$($\System$,$\simmap_\Delta$,$\epsilon$)
from \cite{zutshi2014multiple} that, given the $\System$, the forward
simulation map $\simmap_\Delta$, and a precision $\epsilon$ creates an
$\epsilon$-TS abstraction of $\System$.  The
essence of the procedure is a breadth-first exploration of
$\ContStates$ to build $\TSAbstraction{\epsilon}{\System}$ =
$(\Cells,\CellEdges,\CellLabelingFunction)$. It consists of the
following steps:
\begin{enumerate}[leftmargin=1em,labelsep=1em,label={\arabic*.}]
\item
Add a set of random samples from the initial set of states to a queue
(\ie some states $\contState$ at time $0$).
\item
Pop the head of the queue (say state $\contState$ at time $t$), identify
the corresponding cell by applying $\quant_\epsilon$, and add it to
$\Cells$.
\item
Obtain $\contState' = \simmap_\Delta(\contState, t, u)$. Obtain a a
fixed number of states in the neighborhood of $\contState'$, and add
$\contState'$ and these states (all time $t+\Delta$) to the queue.
\item
Add an edge between the cells corresponding to $\contState$ and
$\contState'$.
\item
Go to step 2 till either the queue is empty or a fixed number of cells
or edges is added to $\TSAbstraction{\epsilon}{\System}$.
\end{enumerate}

\inclfig[width=0.95\linewidth]{vdp_abs.pdf}{The discovered
$0.2$-TS abstraction. Red cells are unsafe cells and green cells are
initial cells.}{vdp-abs}
\begin{example}[Van der Pol Oscillator]~\label{ex:vdp}
We now describe $\epsilon$-TS abstraction using the Van der Pol
    oscillator benchmark from~\cite{zutshi2014multiple}. Simulations
    generated using uniformly random samples are shown in
    \figref{vdp-cont}. We want to check the property $P_3$:
    $x_{unsafe}\in[-1,-6.5]$ and $y_{unsafe}\in[-0.7, -5.6]$, given
    the initial set indicated $x0\in[-0.4, -0.4],\; y_0\in[0.4, 0.4]$.
    We consider the $\epsilon$-TS abstraction defined by the
    quantization function $\quant_{0.2}(\x)$. This results in an
    evenly gridded state-space, where each cell is of size $0.2 \times
    0.2$ units.  $\mathsf{ScatterAndSimulate}$ is then used to
    construct $\TSAbstraction{0.2}{\System}$ using $2$ samples per
    cell and a time step of $\Delta = 0.1s$. The complete process
    follows.  The $\epsilon$-TS abstraction thus discovered is shown
    in \figref{vdp-abs}. The red cells are unsafe and green cells are
    initial cells.
\end{example}


\subsection{CEGAR for falsifying $\TSAbstraction{\epsilon}{\System}$}

We remark that our abstraction is {\em neither an overapproximation
nor an underapproximation} of the continuous-time continuous-space
reachability relation between states. The purpose of the $\epsilon$-TS
is to discover falsifying trajectories. This can be accomplished by
Algorithm~\ref{algo:s3camBMC} that uses CounterExample-Guided
Abstraction and Refinement (CEGAR).

Here, we construct $\TSAbstraction{\epsilon}{\System}$
(Line~\ref{algoline:construct}), we check if there is a path in
$\TSAbstraction{\epsilon}{\System}$ that leads to a cell labeled
unsafe. This can be done by a bounded model checking procedure
(Line~\ref{algoline:bmc}) that explores at most $\ell$ transitions of
$\TSAbstraction{\epsilon}{\System}$ (\ie, traces that are at most
$\ell\Delta$ seconds in length). The model checker may return a set of
counterexample trajectories $\Pi$. If this set is empty
(Line~\ref{algoline:nocex}), then either the exploration performed by
$\mathsf{ScatterAndSimulate}$ is not enough or the system could be
safe. In the former case, the user can increase the scope of
exploration for $\mathsf{ScatterAndSimulate}$.  The next step is to
check if the abstract trajectories are spurious, by performing a
concretization step by performing an actual simulation of the system
(Line~\ref{algoline:checkspurious}). If all abstract counterexample
trajectories are found to be spurious, then
$\TSAbstraction{\epsilon}{\System}$ can be refined, basically by
increasing $\epsilon$.
\begin{algorithm}[t]
\DontPrintSemicolon
\caption{CEGAR for $\TSAbstraction{\epsilon}{\System}$\label{algo:s3camBMC}}
\KwIn{$\System$, $\simmap_\Delta$, $\epsilon_0$ , Smallest Precision $\epsilonmin$}
$\epsilon \assign \epsilon_0$ \;
\While{$\epsilon \ge \epsilonmin$}{
    $\TSAbstraction{\epsilon}{\System}$ $\assign$
    $\mathsf{ScatterAndSimulate}$($\System$,$\simmap_\Delta$,$\epsilon$)
    \nllabel{algoline:construct} \;
    $\Pi$ $\assign$ $\mathsf{BoundedModelCheck}$($\TSAbstraction{\epsilon}{\System}$)
    \nllabel{algoline:bmc} \;
    \lIf {$\Pi = \emptyset$}{ return FAIL \nllabel{algoline:nocex} }
    \lIf {$\mathsf{checkIfConcretizable}(\Pi)$}{ return VIOLATION \nllabel{algoline:checkspurious} }
    $\epsilon$ $\assign$ $\frac{\epsilon}{10}$ \nllabel{algoline:refine} \;
}
\end{algorithm}

The biggest drawback of Algorithm~\ref{algo:s3camBMC} is that with
increasing $\epsilon$, the size of $\Cells$ in
$\TSAbstraction{\epsilon}{\System}$ increases exponentially.  Thus,
every subsequent run of the falsification algorithm is slower. In the
next section, we propose an alternative abstraction that enriches the
basic abstraction proposed here by adding labels to the transitions.
We can then perform refinement by enriching the transition relation,
rather than increasing $\epsilon$.

% \subsection{Example: Van der Pol Oscillator}
% \todo{Takes 4 iterations}
% \inclfig[width=0.90\linewidth]{vdp_disc_map.pdf}{Refinement of
% $\epsilon$-TS.}{vdp-map}
%
\begin{example}
\inclfig[width=0.90\linewidth]{vdp_disc_map.pdf}{
Abstract trajectories discovered by $\mathsf{ScatterAndSimulate}$. Each
rectangle denotes a cell which is uniquely identified  by the
tuple against it.
%Cells and trajectory segments used by $1$-relational modeling.
}{vdp-abs-paths}

    Going back to the~\exref{vdp}, $\mathsf{ScatterAndSimulate}$ is able to
    find several abstract trajectories as shown
    in~\figref{vdp-abs-paths}. However, typically, the refinement by
    increasing the quantization precision requires $4$ refinement
    iterations and finally finds a concrete counter-example for
    $\epsilon=0.125$.
\end{example}

%%%%%%%%%%%%%%%%%%%%%%%%%%%%%%

%%%%%%%%%%%%%%%%%%%%%%%%%%%%%%
\section{Kernelized Relational Transition Systems}
\label{sec:scamr}
In this section, we describe a {\em relational modeling} approach to
abstract a system. The technique described in this section can be
viewed as an enrichment of the procedure to construct an
$\epsilon$-precision transition system from the previous section.  The
central idea of this paper is that a discrete abstraction based on a
transition system can be augmented with additional information which
can greatly improve the accuracy of the abstraction. The refined
discrete abstraction takes the form of a $\epsilon$-precision {\em
relational} transition system $\RTSAbstraction{\epsilon}{\System}$. We
use the abbreviation $\epsilon$-RTS for $\epsilon$-precision
relational transition system.

\subsection{$\epsilon$-RTS}
Let $\exprs{\vx}$ denote a set of Boolean-valued expressions over the
variable $\vx$ and their Boolean combinations.  For example, if $\vx =
\setof{x_1,x_2}$, then $\exprs{\vx}$ could be $\setof{ x_1 > 0, x^2_1
+ x_2 > 0, x_2 < 1 \wedge x_1 + x_2 > -1}$.  An $\epsilon$-RTS
$\RTSAbstraction{\epsilon}{\System}$ is a labeled transition system
described by the tuple:
$(\Cells,\vx,\CellEdges,\InitCells,\CellLabelingFunction,
\EdgeLabelingFunction,\UnsafeStates)$, where:

\begin{itemize}[label=--,leftmargin=1em,labelsep=*]
\item
$\Cells$ is a set of cells,
\item
$\vx$ represents a variable that takes values in $\ContStates$,
\item
$\CellEdges$ is a finite subset of $\Cells \times \Cells$,
\item
$\InitCells$ is a set of intial cells,
\item
$\CellLabelingFunction$ is a function mapping a cell $C$ to a
an expression $e(\vx) \in \exprs{\vx}$,
\item
$\EdgeLabelingFunction$ maps an edge $(C,C')$ to a pair $(g(\vx),
r(\vx,\vxp))$, where the {\em guard relation} $g(\vx) \in \exprs{\vx}$
and the {\em reset relation} $r(\vx,\vxp) \in \exprs{\vx,\vxp}$, and
$\vxp$ is a special variable.
\item
$\UnsafeStates \subset \ContStates$ is a set of unsafe states.
\end{itemize}


The operation of $\RTSAbstraction{\epsilon}{\System}$ is described in
terms of its configuration and its moves. A configuration is a pair
$(C, \nu)$, where $C \in \Cells$, and $\nu$ is a valuation function
assigning values in $\ContStates$ to the variable $\vx$.  The
transition system starts in some cell $C \in \InitCells$.  The move
function $\mu$ maps a configuration $(C,\nu)$ to the next
configuration $(C',\nu')$, iff there is a transition $(C,C') \in
\CellEdges$ such that:
\begin{enumerate}
\item
There exists $\contState \in \ContStates$ such that
$\quant_\epsilon(\contState) = C$ and exists $\contState' \in
\ContStates$ such that $\quant_\epsilon(\contState') = C'$,
and $\contState' = \simmap_\Delta(\contState, t, u)$,
\item
If $\CellLabelingFunction(C) =  e(\vx)$, and if
$\CellLabelingFunction(C') = e'(\vx)$, then for $\nu(\vx) =
\contState$ and $\nu(\vx') = \contState'$, the expressions
$e(\nu(\vx))$ and $e'(\nu'(\vx))$ evaluate to true,
\item
If $\EdgeLabelingFunction((C,C'))$ = $(g(\vx), r(\vx,\vxp))$,
for $\nu(\vx) = \contState$ and $\nu'(\vxp) = \contState'$,
$g(\nu(\vx))$ and $r(\nu(\vx), \nu'(\vx))$ evaluate to true.
\end{enumerate}

\begin{example}
Consider a system with the following two reachability relations
discovered using simulations. The values of the inputs are irrelevant
to this example, so we just assume it is some fixed constant $u$.
Also, let $\Delta = 0.1$.
\begin{eqnarray}
\label{eq:transitions}
(0.05, 0.1) & = & \simmap_\Delta((0.1, 0.1), 0, u), \\
(1.35,1.15) & = & \simmap_\Delta((0.9, 0.8), 0.1, u).
\end{eqnarray}

Consider the following $\RTSAbstraction{\epsilon}{\System}$ where,
$\Cells = \setof{C_1,C_2}$, $\vx = (x_1,x_2)$, $\InitCells = \setof{C_1}$,
$\CellEdges = \setof{(C_1,C_1), (C_1, C_2)}$, and,
\[
\begin{array}{l}
\CellLabelingFunction(C_1) = (0 \le x_1 < 1) \wedge (0 \le x_2 < 1), \\
\CellLabelingFunction(C_2) = (1 \le x_1 < 2) \wedge (1 \le x_2 < 2), \\
\EdgeLabelingFunction((C_1,C_1)) = (g_1,r_1), \text{ where, } \\
\qquad g_1 = (0 \le x_1 < 0.5) \wedge (0 \le x_2 < 1), \\
\qquad r_1 = (0 < x^+_1 - 0.5x_1 < 0.1) \wedge (0 < x^+_2 - x_2 < 0.001), \\
\EdgeLabelingFunction((C_1,C_2)) = (g_2,r_2), \text{ where, } \\
\qquad g_2 = (0.5 \le x_1 < 1) \wedge (0 \le x_2 < 1), \\
\qquad r_2 = (0.4 < x^+_1 - x_1 < 0.5) \wedge (0.3 < x^+_2 - x_2 < 0.4). \\
\end{array}
\]

Observe that the $\epsilon$-RTS is a valid abstraction of the given
trajectory fragments in Eq.~\ref{eq:transitions}.
\end{example}




% % We define a PWA transition system $A$ as a tuple
% % $\tupleof{\pLocs,\vx,\scrT, G, U, \pLocI,\Theta}$, where:
% % \begin{itemize}[noitemsep, leftmargin= 1.5 em]
% % \item
% % $\pLocs$ is a set of locations,
% % \item
% % $\vx$ is a variable that takes values in $\reals^m$,
% % \item
% % $\scrT$ is a finite subset of $\pLocs \times G \times U \times G \times \pLocs$,
% % \item
% % $G$ is a set of conjunctions of affine predicates,
% % \item
% % $U$ is a set of affine relations,
% % \item
% % $\pLocI$ is an initial location, and,
% % \item
% % $\Theta$ is an initial affine predicate.
% % \end{itemize}

% The operation of $A$ is described in terms of its configuration and
% its moves. A configuration of $A$ is a pair $(\ell, \nu)$, where $\ell
% \in \pLocs$, and $\nu$ is a valuation function assigning values in
% $\reals^m$ to the variable $\vx$.  The move function $\mu$ maps a
% configuration $(\ell,\nu)$ to the next configuration $(\ell',\nu')$,
% iff there is a transition $\tau$ of the form $(\ell,g,f,g',\ell')$,
% and $\nu(\vx)$ satisfies predicate $g$, $\nu'(\vx)$ satisfies
% predicate $g'$ and $\nu'(\vx) \in f(\nu(\vx))$.

% For a given $\tau$, we call the affine relation $f_\tau$ its update
% relation, and the predicates $g_\tau$ and $g'_\tau$ as pre- and
% post-conditions on the transition. The machine starts its operation in
% location $\pLocI$ with an initial valuation $\nu_0$ to the variable
% $\vx$ such that $\nu_0(\vx)$ satisfies $\Theta$. For convenience, we
% abuse notation, and use the name a variable as a proxy for a certain
% valuation, and use the primed version of the variable as a proxy for
% its valuation after a transition, \ie, $\vx$ is used instead of
% $\nu(\vx)$ and $\vx'$ is used instead of $\nu'(\vx)$.

% % We define the PWA transition system as a discrete transition system
% % $\rho:\tupleof{\pLocs,\vx,\scrT, \pLocI,\Theta}$, where $\pLocs$ is
% % a nonempty set of locations, $\vx$ is a set of variables, $\scrT$
% % is a finite set of transitions of the form $\tupleof{l, \tau,
% % l'}$, where $l\in\pLocs$ is the pre-location, $l'\in\pLocs$ the
% % post-location and $\tau \in \scrT$ is the transition. $\tau$
% % is labeled by the transition relation $\rho_{\tau}$
% % over the current states and the next states. The transition relation
% % is defined in terms of affine predicates on the current and next
% % states and an affine update between them.
% % \[
% %     \rho_{\tau}(\vx,\vx') \subseteq \setof{(\vx,\vx')) \;|\;
% %     g_\tau(\vx) \land g'_\tau(\vx') \land f_{\tau}(\vx, \vx')}
% % \]
% % where $f_{\tau}$ is an affine relation on the pre and post states
% % $\vx,\vx'$, $g_{\tau}$ and $g'_{\tau}$ are affine guards (pre and
% % post conditions) on the states. Finally, $\pLocI$ is an initial
% % location $\Theta$ is an affine assertion characterizing the initial
% % states included in $\pLocI$.


% % A configuration of the PWATS is
% % a pair (l,\nu(v)), where \nu is a valuation function assigning values to
% % variables.

% % A run of the PWA TS is as follows: the PWA TS starts in the initial
% % location from a valuation \nu(v) \in theta.

% % A move of the PWA TS is described by the function mu: from (L,R^n) to
% % (L,R^n), and mu follows the rules that
% % mu(l,\nu(v)) = (l',\nu'(v)), iff (l,g,g',f,l') in T, and
% % \nu(v) satisfies g, \nu(v') satisfies g', and \nu(v') = f(\nu(v)).

% We how we can use the formal machinery of a PWA transition system $A$
% to approximate the behavior of a hybrid dynamical system $\System$
% defined by a simulator $\simulate^{\System}$~\footnote{Due to $f_\tau$
% being restricted to an affine form, we can only approximate general
% hybrid systems.}.


% %defines a set valued relation of the form $R:\setof{(\x,\x') \;|\; \x'
% %\in f_\tau(\x)}$.


% % Given a state vector $\x\in\reals^n$, a guarded affine map
% % $\gm:(\guard, \amap)$ defines the discretized consecution rule as a
% % pair of an affine guard predicate $\guard:C\x - \vd \le 0$ and an affine map
% % $\amap:A\x+\vb$, where $A \in \reals^{n \times n}$, $C \in
% % \reals^{m \times n}$ are matrices and $\vb \in \reals^n$, $\vd
% % \in \reals^m$ are vectors. A guarded affine map is satisfied if its
% % guard is satisfied.

% We now formalize the PWA relational model for a dynamical system.
% \begin{definition}[PWA Relational Model]

% A PWA relational model $A(\System)$ of a hybrid dynamical system
% $\System$ is simply a PWA transition system, where:

% \begin{itemize}[noitemsep, leftmargin= 1.5 em]
% =======

% for the $N$ tuples $(v,v')$ discovered by simulation, if $v' = Av
% + B + \vec{\delta}$ represents the affine relation discovered using
% linear regression, and if $h$ is the convex hull of all the $v$
% states across all tuples and $h'$ is the convex hull of all the $v'$
% states across all tuples, then $f_\tau \subseteq
% \setof{(\x,\x') \mid \x' \in A\x + B + \vec{\delta}}$,
% and $g$ states that $\x$ is in $h$, and $g'$ states that
% $\x' \in h'$.
% \end{itemize}

% Note that the guard predicates $g_\tau$ represent a conjunction of $m$
% affine inequalities, then we can represent $g_\tau$ as follows:
% \[
% g_\tau  \equiv C\x - \vd \le 0
% \]
% where $C$ is an $m \times n$ matrix and $\vd$ is $m \times 1$ matrix.


% % Given a hybrid dynamical system over a state-space $\reals^n$, a PWA
% % relational model $\rho$ is given by the tuple
% % $\tupleof{\pLocs,\x,\scrT,\pLocI,\Theta}$, where $\tau \in \scrT$ is a
% % collection of affine transitions  and $\Theta$ is an affine predicate
% % over $\x\in\real^n$. The transition relation is then defined by
% % $\scrT$ with $n$ transition relations as follows
% %
% % \begin{equation}
% %     \scrT = \left\{
% %         \begin{array}{ll}
% %             g_1(\x) \land g'_1(\x') \implies f_1(\x,\x') \\
% %             \ldots\\
% %             g_n(\x) \land g'_n(\x') \implies f_n(\x,\x') \\
% %         \end{array}
% %     \right.
% % \end{equation}
% \end{definition}

% % where $g_i$ and $g'_i$ are affine predicates and $f_i$ are affine
% % relations and $\x'$ denotes the next state of the system.

% A PWA relational model is \textit{deterministic}, iff for every state
% $\x\in\reals^n$, a unique guarded affine map is satisfied and if all
% entries in the error interval vector $\vec{\delta}$ are singletons.
% However, in practice such a case rarely exists. A PWA model will be
% usually non-deterministic, both due to multiple choices of transitions
% from any location and the reachable states at the $k^{th}$ time step
% (or after $k\Delta$ units of time) being a set.  Abusing the notation,
% we denote the set of states reachable in a single step in $\rho$ by
% $\x' = \rho(\x)$.

% A PWA relational model system is \textit{complete}, iff for every
% state $\x\in\HybridStates^n$, there exists at least one satisfied
% transition relation $\scrT$. If this is not the case, the system can
% deadlock, with no further executions. This usually results from
% modeling errors, and from here on, we do not consider such cases.

% In what follows, we use the terms PWA relational model and PWA
% transition system interchangeably to mean PWA relational model.

% \subsection{Bounded Model Checking (BMC)}
% The PWA transition system is a finite location, infinite state
% transition system. As all the guards and transitions are affine, a
% bounded reachability query is equivalent to checking all
% combinations of discrete locations, where each combination can be
% summarized as a linear program. As the time is bounded, the number of
% combinations are finite, but, exponential in number.

% It should be noted that an explicit execution of the transition system
% represents a directed acyclic graph, with each location being a node,
% and a directed edge from each node to all nodes reachable in forward
% time. A path on this graph is then a linear program, with each
% branching node corresponding to a logical disjunction.

% Temporal properties of PWA transitions systems can be checked using
% off-the-shelf model checkers like SAL~\cite{SAL-SRI}, which can reason
% over infinite state transitions systems using SMT solvers.
% Furthermore, lazy SMT solvers can be employed for this specific
% problem instance to achieve better efficiency~\cite{shoukry2017smc}.
% We now show how relational PWA abstractions can be viewed as PWA
% transition systems, and how the problem of falsification can
% be answered by a BMC query.

% % Given a dynamical system over a continuous state-space
% % $\ContStates\in\reals^n$, a PWA model is a map $\ContStates \mapsto
% % \setof{\gm_1, \gm_2, \ldots, \gm_n}$ from the state-space of the
% % dynamical system to a finite set of guarded affine maps
% % $\setof{\gm_1, \gm_2, \ldots, \gm_n}$. It defines the consecution
% % rule by the affine map of the satisfied $\gm$.

% % \begin{equation}
% %     \pwa = \left\{
% %         \begin{array}{ll}
% %             \gm_1: \amap_1(\x) &\text{if}\; \guard_1(\x) \\
% %             \gm_2: \amap_2(\x) &\text{if}\; \guard_2(\x) \\
% %             \ldots & \ldots\\
% %             \gm_n: \amap_n(\x) &\text{if}\; \guard_n(\x) \\
% %         \end{array}
% %     \right.
% % \end{equation}

% % \end{definition}

% % where $\amap_i = A_i\x + \vb_i$, and $\guard_i(\x) \equiv C_i\x-\vd_i\le0$.
% % Abusing the notation, we denote the next state computed by the PWA
% % model as $\x' = \pwa(\x)$. A PWA model is \textit{deterministic}, iff
% % for every state $\x\in\ContStates$, a unique guarded affine map is
% % satisfied. A PWA model is \textit{complete}, iff for every state
% % $\x\in\ContStates$, there exists at least one satisfied guarded affine
% % map $\gm$.


% \section{Learning Dynamics using Linear Regression}

% The approach in \cite{zutshi2014multiple} uses the forward simulation
% map to explore a fixed number of tuples in the reachability relation
% $\reach{t,\vu}$ specified by a user-provided budget.  In general, the
% budget required to adequately explore the transition space requires
% exploring an exponential (in the dimension of $\HybridStates$) number
% of such tuples. This can cause undue burden in terms of computation
% time as well as storage.  The central idea in this paper is to {\em
% summarize} several tuples as a single affine map obtained using linear
% regression.  In other words, we can replace several ``proximal''
% tuples $(\vx,\vx')$ in $\reach{t,u}$ by an affine map of the form
% $\vx' = A\vx + B\vu + \delta$, where $A$ and $B$ are matrices and
% $\delta$ is a worst-case error estimate.

% In statistics, linear regression is the problem of finding an affine
% function or \textit{predictor}, which can `best' summarize the
% relationship between the given set of observed input $\x$ and output
% $\x'$ vectors. The notion of `best' is formally captured using a loss
% function (which can be non-linear), resulting in different kinds of
% regression analysis including ridge, lasso, huber, generelized and
% many other regressions. For simplicity, we present the commonly used
% loss function \textit{ordinary least squares} (OLS) in the next
% section.
% ~\footnote{The choice of regression analysis should ideally be based on the
% quality of data. As we assume the data (from the simulator) to be
% noise free, the affecte on result of the falsification search is
% minimal.
% Though it is important,  not central to the
% discussion as long as the learnt models  can be used to learn different
% models}.

% \mypara{Ordinary Least Squares (OLS)}
% For rest of the section, we elide the discrete mode from the hybrid
% state of the system, and only focus on the continuous state. Note that
% we can do this because of our assumption that the system is a
% switched-mode dynamical system.  Suppose we have $N$ tuples of the
% form $(\x,\x')$ in the reachability relation that we have explored,
% where $\x\in\reals^n$ and $\x'\in\reals^n$. Then, we wish to learn an
% affine function described as
% \begin{equation}
% \label{eq:affinemap}
% \x' = A\x + B
% \end{equation}
% Here, $A$ is a $n\times n$ matrix, and $B$ is a $n\times 1$ matrix.


% If $N>n$, which is the case in the current context of \textit{finding
% the best fit}, the problem is over-determined: there are more
% equations than unknowns. Hence, a single affine function cannot be
% found which satisfies the equation \eqref{eq:affinemap}. Instead, we
% need to find the `best' choice for $\vec{a}$ and $b$. This choice is
% formally defined using a loss function. For the case of OLS, the loss
% function is the sum of squares of the errors in prediction.  The task
% is then to determine $A$ and $B$, such that the least square error of
% the affine predictor is minimized for the given data set.

% \begin{equation}
%     \operatornamewithlimits{argmin}_{A,B}\displaystyle\sum_{i=1}^{N}{\left(\x' - (A\x + B)\right)^2}
% \label{reg}
% \end{equation}

% To ease the presentation, we can rewrite the above as a homogeneous
% expression by using $\hat{\x} = [\x^T\ 1]^T$ and replacing $A$ and $B$ by the vector
% $\vec{A} = [A\ B]$. The equation~\ref{reg} now becomes
% \[
% \operatornamewithlimits{argmin}_{\vec{A}}\displaystyle\sum_{i=1}^{N}{(\x' - \vec{A}^T \hat{\x})^2}
% \]

% The solution of OLS can be analytically computed as the Moore-Penrose
% pseudo-inverse: $\vec{A} = (X^TX)^{-1}X^T X'$, where $X$ is the matrix
% representing the horizontal stacking of all $\hat{\x}$ and $X'$
% represents the stacking of all $\x'$ values.  The details can be found
% in standard texts on learning and statistics
% \cite{friedman2001elements}. Alternative methods based on gradient
% descent can also be used to find a suitable $A$ matrix.

% In this fashion, we can we can use OLS to approximate its trajectories
% of system $\System$ at fixed time step $\Delta$ by a discrete map $\x'
% = \vec{A}\hat{\x}$.

% % where
% % $A = \begin{bmatrix}\vec{a}_1^T \\ \vdots \\ \vec{a}_n^T
% % \end{bmatrix}$ and $\vb = \begin{bmatrix}b_1 \\ \vdots \\ b_n
% % \end{bmatrix}$.

% We note that linear regression tools are often able to provide an
% $n\times 1$ vector $\vec{\delta}$ of error intervals, where the
% $i^{th}$ entry of the form $[{\delta_i}_{\min},{\delta_i}_{\max}]$
% indicates the best and worst possible error in computation of the
% $i^{th}$ state in $\x'$.


% Affine maps are poor approximations for arbitrary non-linear
% functions. Hence, we use a collection of affine maps to approximate
% the local behaviors (in state-space) of the system $\System$. This
% results in a piece wise approximation, as described in the next
% section.


% % \mypara{Regression Analysis}

% % In statistics, regression is the problem of finding a
% % \textit{predictor}, which can suitably predict the relationship
% % between the given set of observed input $\x$ and output $\y$ vectors.
% % In other words, assuming that $\y$ depends on $\x$, regression
% % strategies find either a parameterized or a non-parameterized
% % prediction function to explain the dependence. We now discuss simple
% % linear regression, which is parametric in nature and searches for an
% % affine predictor. It is also called \textit{ordinary least squares}.

% % \mypara{Ordinary Least Squares (OLS)}

% % Let the data set be comprised of $N$ input and output pairs $(\x,
% % y)$, where $\x\in\reals^n$ and $y\in\reals$. If $N>n$, which is
% % the case in the current context of \textit{finding the best fit}, the
% % problem is over-determined; there are more equations than
% % unknowns. Hence, a single affine function cannot be found which
% % satisfies the equation $\forall i\in\{1\ldots N\}. y_i = \vec{a}^T\x_i + b$. Instead, we
% % need to find the `best' choice for $\vec{a}$ and $b$. This is formally
% % defined using a loss function. For the case of simple linear
% % regression or OLS, the loss function is the sum of squares of the
% % errors in prediction.  The task is then to determine the vector of
% % coefficients $\vec{a}$ and an offset constant $b$, such that the least
% % square error of the affine predictor is minimized for the given data
% % set.
% % \begin{equation}
% %     \operatornamewithlimits{argmin}_{\vec{a}, b}\displaystyle\sum_{i=1}^{N}{\left(\y_i - (\vec{a}^T\x_i + b)\right)^2}
% % \label{reg}
% % \end{equation}
% % To ease the presentation, we can rewrite the above as a homogeneous
% % expression by augmenting $\x$ by $\hat{\x} = \begin{bmatrix}\x \\ 1\end{bmatrix}$ and replacing $\vec{a}$ and $b$ by the vector
% % $\vec{ab} = \begin{bmatrix}\vec{a} \\ b \end{bmatrix}$. The equation~\ref{reg} now becomes
% % \[
% %     \operatornamewithlimits{argmin}_{\vec{ab}}\displaystyle\sum_{i=1}^{N}{(\y_i - \vec{ab}^T \hat{\x}_i)^2}
% % \]
% % The solution of OLS can be analytically computed as
% % \[ Ab = (X^TX)^{-1}X^T\y\]
% % where $X$ is the matrix representing the horizontal stacking of all
% % $\hat{\x}$.
% % The details can be found in several texts on learning and statistics
% % \cite{friedman2001elements}.

% % Given a time invariant dynamical system $\x' = \simulate(\x, \Delta)$,
% % where $\x\in\reals^n$ we can use OLS to approximate its trajectories
% % at fixed time step $\Delta$ by a discrete map $\x' = A\x + \vb$, where
% % $A = \begin{bmatrix}\vec{a}_1^T \\ \vdots \\ \vec{a}_n^T \end{bmatrix}$ and $\vb =
% % \begin{bmatrix}b_1 \\ \vdots \\ b_n \end{bmatrix}$. This map is a
% %     global relational model for the system. The data set for the
% %     regression is a set of pairs $\setof{(\x,\x') | \x' =
% %     \simulate(\x, \Delta)}$ and can be generated on demand. Another
% %     set can be generated to compute the error $\delta$ as an interval
% %     of vectors, where each element is an interval $\delta_i \in
% %     [\delta_{min}, \delta_{max}]$.
% %
% %     Affine maps are poor
% %     approximations for arbitrary non-linear functions. Hence, we use a
% %     collection of affine maps to approximate the local behaviors (in
% %     state-space) of the system $\simulate$. This results in a piece
% %     wise approximation, as described in the next section.



% % \subsection{Relational Abstractions}
% % For a hybrid automaton model, they can summarize the continuous
% % dynamics of each mode using a binary relation over continuous states.
% % The resulting relations are timeless (independent) and hence valid for
% % all time as long as the mode invariant is satisfied. The relations
% % take the general form of $R(\x,\x') \bowtie 0$, where $\bowtie$
% % represents one of the relational operators $=, \ge, \le, <, >$. For
% % example, an abstraction that captures the monotonicity with respect to
% % time for the differential equation $\dot{x} = 2$ is $x' > x$. The
% % abstraction capturing the relation between the set of ODEs: $\dot{x} =
% % 2$, $\dot{y} = 5$, is $5(x' - x) = 2(y' - y)$.

% %%  Similar to timeless relational abstractions, time-aware relational
% %%  abstractions \cite{mover2013time} construct binary relations between
% %%  the current state $\x$ of the system and any future reachable state
% %%  $\x'$. The difference lies with the latter also constructing relations
% %%  between the current time $t$ and any future time $t'$. This is
% %%  achieved by a case by case analysis of the eigen structure of the
% %%  matrix $A$ (of an affine ODE). Separate abstractions are used for the
% %%  case of linear systems with constant rate, real eigen values and
% %%  complex eigenvalues.


\section{Constructing Relational Abstractions}
The $\mathsf{ScatterAndSimulate}$ procedure helps in constructing the
$\epsilon$-TS abstraction using a user-specified budget of trajectory
fragments (obtained using simulation).  In general, the budget
required to adequately explore the transition space requires exploring
a number of states that is exponential in the dimension of
$\HybridStates$.  This can cause undue burden in terms of computation
time as well as storage.  The central idea in this paper is to {\em
summarize} enrich the transitions in the $\epsilon$-TS abstraction
using relations to obtain an $\epsilon$-RTS. 

In other words, suppose there is an edge from cell $C$ to $C'$, and if
$\vx \in C$ and $\vx' \in C'$, then we can consider all concrete
trajectories that go from $C$ to $C'$, and use these to replace
several ``proximal'' tuples $(\vx,\vxp)$ by a map $F$ of the form
$\vxp = F(\vx, \vu) \pm \delta$, where $\delta$ is a worst-case error
estimate. The form of $F$ depends on the basis function used for
regression. We focus on linear regression with different basis
functions such as affine and polynomial. For the special case of
linear regression, the resulting update relation can be represented by
matrices $A$ and $B$, \ie, $r(\vx,\vxp) = \setof{(\vx,\vxp) \mid |\vxp
- (A\vx + B\vu)| < \delta}$.

In statistics, parametric regression is the problem of finding the
paremters of a function or \textit{predictor}, which can `best'
summarize the relationship between the given set of observed input
$\x$ and output $\x'$ vectors. The basis functions used for the
predictor are fixed beforehand and the paramters are searched for.
The notion of `best' is formally captured using a loss function which
can be non-linear. We use the commonly used loss function
\textit{ordinary least squares} (OLS) to fit affine and polynomial
kernels of fixed degree.  ~\footnote{The choice of loss function and
regression analysis affects the quality of the learnt models. As the
procss of selecting an appropriate heuristic is not straightforward,
we just use the simplest one.} OLS defines its loss function as the
sum of squares of the errors in prediction. The task is then to
determine the parameters, such that the least square error of the
predictor is minimized for the given data set.  The details on
performing linear regressoin can be found in standard texts on
learning and statistics~\cite{friedman2001elements}.

% Suppose we have $N$ tuples of the
% form $(\vx,\vxp)$ in the reachability relation that we have explored,
% where $\x\in\reals^n$ and $\x'\in\reals^n$. Then, we wish to learn a
% function described as $\x' = F(\vx, \vu)$.  


% In this fashion, we can we can use OLS to approximate its trajectories
% of system $\System$ at fixed time step $\Delta$ by a discrete map $\x'
% = F(\hat{\x})$.

We note that regression tools are often able to provide an
$n\times 1$ vector $\vec{\delta}$ of error intervals, where the
$i^{th}$ entry of the form $[{\delta_i}_{\min},{\delta_i}_{\max}]$
indicates the best and worst possible error in computation of the
in $\vxp_i$.

The choice of basis functions for constructing an $\epsilon$-RTS
involves a tradeoff between computing precise local models and
efficiency of the analysis of the resulting relational abstraction.
Affine basis functions can be poor approximations for complex dynamics
and can result in large errors, but they can be very efficiently
analyzed in practice.  On the other hand, analysis of arbitrary
non-linear functions is not generally tractable. We restrict ourselves
to affine and quadratic polynomial basis functions.  Moreover, as we
use a collection of relations to approximate the local behaviors (in
state-space) of the system $\System$, we can often do so with simple
basis functions. 

\begin{example}
    For the~\exref{vdp}, for every abstract relation between two cells
    $C, C'$, linear regression analysis is performed on the respective
    set of trajectory segments, and the affine relation is computed.
    \figref{vdp-abs-paths} shows the cells and the trajectory
    segments, which are part of the data sets constructed using the
    $1$-relational modeling. Against each cell, its unique identifier
    (integer co-ordinate) is mentioned.
\end{example}

%%%%%%%%%%%%%%%%%%%%%%%%%%%%%%

%%%%%%%%%%%%%%%%%%%%%%%%%%%%%%
\section{Refining a Relational Model}
\label{sec:rel-mod}
\begin{figure}[!htbp]
\vspace*{-.2cm}
\begin{center}
\tikzstyle{line} = [thick]
\tikzstyle{arw} = [->, thick,>=stealth,shorten <=2pt, shorten >=2pt]
\begin{tikzpicture}

\begin{scope}[scale=0.8]
\draw [line] (-1.0,0) circle (0.5);
\draw [line] (-4.0,0) circle (0.5);
\draw [line] (2.0,1) circle (0.5);
\draw [line] (2.0,-1) circle (0.5);
\draw[arw] (-0.5,0) -- (1.5,1.0);
\draw[arw] (-0.5,0) -- (1.5,-1.0);
\draw[arw] (-3.5,0) -- (-1.5,0.0);
\node at (-4.0,0) {$C$};
\node at (-1.0,0) {$C'$};
\node at (2.0,1.0) {$C''_1$};
\node at (2.0,-1.0) {$C''_2$};
\draw [line] (2.0,-1) circle (0.5);
\draw[arw] (-5.0,-2.0) -- (3.0,-2.0);
\node at (-4.0,-2.5) {$k=0$};
\node at (-1.0,-2.5) {$k=1$};
\node at (+2.0,-2.5) {$k=2$};
\draw [fill=black] (-4.0,-2.0) circle (0.05);
\draw [fill=black] (-1.0,-2.0) circle (0.05);
\draw [fill=black] (2.0,-2.0) circle (0.05);

\end{scope}
\end{tikzpicture}
\end{center}
\vspace*{-.3cm}
\caption{Paths through an $\epsilon$-RTS}
\label{fig:k-rel}
\vspace*{-.3cm}
\end{figure}

% We now describe how relational models can be computed for a given
% black box system by enriching the abstract reachability graph. We
% first formalize the notion of an enriched graph, and then show how
% they can be interpreted as a PWA transition system using relational
% modeling. Finally, we show how the generalization of relational
% modeling: $k$-relational modeling can be used to refine the relational
% abstraction.

% The existential abstraction relation in \cite{zutshi2014multiple} was defined as
% follows:
% \[
% %\exists \x\in\C. \exists \x'\in\C'. \x'=\simulate(\x, \tau) \iff \C\rel{\tau}\C'
%     C \areach{t} C' \iff \exists \x \in C.\ \exists \x' \in C'.\ \x \areach{t} \x'
% \]
% The abstract relation $\areach{t}$ can be enriched by incorporating
% the affine relation between $\x$ and $\x'$. For an arbitrary dynamical
% system, such a relation can be rarely represented using an exact affine map.
% This is due to the presence of non-linear and hybrid behaviors. But,
% an affine map can always be estimated with an error.
% 
% If the system dynamics are completely specified in the form of a white
% box model, we can use first order approximations to find the affine
% expressions for the relations. Using a tool like \flowstar, we can
% obtain sound over-approximate affine maps of the form $A\x + [\vb^l,
% \vb^h]$, where $[\vb^l,\vb^h]$ denotes the interval of vectors, such
% that, every element $b_i$ of the vector $\vb$ is contained in the
% respective scalar interval $b_i\in[b^l_i,b^h_i]$.
% 
% For the case of black box systems, such a sound approximation is not
% possible. Instead, we rely on statistical methods like simple linear
% regression to estimate the affine relation $f(\vx,\vx'): \x'=A\x + \vb$. We then
% estimate the error $\delta$ and generalize $\amap$ to an interval
% affine map as before $f(\vx,\vx'): \x' \in A\x + \vb + \delta$.  Finally, we get an
% abstraction with the below relation
% \[
%     C \areach{f} C' \iff \exists \x \in C.\ \exists \x' \in C'.\ \x' \in A\x + \vb \pm \delta.
%     %C \areach{A\x+\vb+\delta} C' \iff \exists \x \in C.\ \exists \x' \in C'.\
% \]
% 
% However, in the presence of complex non-linear behavior, using a
% single affine map can lead to a poor approximation. To increase the
% precision, one can use more than one affine map. Let us denote this set as
% $R_{(C,C')}$.% (ref.~\secref{ovw}).
% \[
%     R_{(C,C')}(\x,\x'): \setof{(\vx,\vx') \;|\; \exists \x \in C.\ \exists \x' \in C'.\ f(\x,\x')}
% \]
% 
% % \begin{figure}[!htbp]
% % \begin{center}
% % \tikzstyle{line} = [thick]
% % \tikzstyle{arw} = [->, thick,>=stealth,shorten <=2pt, shorten >=2pt]
% % \begin{tikzpicture}
% 
% %     \draw [line] (-1.2,-1.2) rectangle (1.2,1.2);
% %     \draw[arw] (-3,0) -- (-1.2,0);
% %     \draw[arw] (1.2,0) -- (3,0);
% 
% % \node at (-3.4,0) {$G$};
% % \node at (3.4,0) {$G^R$};
% % \node[align=center] at (0, 0) {\scriptsize Enrich using \\ \scriptsize Regression \\ \scriptsize(OLS)};
% 
% % \end{tikzpicture}
% % \end{center}
% % \vspace*{-.3cm}
% % \caption{Using OLS, $G^R$ is computed by determining the appropriate
% % $R_{(C,C')}$ for each edge of $G$.}
% % \label{fig:enrichment}
% % \vspace*{-.3cm}
% % \end{figure}
% 
% As shown in \figref{enrichment}, we use OLS to compute an enrichment
% of the abstraction graph $G$. For every edge $(C,C') \in edges(G)$,
% the set of relations $R_{(C,C')}(\x,\x')$ is computed. The semantics
% of the enriched edge are non-deterministic. Form a state $\x
% \in C$, any $f\in R_{(C,C')}$ can be taken as long as $\x' \in C$'
% is satisfied. This interpretation results in an infinite state
% transition system. We now detail the construction of the set
% $R_{(C,C')}$ using $k$-relational modeling.
% 
% % We now compute a PWA model for a given black box system by estimating
% % the dynamics using OLS. We can estimate the affine relations between
% % every cell, but as discussed in ~\chapref{abs}, it is a futile
% % approach.  Instead, we use the same heuristic as before:
% % scatter-and-simulate, to select the cell relations for estimation. We
% % build the reachability graph as before, but in addition, annotate each
% % edge with the values of estimated $A$, $\vb$, and $\delta$ for the
% % respective relation. The reachability graph thus computed, is a
% % transition system. Using off-the-shelf bounded model checkers, we
% % reason about the system's safety properties.

In this section, we show a technique to refine an $\epsilon$-RTS that
does not incur the curse of dimensionality faced by the CEGAR-based
refinement procedure for $\epsilon$-TS abstractions. The main idea is
still to split cells of the $\epsilon$-RTS; however, cells are not
split spatially.  Instead, we re-define a cell as a tuple, the first
element of which is the original cell number (indicating the
equivalence class generated by the quantization function), and the
subsequent $k$ elements indicate the indices of the $k$ cells through
which a trajectory originating in the cell may pass. We call such a
representation a $k$-relational $\epsilon$-RTS, or $(k,\epsilon)$-RTS
for short. 

To further, explain, let us consider a $2$-relational $\epsilon$-RTS.
Consider Figure~\ref{fig:krela}. The intuitive idea for the
$k$-relational $\epsilon$-RTS abstraction is that a trajectories that
follow the path $C \rightarrow C' \rightarrow C_1''$ typically obey a
{\em different} guard/update relation going from cell $C$ to $C'$ than
trajectories following the path $C \rightarrow C' \rightarrow C_2''$.
Thus, based on the cell of eventual destination, we can split the cell
$C$ into two cells: $(C,C',C_1'')$ and $(C,C',C_2'')$. Furthermore, we
can identify different guard/update relation on the transition from
these cells to cells of the form $(C',*,*)$ as shown in
Figure~\ref{fig:krelb}. Here $*$ denotes any arbitrary cell. In other
words the ``trajectory bundle'' corresponding to simulations from cell
$C$ to $C'$ now is split into two bundles, based on where the
trajectories from each bundle may end up in two steps.

Suppose we are given an $\epsilon$-RTS
$(\Cells,\vx,\CellEdges,\InitCells,\CellLabelingFunction,\EdgeLabelingFunction,\UnsafeStates)$,
then, a $(k,\epsilon)$ is defined as a tuple: $(\kCells, \vx,
\kCellEdges, \kInitCells, \kCellLabelingFunction,
\kEdgeLabelingFunction)$, where:

\begin{itemize}[label=--,leftmargin=1em,labelsep=*]
\item
$\kCells \subset \Cells^k$, \\
\item
$\kCellEdges \subset \kCells \times \kCells$, \\
\item
$(C_1,\ldots,C_k) \in \kInitCells$ iff $C_1 \in \InitCells$, \\
\item
$\kCellLabelingFunction((C_1,\ldots,C_k))$ =
$\CellLabelingFunction(C_1)$, \\
\item
$\kEdgeLabelingFunction((C_1,\ldots,C_k),(C'_1,\ldots,C'_k)) =
(g(\vx),r(\vx,\vxp))$, as before.
\end{itemize}

The operational semantics of a $(k,\epsilon)$-RTS are similar to that
of an $\epsilon$-RTS. The key difference is in how it is constructed.
We note that the $\epsilon$-RTS introduced in Sec.~\ref{sec:scamr} is
basically a $1$-relational $\epsilon$-RTS.


% 
% Given $G$, when computing the set of relations $R$ for an edge, we
% consider only the set of abstract states reachable at a specific time
% step.  Intuitively, one might consider the states reachable in one
% time step $\Delta$.  However, such a process can be generalized by
% looking at the states reachable at ${0\Delta, 1\Delta, \ldots,
% k\Delta}$ time steps. We denote these states as $\vx, \vx', \ldots
% \vx^k$.

% Using \figref{k-rel}, we illustrate this notion. To compute
% $R_{(C,C')}$, we
% split the data set $\ds(C,C')$ consisting of trajectory segments of time lengths
% $k\Delta$ as follows.  We observe the local behavior of the system by
% noting the evolution of the system $\System$ from a state $\vx \in C$
% reachable at time $t$ for a time length dependant on $k$. For
% \begin{itemize}
%     \item{$k=0$}: we observe all trajectory segments of length $\Delta$.
%     \item{$k=1$}: we observe trajectory segments of length $\Delta$,
%         which satisfy $\vx'\in C'$, where $\vx'$ is reachable from
%         $\vx$ in time $\Delta$.
%     \item{$k=2$}: we observe trajectory segments of length $2\Delta$,
%         and split them in two sets (a) and (b) on the basis of the
%         cells reached at time $(t+\Delta)$ and $(t+2\Delta)$.
%         \begin{enumerate}[label=\alph*]
%             \item $\x' \in C' \land \x''\in C''_1$
%             \item $\x' \in C' \land \x'' \in C''_2$
%         \end{enumerate}
%     \item{$k=n$}: we observe trajectory segments of length $n\Delta$,
%         and split them in to multiple sets on the basis of the cells
%         reached at time $(t+\Delta),\; (t+2\Delta),\;
%         \ldots,\;(t+n\Delta)$.
% \end{itemize}
% 
% $k$-relational modeling can be understood as a heuristic, which uses
% the underlying abstraction to differentiate behaviors of the system
% which `diverge' (or are revealed to be `distinct' at a future time).
% Such a heuristic can be useful in increasing the precision of the
% learnt affine maps.  We now formalize this notion for $k=0$, $1$, and
% $n$ and illustrate using examples.

% \mypara{$0$-Relational Model}
% 
% \begin{figure}[!htbp]
% \begin{center}
% \tikzstyle{line} = [thick]
% \tikzstyle{arw} = [->, thick,>=stealth,shorten <=2pt, shorten >=2pt]
% \begin{tikzpicture}
% \begin{scope}[scale=0.8]
% 
%     \draw [line] (-1.0,-1.0) rectangle (1.0,1.0);
%     %\draw [line] (3.0,1.0) rectangle (5.0,3.0);
%     %\draw [line] (3.0,-3.0) rectangle (5.0,-1.0);
%     \draw [line] (-0.7,0.5) .. controls +(2.0,2.0) and +(-2.0,-2.0) ..  (3.5,2.2);
%     \draw [line] (-0.5,0.0) .. controls +(2.0,2.0) and +(-2.0,-2.0) ..  (3.5,1.7);
%     \draw [line] (-0.0,-0.5) .. controls +(2.0,2.0) and +(-2.0,-1.0) ..  (3.5,-1.2);
% 
% \node at (0.0, 0.0) {$C$};
% \node at (-0.75,0.7) {\scriptsize{$\x_0$}};
% \node at (3.7,2.2) {\scriptsize{$\x_0'$}};
% \node at (-0.7,0.0) {\scriptsize{$\x_1$}};
% \node at (3.7,1.7) {\scriptsize{$\x_1'$}};
% \node at (-0.2,-0.5) {\scriptsize{$\x_2$}};
% \node at (3.7,-1.2) {\scriptsize{$\x_2'$}};
% \node at (2,1.5) {$\pi_1$};
% \node at (2,0.5) {$\pi_2$};
% \node at (1.7,-1) {$\pi_3$};
% \draw[arw] (-1.5,-2.0) -- (4.5,-2.0);
% \node at (0,-2.5) {$t=0$};
% \node at (4,-2.5) {$t=\Delta$};
% \draw [fill=black] (0.0,-2.0) circle (0.05);
% \draw [fill=black] (4.0,-2.0) circle (0.05);
% 
% \end{scope}
% \begin{scope}[xshift=0,yshift=2.1cm,scale=0.5]
%     \node at (2,0) {\footnotesize $\scr{T} \equiv g_1(\x):\vx \in C
%     \implies \vx' \in A_1\vx + \vb_1 + \delta_1$};
% \end{scope}
% 
% \end{tikzpicture}
% \end{center}
% \vspace*{-.3cm}
% \caption{All the trajectory segments will be used to construct the
% model; $\ds = \setof{\pi_1, \pi_2, \pi_3}$.}
% \label{fig:k0}
% \vspace*{-.3cm}
% \end{figure}
% 
% When $k=0$, the $0$-relational model is a PWA transition system which only
% defines the evolution of states $\x$, and does not specify the
% reachable cell. The guard predicates of the relations are defined over
% cells: $g_i(\x): \x \in C_i$ while $g'_i(\x'): True$.
% 
% We use regression to estimate the dynamics for the outgoing trajectory
% segments form a cell $C$. Hence, the data set $\ds$ for the regression
% includes all trajectory segments beginning from the same cell
% \[
%     \ds = \setof{(\vx,\vx') | \vx \in C}.
% \]
% This includes trajectory segments ending in different cells, as shown
% in ~\figref{k0}. For $N$ cells, this results in a $\rho$ with $N$ transition relations.
% \begin{equation*}
%     \scr{T} = \left\{
%         \begin{array}{ll}
%             g_1(\x):\vx \in C_1 \implies \vx' \in A_1\vx + \vb_1 + \delta_1\\
%             \ldots \\
%             g_n(\x):\vx \in C_n \implies \vx' \in A_n\vx + \vb_n + \delta_n\\
%         \end{array}
%     \right.
% \end{equation*}
% 
% Note that this can be quite imprecise when the cells are big,
% containing regions of state-space with complex dynamics. This is true
% for both non-linear systems and hybrid dynamical system, where a cell
% can contain two or more modes with differing continuous dynamics.
% 
% \mypara{$1$-Relational Model}
% 
% \begin{figure}[!htbp]
% \begin{center}
% \tikzstyle{line} = [thick]
% \tikzstyle{arw} = [->, thick,>=stealth,shorten <=2pt, shorten >=2pt]
% \begin{tikzpicture}
% \begin{scope}[scale=0.8]
% 
%     \draw [line] (-1.0,-1.0) rectangle (1.0,1.0);
%     \draw [line] (3.0,1.0) rectangle (5.0,3.0);
%     \draw [line] (3.0,-3.0) rectangle (5.0,-1.0);
%     \draw [line] (-0.7,0.5) .. controls +(2.0,2.0) and +(-2.0,-2.0) ..  (3.5,2.2);
%     \draw [line] (-0.5,0.0) .. controls +(2.0,2.0) and +(-2.0,-2.0) ..  (3.5,1.7);
%     \draw [line] (-0.0,-0.5) .. controls +(2.0,2.0) and +(-2.0,-1.0) ..  (3.5,-1.2);
% 
% \node at (0.0, 0.0) {$C$};
% \node at (4.0, 2.0) {$C'_1$};
% \node at (4.0, -2.0) {$C'_2$};
% \node at (-0.75,0.7) {\scriptsize{$\x_0$}};
% \node at (-0.7,0.0) {\scriptsize{$\x_1$}};
% \node at (-0.2,-0.5) {\scriptsize{$\x_2$}};
% \node at (3.3,2.5) {\scriptsize{$\x_0'$}};
% \node at (3.7,1.5) {\scriptsize{$\x_1'$}};
% \node at (3.7,-1.5) {\scriptsize{$\x_2'$}};
% \node at (2,1.5) {$\pi_1$};
% \node at (2,0.5) {$\pi_2$};
% \node at (1.7,-1) {$\pi_3$};
% \draw[arw] (-1.5,-3.5) -- (5.5,-3.5);
% \node at (0,-4.0) {$t=0$};
% \node at (4,-4.0) {$t=\Delta$};
% \draw [fill=black] (0.0,-3.5) circle (0.05);
% \draw [fill=black] (4.0,-3.5) circle (0.05);
% 
% \end{scope}
% \begin{scope}[xshift=1cm,yshift=3.1cm,scale=0.5]
%     \node at (2,0) {\begin{minipage}{\linewidth}
%     \footnotesize \begin{equation*}
%         \scr{T} = \left\{
%         \begin{array}{ll}
%             g_1(\x):\vx \in C \land g'_1(\x):\vx \in C'_1 \implies \vx' \in A_1\vx + \vb_1 + \delta_1\\
%             g_2(\x):\vx \in C \land g'_2(\x):\vx \in C'_2 \implies \vx' \in A_2\vx + \vb_2 + \delta_2
%         \end{array}
%     \right.
% \end{equation*}
% \end{minipage}};
% \end{scope}
% 
% \end{tikzpicture}
% \end{center}
% \vspace*{-.3cm}
% \caption{The data gets split into two sets $\ds_1 = \setof{\pi_1,
%     \pi_2}$ and $\ds_2 = \setof{\pi_3}$ and two relations:
%     $R_{(C,C'_1)}$ and $R_{(C,C'_2)}$, each with one affine map, are constructed.}
% %Cardinality($R_{(C,C')}$) = $2$.}
% \label{fig:k1}
% \vspace*{-.3cm}
% \end{figure}
% 
% 
% % \begin{scope}[xshift=0,yshift=4.0cm,scale=1]
% %     \node at (2,0) {\footnotesize $\begin{equation*}
% %     \scr{T} = \left\{
% %         \begin{array}{ll}
% %             g_1(\x):\vx \in C \land g'_1(\x):\vx \in C'_1 \implies A_1\vx + \vb_1 + \delta_1\\
% %             g_2(\x):\vx \in C \land g'_2(\x):\vx \in C'_2 \implies A_2\vx + \vb_2 + \delta_2
% %         \end{array}
% %     \right.
% % \end{equation*}$};
% % \end{scope}
% 
% To improve the preciseness of learnt dynamics, we include the
% reachability relation in the
% regression. For every relation $C\areach{t}C'$, the data set $\ds$ is
% comprised only of trajectory segments $\pi_t$ which start and end in
% the same set of cells.
% \[
%     \ds = \setof{(\vx,\vx') | \vx \in C \land \vx' \in C'}.
% \]
% For $N$ edges in $G$, $1$-relationalization results in a $\rho$ with
% $N$ transition relations.  \figref{k1} illustrates the $k=1$ refinement of the case shown in
% \figref{k0}. The data set $\ds$ is split into two data sets $\ds_1
% = \setof{\pi_1, \pi_2}$ and $\ds_2 = \setof{\pi_3}$, using which, two
% relations are constructed. The resulting $1$-relational model
% is at least as precise as the corresponding $0$-relational model.

\mypara{$k$-Relational Model}

\begin{figure}[!htbp]
\begin{center}
\tikzstyle{line} = [thick]
\tikzstyle{arw} = [->, thick,>=stealth,shorten <=2pt, shorten >=2pt]
\begin{tikzpicture}
\begin{scope}[xshift=0cm, scale=0.75]

    \draw [line] (-1.0,-1.0) rectangle (1.0,1.0);
    \draw [line] (3.0,1.0) rectangle (5.0,3.0);
    %\draw [line] (3.0,-3.0) rectangle (5.0,-1.0);
    \draw [line] (7.0,1.0) rectangle (9.0,3.0);
    \draw [line] (7.0,-3.0) rectangle (9.0,-1.0);
    \draw [line] (-0.7,0.5) .. controls +(2.0,2.0) and +(-2.0,-2.0) .. (3.5,2.2) .. controls +(2.0,2.0) and +(-2.0,-2.0) .. (8.5,1.8) ;
    \draw [line] (-0.5,0.0) .. controls +(2.0,2.0) and +(-2.0,-2.0) .. (3.5,1.7) .. controls +(2.0,2.0) and +(-2.0,-2.0) .. (8.5,-2.2) ;

%     \draw [line] (-0.7,0.5) .. controls +(2.0,2.0) and +(-2.0,-2.0) ..  (3.5,2.2);
%     \draw [line] (-0.5,0.0) .. controls +(2.0,2.0) and +(-2.0,-2.0) ..  (3.5,1.7);

    %\draw [line] (-0.0,-0.5) .. controls +(2.0,2.0) and +(-2.0,-1.0) ..  (3.5,-1.2);

\node at (0.0, 0.0) {$C$};
\node at (4.0, 2.0) {$C'_1$};
\node at (8.0, 2.0) {$C''_1$};
\node at (8.0, -2.0) {$C''_2$};
%\node at (4.0, -2.0) {$C'_2$};
\node at (-0.75,0.7) {\scriptsize{$\x_0$}};
\node at (-0.7,0.0) {\scriptsize{$\x_1$}};
%\node at (-0.2,-0.5) {\scriptsize{$\x_2$}};
\node at (3.3,2.5) {\scriptsize{$\x_0'$}};
\node at (3.7,1.5) {\scriptsize{$\x_1'$}};
%\node at (3.7,-1.5) {\scriptsize{$\x_2'$}};
\node at (8.7,1.6) {\scriptsize{$\x_0''$}};
\node at (8.5,-2.7) {\scriptsize{$\x_1''$}};
\node at (2,1.5) {$\pi_1$};
\node at (2,0.5) {$\pi_2$};
\draw[arw] (-1.5,-3.5) -- (9.5,-3.5);
\node at (0,-4.0) {$t=0$};
\node at (4,-4.0) {$t=\Delta$};
\node at (8,-4.0) {$t=2\Delta$};
\draw [fill=black] (0.0,-3.5) circle (0.05);
\draw [fill=black] (4.0,-3.5) circle (0.05);
\draw [fill=black] (8.0,-3.5) circle (0.05);
\end{scope}

\begin{scope}[xshift=1cm,yshift=3.0cm,scale=0.5]
    \node at (4,0) {\begin{minipage}{\linewidth}
        \footnotesize \begin{equation*}
    \scr{T} = \left\{
        \begin{array}{ll}
            g_1(\x):\vx \in C \land g'_1(\x):\vx \in C'_1 \implies A_1\vx + \vb_1 + \delta_1\\
            g_1(\x):\vx \in C \land g'_1(\x):\vx \in C'_1 \implies A_2\vx + \vb_2 + \delta_2
        \end{array}
    \right.
\end{equation*}
\end{minipage}};
\end{scope}

\end{tikzpicture}
\end{center}
\vspace*{-.3cm}
\caption{The $k=2$ refinement further splits the data set $\ds_1$
    into two sets $\ds_{11} = \setof{\pi_1}$ and $\ds_{12} =
    \setof{\pi_2}$ and $R_{(C,C'_1)}$ now has two affine maps,
    non-deterministically defining the system behavior.}
\label{fig:k2}
\vspace*{-.3cm}
\end{figure}

\todo{Add ref to the refinement construction algo from previous
section, and show where it varies. Explain how the look-ahead cells
are used in the construction.}

Basically, the transition updates in a $(k,\epsilon)$-RTS are
constructed by using $k$ step trajectories which have their end points
in the same cells.
\[
    \ds = \setof{(\vx,\vx') | \vx \in C \land \vx' \in C' \land \vx''
    \in C'' \land \ldots \land \vx^k \in C^k }.
\]



%%%%%%%%%%%%%%%%%%%%%%%%%%%%%%
\section{Model Checking}

As a final step, the enriched graph is model checked for the given
safety property. If a counter-example is found, it is simulated using
the $\simulate$ function and if reproducible, a violation is
concluded. In all other cases the falsification results in an
`unknown' result.

% % include an algorithm
% We now describe the entire procedure in three steps.
% \begin{enumerate}
%     \item Given a black box system $\System$ specified by
%         $\simulate^\System$, we
%         use scatter-and-simulate with ($\Delta$-length) trajectory
%         segments to sample a finite reachability graph
%         $\scr{H}(\Delta)$.
%     \item Using regression, we enrich the graph relations and annotate
%         the edges with the estimated affine maps.
%     \item Interpreting the directed reachability graph as a transition
%         system, we use a bounded model checker to find fixed length
%         violations to safety properties.
% \end{enumerate}

\section{An Example: Van der Pol Oscillator}

We now describe the pesented technique using the Van der Pol
oscillator benchmark from~\cite{zutshi2014multiple}. Simulations
generated using uniformly random samples are shown in
\figref{vdp-cont}. We want to check the property $P_3$, indicated by
the red box, given the initial set indicated by the green box. The
abstraction is defined by $\quant_{0.2}(\x)$, which results in an
evenly gridded state-space, where each cell is of size $0.2 \times
0.2$ units.  Scatter-and-simulate is then used to construct the
abstract graph $G$, using $2$ samples per cell and the time step
$\Delta = 0.1s$. The complete process follows.

\begin{enumerate}
    \item{\emph{Abstract}}: Assume an implicit abstraction using the quantization
        function $\quant_{\epsilon=0.2}(\x)$. The corresponding reachability
    graph is $\scr{H}(\Delta=0.1)$.
\item{\emph{Discover}}: Using scatter-and-simulate, enumerate a finite number
    of cells (vertices) and the relations (edges) of the graph
    $\scr{H}(0.1)$. Associate the set of generated trajectory
    segments with their respective originating cells using a map $\ds:C
        \mapsto \setof{(\vx,\vx') |\vx \in C}$. The discovered
    abstraction is show in \figref{vdp-abs}. The red cells
    are unsafe and green cells are initial cells.
\item{\emph{Relationalize}}: For each cell $C$, perform regression analysis on
    the respective set of trajectory segments $D(C)$, and compute a
    set of affine relations $R_{C,C'}(\x,\x')$ between $\x \in C$ and $\x'
    \in C'$ s.t., $\text{edge} (C,C') \in \scr{H}(0.1)$. Annotate
    each edge in the graph $\scr{H}(0.1)$ with the respective
    relation.

    \figref{vdp-map} shows the cells and the trajectory segments which
    are part of the data sets constructed using the $1$-relational
    modeling. Against each cell, its unique identifier (integer
    co-ordinate) is mentioned.  Finally, \figref{vdp-graph} and
    \tabref{vdp-pwa} show the enriched graph $G^R$ with its transition
    relations. Note that self loops result when an observed trajectory
    segment has its $start$ and $end$ states in the same cell.

\item{\emph{Model Check}}: The graph $\scr{H}(0.1)$ can now be viewed as a
    transition system $\tupleof{C,\x,\scrT,C_0,\HybridStates_0}$,
    where $C \in vertices(\scr{H}(0.1))$ and $\x \in
    \HybridStates$. A transition $\tau \in \scrT$ is of the form
    $\tupleof{C, C',\rho_{\tau}}$, where $\rho_{\tau}(\x,\x')
    \subseteq \setof{R_{(C,C')}(\x,\x') |( C,C') \in
    edge(\scr{H}(0.1))}$.
\item{\emph{Check Counter-example}}: The infinite state (but finite location) transition system can
    be model checked to find a concrete counter-example, which if
    exists, can indicate the existence of a similar trace in the
    original black-box system $\System$. The latter check is carried
    out as before, using the numerical simulation function
    $\simulate$. For the given example, the model checker is unable to
    find a counter-example.
\end{enumerate}

\inclfig[width=0.95\linewidth]{vdp_cont_traces.pdf}{Van der
Pol: continuous trajectories. Red and green boxes indicate unsafe and
initial sets.}
\label{fig:vdp-cont}

\inclfig[width=0.95\linewidth]{vdp_abs.pdf}{The discovered
abstraction $\scr{H}(0.1)$. Red cells are unsafe cells and green cells are
initial cells.}
\label{fig:vdp-abs}

\inclfig[width=0.90\linewidth]{vdp_disc_map.pdf}{Cells and
trajectory segments used by $1$-relational modeling.}
\label{fig:vdp-map}

\inclfig[width=0.50\linewidth]{vdp_graph.pdf}{Enriched
graph $G^R$. The affine maps $f$ for the transition relations are show
in \tabref{vdp-pwa}.}
\label{fig:vdp-graph}

\begin{table}[h]
\centering
\caption{PWA model computed using OLS. The affine model for each edge $(C,C')$ in the graph \figref{vdp-graph} is given by $x'\in Ax + b + \delta$, where $\delta$ is a vector of intervals.}
\label{tab:vdp-pwa}
\begin{tabular}{@{}ccccc@{}}
\toprule
$C$ & $C'$ & $A$ & $b$ & $\delta$\\
\midrule
(0, -1)   & (0, -1)   & $\begin{bmatrix}0.99& 0.12 \\-0.12&1.63\end{bmatrix}$&$\begin{bmatrix}0\\0\end{bmatrix}  $&$\begin{bmatrix}[0, 0]\\ [0, 0]\end{bmatrix}$\\
(0, -1)   & (0, -2)   & $\begin{bmatrix}0.99& 0.12 \\-0.10&1.63\end{bmatrix}$&$\begin{bmatrix}0\\0\end{bmatrix}  $&$\begin{bmatrix}[0, 0]\\ [0, 0]\end{bmatrix}$\\
\midrule
(0, -2)   & (0, -2)   & $\begin{bmatrix}0.99& 0.12 \\-0.09&1.63\end{bmatrix}$&$\begin{bmatrix}0\\0\end{bmatrix}  $&$\begin{bmatrix}[0, 0]\\ [0, 0]\end{bmatrix}$\\
(0, -2)   & (0, -4)   & $\begin{bmatrix}0.99& 0.12 \\-0.07&1.62\end{bmatrix}$&$\begin{bmatrix}0\\0\end{bmatrix}  $&$\begin{bmatrix}[0, 0]\\ [0, 0]\end{bmatrix}$\\
\midrule
(0, -9)   & (-1, -15) & $\begin{bmatrix}0.99& 0.12 \\-0.09&1.61\end{bmatrix}$&$\begin{bmatrix}0\\-0.04\end{bmatrix} $&$\begin{bmatrix}[0, 0]\\ [0, 0.01]\end{bmatrix}$\\
\midrule
(-1, -15) & (-3, -24) & $\begin{bmatrix}0.95& 0.12 \\-1.29&1.47\end{bmatrix}$&$\begin{bmatrix}-0.01\\-0.41\end{bmatrix}$&$\begin{bmatrix}[0, 0]\\ [0, 0.01]\end{bmatrix}$\\
\midrule
(-3, -24) & (-4, -29) & $\begin{bmatrix}1.11&-0.17 \\-1.71&0.66\end{bmatrix}$&$\begin{bmatrix}-1.07\\-3.31\end{bmatrix}$&$\begin{bmatrix}[-0.04, 0.04]\\ [-0.07, 0.09]\end{bmatrix}$\\
\midrule
(-4, -29) & (-4, -29) & $\begin{bmatrix}0.99& 0.01 \\-0.41&1.02\end{bmatrix}$&$\begin{bmatrix}0\\-0.28\end{bmatrix} $&$\begin{bmatrix}[0, 0]\\ [0, 0]\end{bmatrix}$\\
\midrule
(-1, -10) & (-2, -16) & $\begin{bmatrix}0.97& 0.12 \\-0.71&1.56\end{bmatrix}$&$\begin{bmatrix}0\\-0.11\end{bmatrix} $&$\begin{bmatrix}[0, 0]\\ [0, 0.01]\end{bmatrix}$\\
(-1, -10) & (-2, -15) & $\begin{bmatrix}0.97& 0.12 \\-0.70&1.58\end{bmatrix}$&$\begin{bmatrix}0\\-0.08\end{bmatrix} $&$\begin{bmatrix}[0, 0]\\ [0, 0.01]\end{bmatrix}$\\
\midrule
(-2, -16) & (-3, -23) & $\begin{bmatrix}0.92& 0.12 \\-1.84&1.39\end{bmatrix}$&$\begin{bmatrix}-0.02\\-0.73\end{bmatrix}$&$\begin{bmatrix}[0, 0]\\ [0, 0]\end{bmatrix}$\\
\midrule
(-3, -23) & (-5, -29) & $\begin{bmatrix}3.14&-0.81 \\-0.96&0.23\end{bmatrix}$&$\begin{bmatrix}-3.13\\-4.99\end{bmatrix}$&$\begin{bmatrix}[-0.04, 0.03]\\ [-0.03, 0.04]\end{bmatrix}$\\
\midrule
(-2, -15) & (-3, -22) & $\begin{bmatrix}0.93& 0.12 \\-1.75&1.40\end{bmatrix}$&$\begin{bmatrix}-0.02\\-0.68\end{bmatrix}$&$\begin{bmatrix}[0, 0]\\ [0, 0]\end{bmatrix}$\\
\midrule
(-3, -22) & (-5, -29) & $\begin{bmatrix}3.08&-0.72 \\-1.69&0.40\end{bmatrix}$&$\begin{bmatrix}-2.77\\-4.57\end{bmatrix}$&$\begin{bmatrix}[-0.02, 0.02]\\ [-0.01, 0.02]\end{bmatrix}$\\
\bottomrule
\end{tabular}
\end{table}



% (0, -1)   & (0, -1)   &  \begin{bmatrix}0.9944498 & 0.12911959 \\-0.12013468&1.63250531\end{bmatrix} &\begin{bmatrix}-0.00003683\\-0.00085411\end{bmatrix}&\begin{bmatrix}[-0.0, 0.0]\\ [-0.0, 0.0]\end{bmatrix}\\
% (0, -1)   & (0, -2)   &  \begin{bmatrix}0.99523928& 0.1290447  \\-0.10236736&1.6314103 \end{bmatrix} &\begin{bmatrix}-0.00013019\\-0.00289192\end{bmatrix}&\begin{bmatrix}[-0.0, 0.0]\\ [-0.0, 0.0]\end{bmatrix}\\
% \midrule
% (0, -2)   & (0, -2)   &  \begin{bmatrix}0.99565348& 0.12916002 \\-0.09352462&1.6341424 \end{bmatrix} &\begin{bmatrix}-0.00015574\\-0.00339664\end{bmatrix}&\begin{bmatrix}[-0.0, 0.0]\\ [-0.0, 0.0]\end{bmatrix}\\
% (0, -2)   & (0, -4)   &  \begin{bmatrix}0.99668599& 0.12892455 \\-0.0725035 &1.62937613\end{bmatrix} &\begin{bmatrix}-0.00037391\\-0.00796189\end{bmatrix}&\begin{bmatrix}[-0.0, 0.0]\\ [-0.0, 0.0]\end{bmatrix}\\
% \midrule
% (0, -9)   & (-1, -15) &  \begin{bmatrix}0.99985362& 0.12858788 \\-0.09447884&1.61476417\end{bmatrix} &\begin{bmatrix}-0.0018997 \\-0.04200363\end{bmatrix}&\begin{bmatrix}[-0.0, 0.0]\\ [-0.0, 0.01]\end{bmatrix}\\
% \midrule
% (-1, -15) & (-3, -24) &  \begin{bmatrix}0.95473217& 0.12473489 \\-1.29198505&1.47155614\end{bmatrix} &\begin{bmatrix}-0.01248412\\-0.41100129\end{bmatrix}&\begin{bmatrix}[-0.0, 0.0]\\ [-0.0, 0.01]\end{bmatrix}\\
% \midrule
% (-3, -24) & (-4, -29) &  \begin{bmatrix}1.11789749&-0.17261206 \\-1.71017779&0.66006597\end{bmatrix} &\begin{bmatrix}-1.0745344 \\-3.31246863\end{bmatrix}&\begin{bmatrix}[-0.04, 0.04]\\ [-0.07, 0.09]\end{bmatrix}\\
% \midrule
% (-4, -29) & (-4, -29) &  \begin{bmatrix}0.99796588& 0.01012817 \\-0.41529747&1.02432211\end{bmatrix} &\begin{bmatrix}-0.00135199\\-0.28016983\end{bmatrix}&\begin{bmatrix}[-0.0, 0.0]\\ [-0.0, 0.0]\end{bmatrix}\\
% \midrule
% (-1, -10) & (-2, -16) &  \begin{bmatrix}0.97381953& 0.12768367 \\-0.71242741&1.56764585\end{bmatrix} &\begin{bmatrix}-0.00320265\\-0.11964102\end{bmatrix}&\begin{bmatrix}[-0.0, 0.0]\\ [-0.0, 0.01]\end{bmatrix}\\
% (-1, -10) & (-2, -15) &  \begin{bmatrix}0.97373058& 0.12859163 \\-0.70173855&1.58540872\end{bmatrix} &\begin{bmatrix}-0.0015415 \\-0.08616376\end{bmatrix}&\begin{bmatrix}[-0.0, 0.0]\\ [-0.0, 0.01]\end{bmatrix}\\
% \midrule
% (-2, -16) & (-3, -23) &  \begin{bmatrix}0.92850123& 0.12242812 \\-1.84720528&1.39159826\end{bmatrix} &\begin{bmatrix}-0.02334236\\-0.73348865\end{bmatrix}&\begin{bmatrix}[-0.0, 0.0]\\ [-0.0, 0.0]\end{bmatrix}\\
% \midrule
% (-3, -23) & (-5, -29) &  \begin{bmatrix}3.14947585&-0.81150646 \\-0.96159406&0.23543254\end{bmatrix} &\begin{bmatrix}-3.13860326\\-4.99696623\end{bmatrix}&\begin{bmatrix}[-0.04, 0.03]\\ [-0.03, 0.04]\end{bmatrix}\\
% \midrule
% (-2, -15) & (-3, -22) &  \begin{bmatrix}0.93156623& 0.12260264 \\-1.75117219&1.40104348\end{bmatrix} &\begin{bmatrix}-0.02214125\\-0.68423569\end{bmatrix}&\begin{bmatrix}[-0.0, 0.0]\\ [-0.0, 0.0]\end{bmatrix}\\
% \midrule
% (-3, -22) & (-5, -29) &  \begin{bmatrix}3.08806462&-0.72332776 \\-1.69271079&0.40308293\end{bmatrix} &\begin{bmatrix}-2.77908467\\-4.57411339\end{bmatrix}&\begin{bmatrix}[-0.02, 0.02]\\ [-0.01, 0.02]\end{bmatrix}\\






\subsection{Search Parameters}

The search parameters for S3CAM-R include the parameters of S3CAM:
$N$, $\epsilon$ and $\Delta$. Additionally, they also include the
maximum error budget $\delta_{max}$ for OLS and the maximum length of
segmented trajectory for building $k$-relational models.

We have already discussed the effects of $N$, $\epsilon$ and $\Delta$
on S3CAM's performance. However, they also have an effect on
relational modeling. A finer grid with small cells produces more
accurate models than a coarser grid with large cells. Similarly,
small time length trajectory segments result in more accurate
modeling.
%An increase in the number of samples can increases the
%accuracy of regression.

\paragraph{Maximum Model Error ($\boldsymbol{\delta}_{\bold{max}}$) .}
Given a $\delta_{max}$, we keep increasing $k$ during the
$k$-relational modeling process till $\delta_i \le \delta_{max}$ is
satisfied. A $k_{max}$ is introduced to bound the longest
segmented trajectory which can be considered.

% \paragraph{Bounded k relations ($\boldsymbol{k}_{\bold{max}}$.) }

\subsection{Reasons for Failure}
The approach can fail to find a counter-example, even when it exists, in
one of three ways.

\begin{itemize}

    \item \emph{S3CAM Fails} No abstract counter-example is found by
        S3CAM. We can remedy it by increasing search budgets and/or
        restarting.

    \item \emph{BMC Fails} An abstract counter-example is found, but
        the BMC fails to find a concrete counter-example in the PWA
        relational model. The failure can be attributed to either (a)
        a spurious abstract counter-examples or (b) a poorly estimated
        model. The former can be addressed by restarting but the
        latter requires that the maximum model error $\delta_{max}$ be
        decreased.

    \item \emph{Inaccurate PWA Modeling} An abstract counter-example
        is found, and it is successfully concretized in the PWA
        relational model by the BMC. However, it is not reproducible
        in the black box system. This happens when the PWA relational
        model is not precise enough.

\end{itemize}

%%%%%%%%%%%%%%%%%%%%%%%%%%%%%%


%%%%%%%%%%%%%%%%%%%%%%%%%%%%%%
% \section{Bounded Model Checking for Black Box Systems}
% \label{sec:bmc}
%%%%%%%%%%%%%%%%%%%%%%%%%%%%%%


%%%%%%%%%%%%%%%%%%%%%%%%%%%%%%
% \section{Implementation}
% \label{sec:impl}
%%%%%%%%%%%%%%%%%%%%%%%%%%%%%%

%%%%%%%%%%%%%%%%%%%%%%%%%%%%%%
\section{Experimental Results}
\label{sec:res}

The implementation was prototyped as S3CAM-R, an extension to our
previously mentioned tool S3CAM (\chapref{case}). OLS regression
routines were used from Scikit-learn~\cite{pedregosa2011scikit}, a
Python module for machine learning. SAL~\cite{SAL-SRI}
with Yices2~\cite{dutertre2014yices} was the model checker and the SMT
solver.

We used S3CAM to discover the abstraction graph $G$, which was then
trimmed of the nodes which did not contribute to the error search
(nodes from which error nodes were not reachable). In our experiments we
used $1$-relational modeling to create the transition system from $G$.
Using SAL, we checked for the given safety property. If a concrete
trace was obtained from the BMC, it was further checked for validity
by simulating using $\simulate$ to see if it was indeed an error
trace.  If not, $100$ samples from its associated abstract state was
sampled to check the presence of a counter-example in it
neighborhood. This can be further extended by obtaining different
discrete sequences or discrete trace sequences from the model checker.

We tabulate our preliminary evaluation in \tabref{res-rel}. We used
few of the benchmarks described in \chapref{case}, including the Van
der Pol oscillator, Brusselator, Lorenz attractor and the Navigation
benchmark. As before, we ran S3CAM-R $10$ times with different seeds
and averaged the results. We tabulate both the total time taken and
the time taken by SAL to compute the counter-example and compare
against a newer implementation of S3CAM. Note that different
abstraction parameters are used for S3CAM and S3CAM-R, due to the
differences in which they operate.

\begin{table*}[!htbp]
\centering
\caption{Avg. timings for benchmarks. The \textbf{BMC} column lists
time taken by the BMC engine. The total time in seconds (rounded off
to an integer) is noted under \textbf{S3CAM-R} and \textbf{S3CAM}.
$TO$ signifies time $>5hr$, after which the search was killed.}
\label{tab:res-rel}
\begin{tabular}{@{}llll@{}}
\toprule
Benchmark & BMC & S3CAM-R & S3CAM\\
\midrule
Van der Pol ($\prop_3$)   &$<$1 & $130$ & $24$\\
Brusselator               &$<$1  & $4$    & $2$\\
Lorenz                    &$<$1 & $122$  & $35$\\
%B.Ball                    &  &   & \\
Nav ($\prop_P$)           &$4480$ &$9423$   &$603$ \\
Nav ($\prop_Q$)           &$970$  &$8105$   &$546$ \\
Nav ($\prop_R$)           &-  &$TO$   & $2003$\\
Nav ($\prop_S$)           &-  &$TO$   & $2100$\\
Bouncing Ball             &-  &$TO$   & $450$\\

\bottomrule
\end{tabular}
\end{table*}

% 1m8s, 2m56.184s, 1m31.059s[f], 1m32.665s,  1m45.312s, 1m22.397s, 2m1.451s[f], 1m12.118s, 1m26.596s
% 0.12, 0.06,         0.33       0.13,       0.22,     FAIL,       ,X           0.08        0.15

The results show promise, but clearly S3CAM performs better. Also,
S3CAM-R timed out on some benchmarks like Nav $\prop_R$, Nav $\prop_S$
and the bouncing ball. However, we need to explore more benchmarks to
be conclusive. S3CAM-R's performance can be explained by the
difficulty in finding a good abstraction ($\quant_\epsilon$), which
needs to be fine enough, to obtain good prediction models but coarse
enough, to be manageable by the current SMT solvers. Specifically, to
obtain a good quality counter-example from the model checker (and
avoid false positives), the transition system should be created from
models with high accuracy, for which a finer abstraction is required.
However, the exploration of a finer abstraction is exponentially more
complex and results in a much bigger transition system, which the
current state of the art bounded model checkers (which use SMT
solvers) can not handle under reasonable resources.

\input{results.tex}
%%%%%%%%%%%%%%%%%%%%%%%%%%%%%%

%%%%%%%%%%%%%%%%%%%%%%%%%%%%%%
\section{Conclusions}
\label{sec:concl}

We have presented a methodology to find falsifications in black box
dynamical systems. Combining the ideas from abstraction based search
\cite{zutshi2014multiple}, with relational
abstractions~\cite{sankaranarayanan2011relational}, we proposed
$k$-relational modeling to compute richer abstractions. Simple linear
regression can be used to estimate the local dynamics of the
trajectory segments and compute PWA relational models which can be
interpreted as a transition system.  These can then be checked for
safety violations using an off-the-shelf bounded model checker.
Finally, we implemented the ideas as a tool PWA-Rel and demonstrated
its performance on a few examples.

%\todo{memory required: pre-simulated data}
%\todo{extension to LTL/MTL props}

% \subsection{Improvements}

% The approach seems promising as it avoids an expensive refinement
% step. But, before it can be applied to more complex systems, we need
% to overcome its shortcomings.  As a future extension, we are
% investigating ideas to improve the performance of the underlying three
% main sub-processes, namely, abstraction, modeling, and BMC. We detail
% these below.

% \begin{enumerate}

% \item \emph{Abstraction} The abstract domain we have used, has been
%     confined to the interval or rectangular domain. Such a domain is
%     very coarse when compared to more general polyhedral domains, using
%     which, we can gain better precision. Due to polyhedral domains
%     being computational expensive, their integration in S3CAM needs to
%     be explored thoroughly.

% \item \emph{Modeling} We have used $k$-relational modeling to increase
%     the precision of the enriched abstractions, however, there are
%     several other refinement strategies that remain to be explored.
%     This includes refining the abstract states by using guard
%     predicates (similar to predicate abstraction), using non-linear
%     templates in parametric regression to compute more precise
%     quadratic relations, and using an adaptive time step instead of a
%     fixed one to address the non-linearities due to a long time
%     horizon.

% \item \emph{Model Checking} As of now, we use SMT solvers to search
%     for concrete counter-examples. Although they are very efficient,
%     owing to extensive engineering effort, reachability of transition
%     systems resulting from dynamical systems remains a difficult
%     problem. As the time horizon of the safety property increases, the
%     possible combinations of discrete transitions increase
%     exponentially. Hence, to find a counterexample which is a sequence
%     of discrete transitions over a `long' time horizon is not
%     tractable for most, but the simplest of dynamical systems.
%     Instead, we can use linear programming solvers, by enumerating
%     each path in the graph $G^R$. Note that the constraints are all
%     linear (conjunctions) along an abstract path. However, in the
%     worst case, the number of paths in a graph can be of the order
%     $n!$ where $n$ is the number of vertices of a graph. Hence, this
%     will not be feasible, unless, we can prioritize paths by using a
%     triage process similar to one used by S3CAM and using a budget on
%     the maximum number of paths.

%     Another approach to address the issue would be to use an adaptive
%     time discretization technique, where relations over both shorter
%     and longer time steps are computed, and the SMT solver can
%     `select' the time step precise enough to find a counter-example.

% \item \emph{SMT Solvers} One major impediment to our approach is the
%     fact that SMT solvers use the theory of reals with \emph{exact
%     precision}. This is important for verification approaches, but can
%     be relaxed for the problem of falsification. An SMT solver which
%     uses approximate reasoning but returns robust counter-examples
%     will be as useful, and perhaps more efficient.

% \end{enumerate}

% \subsection{Data Driven Analysis}

% Finally, we would like to mention that our approach is `simulation'
% driven. It can be easily modified to be `data driven', by working with
% a fixed set of data. Given a data set, we then need to automatically
% find a good abstraction and a high fidelity PWA transition system
% using which, we can summarize the black box hybrid dynamical system.
% Apart from model checking the transition system, one can extend it to
% the analysis of SDCS. More specifically, by combining the transition
% system model of a plant with the control software, a model checker
% like CBMC \cite{kroening2014cbmc} can be used to do a closed loop
% symbolic analysis of the SDCS. This can be an alternative to S3CAM-X.

%which will be
%very useful to for analyzing data for potential property violations.

%%%%%%%%%%%%%%%%%%%%%%%%%%%%%%


\bibliographystyle{ACM-Reference-Format}
\bibliography{refs}  % sigproc.bib is the name of the Bibliography in this case

% \newpage
% \appendix
% \obeyspaces
\obeylines
======================================
EMSOFT 2017 Reviews for Submission 159
======================================

Title: Piece-Wise Relational Models for Falsifying Safety Properties of Hybrid Systems

=====================================
                            REVIEWER 1
=====================================


----------------------------------
Reviewer's Scores
---------------------------------

               Relevance to EMSOFT (1-4): 3
          Presentation/Readability (1-4): 2
                       Originality (1-4): 4
               Technical Soundness (1-4): 2
Account of prior work and references (1-3): 2


------------------------------------
Comments
-----------------------------------


1. The title states that the paper addresses "hybrid systems" and the system
   model includes discrete states, but thereafter the paper only treats
   continuous systems. Apparently this is reasonable because of an
   assumption that the model is a "switched-mode dynamical system".

   It would be nice to define exactly what you mean by "switched-mode
   dynamical system". Does it mean that only the controller model makes
   discrete transitions and that the environment is a continuous model with
   no discrete events or discontinuities?

   ============
   The systems under test are assumed to be switched-mode to ease the
   presentation, but it is not a limitation of the presented
   technique. Black box systems which have hybrid dynamics due to
   multiple modes, are dealt in the same fashion.
   [TODO: $K$-refinement to differentiate the dynamics of different modes.
   We have added a benchmark to showcase this.
   [TODO: which one?]
   [TODO: clarify the absence of any controllers]
   [TODO: remove or clarify switched-mode systems]
   [TODO: add a hybrid benchmark]
   ============

   I imagine that the idea is to combine the generated discrete abstraction
   with a discrete controller model to verify both together. It would be
   good to state this clearly and to treat the resulting issues throughout
   the paper. I note that the experimental results do not treat this issue.
   Otherwise, why not change the title and focus the text on what you
   actually did?

   ============
   We will explore this in the future, but the current presentation
   only works with the plant, which is itself a hybrid dynamical
   system.
   [TODO: Add this in the conclusion/future work]
   ============


   (Similar remarks hold at a smaller scale. The discussion often describes
    what "can be" done, e.g., in sections 3.2 and 4.1. I would rather read
    about what "has been" done and what the result was!)

   ============
   [TODO]
   ============

2. Why do you need a bounded model checker to check safety invariants when
   you already enumerate and characterize the reachable states? That is, why
   not just test the invariant on the convex hulls as they are generated?

   My guess is that you later want to restrict the state-space by composing
   your abstraction with a controller model. Is that right? This probably
   seems obvious to you, but please put yourself in the place of someone who
   is new to the work and not an expert in the domain! In any case, this
   idea is not clearly stated, not addressed by the text, and not considered
   in the experimental evaluation.

   ============
   [TODO: I don't understand this comment, but it is related to
   point (iii) of the summary]
   Although we enumerate the abstract space, we require the model
   checker to find a concrete counter-example.
   ============

   Is this why the PWA transition system characterizes states by a valuation
   function v and not simply as tuples? If yes, how will this work? I do not
   understand why you have a valuation function when the system only has one
   variable (x).
   ============
   [TODO]
   ============
======================================
                            REVIEWER 2
======================================


--------------------------------------
Reviewer's Scores
--------------------------------------

               Relevance to EMSOFT (1-4): 4
          Presentation/Readability (1-4): 3
                       Originality (1-4): 4
               Technical Soundness (1-4): 3
Account of prior work and references (1-3): 2


--------------------------------------
Comments
--------------------------------------

- In section 2.1. the authors state that when N>n “a single affine function
cannot be found”. Should the problem be stated in finding the best choice of
a and b or should the problem considering determining the optimal number of
piece wise affine approximations? Can the magnitude N-n give a hint of what is
a good number of PWA / affine approximations? I feel that this could lead to a
nice results which the authors can easily cast in optimization (2).

>> In general, finding the `optimal' number of PWA models can be
>> framed as a multiobjective optimization problem, where a small
>> number of clusters and an accurate affine map is desired. This is a
>> hard problem for arbitrary non-linear systems.
>> Hence, in this paper, we presented a scalable falsification approach
>> with a sub-optimal PWA identification approach.


Can the authors comment or speculate on how the complexity of the
dynamics might influence the required number of affine maps? How does
the non-smoothness and non-differentiability influence the computation
of R(C,C’)? Is there any relationship between the number of
nondifferentiable points and the number of required affine maps?


- Another critical parameter in the framework seems to be the time steps Delta
used in section 4. Can the authors perform a sensitivity analysis to gauge its
impact on overall quality of the results?

>> It is a good idea, and it can have a big affect on the quality of
the results. However, selecting a `good' Delta is not straightforward.

- Since the main goal of this technique is to find violations of safety
properties in the embedded control system, it would prove useful for the reader
if the authors could comment which of the evaluations are considering this
specific goal. Again, this is not a criticism, but I feel that the
investigation in section 6 could be described in a few more sentences to
connect with embedded control.


======================================
                            REVIEWER 3
======================================


--------------------------------------
Reviewer's Scores
--------------------------------------

               Relevance to EMSOFT (1-4): 3
          Presentation/Readability (1-4): 3
                       Originality (1-4): 2
               Technical Soundness (1-4): 3
Account of prior work and references (1-3): 2


--------------------------------------
Comments
--------------------------------------

A comparison with S3CAM is interesting, but is less informative than a
comparison with simulation-based validation tools.

I also recommend using relations more complex than affine, such as
polynomial, for which Z3 can be used (nonlinear relations may reduce
the number of hyper-rectangles).


======================================
                            REVIEWER 4
======================================


--------------------------------------
Reviewer's Scores
--------------------------------------

               Relevance to EMSOFT (1-4): 3
          Presentation/Readability (1-4): 2
                       Originality (1-4): 3
               Technical Soundness (1-4): 2
Account of prior work and references (1-3): 2


--------------------------------------
Comments
--------------------------------------

Although the author claims that the proposed technique can effectively find a
counter-example for a black-box model, however, the experiment result does not
reflect that statement. Most of the benchmarks are nonlinear continuous systems
that are not black-box. I strongly suggest the author evaluate the proposed
approach on some black-box systems such as industrial Simulink models.

Using OLS does not always result in finding the best fit. For example, if some
data points are excessively small or large compared to the rest of data, they
will have disproportionately large effects on the resulting constant. In this
case, using other techniques like ridge regression or lasso regression may
yield a better result.

>> As the choice of loss function and type of regression analysis
>> depends on the problem at hand, we show our methodology using the
>> simplest one. In practice, sophesticated statistical tests can be used
>> to find the a good analysis.


======================================
                            REVIEWER 5
======================================


--------------------------------------
Reviewer's Scores
--------------------------------------

               Relevance to EMSOFT (1-4): 3
          Presentation/Readability (1-4): 3
                       Originality (1-4): 3
               Technical Soundness (1-4): 3
Account of prior work and references (1-3): 3


--------------------------------------
Comments
--------------------------------------

I suggest leading with the Van Der Pol oscillator to help ignorant
readers such as myself get a handle on what it is you're actually able
to do, then launch into the formalism.

>> We have made the Van Der Pol oscillator as a running example to improve
>> the presentation.

\end{document}

