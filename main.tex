%%%%%%%%%%%%%%%%%%%%%%%%%%%%%%%%%%%%%%%%%%%%%%%%%%%%%%%%%%%%%%%%%%%%%%%%%%%%%%%%
%2345678901234567890123456789012345678901234567890123456789012345678901234567890
%        1         2         3         4         5         6         7         8

\documentclass[letterpaper, 10 pt, conference]{ieeeconf}  % Comment this line out if you need a4paper

%\documentclass[a4paper, 10pt, conference]{ieeeconf}      % Use this line for a4 paper

\IEEEoverridecommandlockouts                              % This command is only needed if 
                                                          % you want to use the \thanks command

\overrideIEEEmargins                                      % Needed to meet printer requirements.

% See the \addtolength command later in the file to balance the column lengths
% on the last page of the document

% The following packages can be found on http:\\www.ctan.org
%\usepackage{graphics} % for pdf, bitmapped graphics files
%\usepackage{epsfig} % for postscript graphics files
%\usepackage{mathptmx} % assumes new font selection scheme installed
%\usepackage{times} % assumes new font selection scheme installed
%\usepackage{amsmath} % assumes amsmath package installed
%\usepackage{amssymb}  % assumes amsmath package installed


\usepackage{tikz}
\usepackage{framed}
\usepackage{stmaryrd}
\usepackage{amsmath}
\usepackage{ragged2e}
\usepackage{varwidth}

\usetikzlibrary{arrows,backgrounds,positioning,fit,automata,shapes,snakes,patterns}
\usetikzlibrary{arrows,backgrounds,decorations,decorations.pathmorphing,positioning,fit,automata,shapes,snakes,patterns}
\usetikzlibrary{shapes.geometric, arrows, positioning, calc, matrix}
\tikzset{block/.style = {draw, fill=blue!20, rectangle,
                         minimum height=3em, minimum width=6em},
        sum/.style = {draw, fill=blue!20, circle, node distance=1cm},
        input/.style = {coordinate},
        output/.style = {coordinate},
        pinstyle/.style = {pin edge={to-,thin,black}}
}

\usepackage{balance}
\usepackage{cite}

%\renewcommand{\baselinestretch}{0.97}

\usepackage{times}
%\usepackage{graphicx,enumerate}
\usepackage{graphicx}

% gives an error for some reason, hopefully not important
%\usepackage{enumitem}

%\usepackage{enumerate}
\usepackage{subfig}
%\usepackage{caption}
%\usepackage{subcaption}
\usepackage{amsmath,amssymb,amsfonts}
\usepackage{fancyhdr}
\usepackage{balance}
\usepackage{cite}
\usepackage{color}
\usepackage{wrapfig}
\usepackage[ruled,vlined,linesnumbered]{algorithm2e}
%\usepackage{booktabs}
\usepackage{booktabs,siunitx}
\usepackage{xcolor,colortbl}
\usepackage{times}
\usepackage{microtype}
\usepackage{url}
\usepackage{listings}
\usepackage{multirow}
%\usepackage{tablefootnote}

% \usepackage{hyperref}
% \hypersetup{
%     colorlinks=false,
%     pdfborder={0 0 0},
% }

%%\usepackage{abstract}

\makeatletter
\def\url@mystyle{%
  %\@ifundefined{selectfont}{\def\UrlFont{\sf}}{\def\UrlFont{\small\ttfamily}}}
  \@ifundefined{selectfont}{\def\UrlFont{\mathtt}}{\def\UrlFont{}}}
  %\@ifundefined{selectfont}{\def\UrlFont{\sf}}{\def\UrlFont{\small}}}
\makeatother
%\urlstyle{mystyle}
\urlstyle{rm}


\definecolor{mygreen}{rgb}{0,0.6,0}
\definecolor{mygray}{rgb}{0.5,0.5,0.5}
\definecolor{mymauve}{rgb}{0.58,0,0.82}

\newtheorem{example}{Example}[section]
\newtheorem{definition}{Definition}[section]
\newtheorem{assumption}{Assumption}[section]
\newtheorem{lemma}{Lemma}[section]
\newtheorem{theorem}{Theorem}[section]

%TODO: Fix line 738 of sig-alternate-05-2015.cls for the copyright
%notice to appear
\DeclareCaptionType{copyrightbox}

% Generic
%\DeclareMathAlphabet{\mathpzc}{OT1}{pzc}{m}{it}

% English
\newcommand{\ie}{{i.e.}\xspace}
\newcommand{\Ie}{{I.e.}\xspace}
\newcommand{\eg}{{e.g.}\xspace}
\newcommand{\etc}{{etc.}\xspace}
\newcommand{\viz}{{viz.\xspace}}
\newcommand{\etal}{{et al.}\xspace}

\renewcommand\vec[1]{\mathbf{#1}}
% Generic refs
\newcommand{\lemref}[1]{Lemma~\ref{lem:#1}}
\newcommand{\secref}[1]{Sec.~\ref{sec:#1}}
\newcommand{\figref}[1]{Fig.~\ref{fig:#1}}
\newcommand{\exref}[1]{Example~\ref{ex:#1}}
\newcommand{\thmref}[1]{Theorem~\ref{thm:#1}}
\newcommand{\tabref}[1]{Table~\ref{tab:#1}}
\newcommand{\algoref}[1]{Algorithm~\ref{algo:#1}}
\newcommand{\lstref}[1]{Listing~\ref{lst:#1}}
\newcommand{\chapref}[1]{Chapter~\ref{chap:#1}}
\newcommand{\defref}[1]{Definition~\ref{def:#1}}

% Comments, Reviewing, Formatting
\newcommand{\ignore}[1]{}
\newcommand\todo[1]{[[\textcolor{red}{\textsf{#1}}]]}
% General Math
\newcommand{\setof}[1]{\ensuremath{\{#1\}}}
\newcommand\tupleof[1]{\ensuremath\left\langle #1 \right \rangle}
\newcommand{\sublist}[2]{\ensuremath{#1_{1},\ldots,#1_{#2}}}
\newcommand{\suplist}[2]{\ensuremath{#1^{1},\ldots,#1^{#2}}}
\newcommand \card[1] {\left| #1 \right|}
\newcommand \floor[1] {\left\lfloor #1 \right\rfloor}
\newcommand \ceil[1] {\left\lceil #1 \right\rceil}

% Complexity
\newcommand{\npcomplete}{\textsc{Np}-\textsc{complete}}
\newcommand{\nphard}{\textsc{Np}-\textsc{Hard}}
\newcommand{\nlogspace}{\textsc{NLogspace}\xspace}
\newcommand{\pspace}{\textsc{PSpace}\xspace}
\newcommand{\expspace}{\textsc{ExpSpace}\xspace}

\newcommand{\mcl}[1]{\multicolumn{1}{l}{#1}}
\newcommand{\mcc}[1]{\multicolumn{1}{c}{#1}}

\newcommand{\mypara}[1]{\vspace{0.6em} \noindent{\bf #1}}
\newcommand{\myipara}[1]{\vspace{0.6em} {\em #1}\xspace}
%\newcommand{\myipara}[1]{\vspace{0.6em} \noindent{\em #1}\xspace}
\newcommand{\myiparaCompact}[1]{ {\em #1}\xspace}

% Sets
\newcommand{\Reals}{\ensuremath{\mathbb{R}}}
\newcommand{\reals}{\Reals}
\newcommand{\real}{\Reals}
\newcommand{\Nats}{\ensuremath{\mathbb{N}}}
\newcommand{\Integers}{\ensuremath{\mathbb{Z}}}
\newcommand{\Rationals}{\ensuremath{\mathbb{Q}}}

%%\newcommand{\mathsf}[1]{\mbox{\textsc{#1}}}
\newcommand{\mathsc}[1]{\mbox{\sc #1}}
\newcommand\scr[1]{\ensuremath\mathcal{#1}}
\newcommand\Gradient{\ensuremath\nabla}
\newcommand\pdiff[2]{\partial_{#1}{#2}}
\newcommand\jth[2]{ #1^{(#2)} }

% Logic
% \newcommand{\land}{\ensuremath\wedge}
% \newcommand{\lor}{\ensuremath\vee}

% Software
\newcommand\flowstar{FLOW*}
\newcommand\MATLAB{Matlab\textsuperscript{\textregistered}}
\newcommand\SIMULINK{Simulink\textsuperscript{\textregistered}}
\newcommand\STATEFLOW{Stateflow\textsuperscript{\textregistered}}
\newcommand\EMCODER{Embedded Coder\textsuperscript{\textregistered}}

%\newcommand\vs{\mathbf{s}}
%\newcommand\vw{\mathbf{w}}
%\newcommand\vx{\mathbf{x}}
%\newcommand\vy{\mathbf{y}}
%\newcommand\vz{\mathbf{z}}
%\newcommand\vu{\mathbf{u}}
%\newcommand\vj{\mathbf{j}}

\newcommand \vx {\vec{x}}
\newcommand \vy {\vec{y}}
\newcommand \vz {\vec{z}}
\newcommand \vu {\vec{u}}
\newcommand \vb{\vec{b}}

% long squiggily arow
\newcounter{sarrow}
\newcommand\xrsquigarrow[1]{%
\stepcounter{sarrow}%
\begin{tikzpicture}[decoration=snake]
\node (\thesarrow) {\strut#1};
\draw[->,decorate] (\thesarrow.south west) -- (\thesarrow.south east);
\end{tikzpicture}
}

% misc
\newcommand\ii{i+1}
\newcommand\denotation[1]{ \left\llbracket #1 \right\rrbracket}



% The \munepsfig command is used to insert a new EPS figure
% into our document.  Usage is:
%
%       \munepsfig[args]{filename}{caption}
%
% where:
%       - the optional 'args' argument is passed to the
%         embedded \includegraphics command, this can be used
%         to scale the figure or rotate it.
%       - 'filename' is the name of the EPS file in the 'figures'
%         directory that is to be inserted (note that 'filename'
%         should not include the '.eps' extension).
%       - 'filename' also serves as the label for the figure.
%         with the text 'fig:' prepended.
%
% Sample Usage:
%       \munepsfig[scale=0.5,angle=90]{barchart}{Population over time}

% inserts the EPS file 'figures/barchart.eps' reduced in size by 50%
% rotated 90 degrees and with the caption "Popuation over Time."
% We can refer to that figure as Figure~\ref{fig:barchart} in the text.
%
\newcommand{\inclfig}[4][scale=1.0]{%
        \begin{figure}[t]
                \centering
                \vspace{2mm}
%               \includegraphics[#1]{figures/#2.eps}
                \includegraphics[#1]{figs/#2}
                \caption{#3}
                \label{fig:#4}
        \end{figure}
}

% Units
%\newcommand{\degree}{^{\circ}}
\newcommand{\degreeC}{^{\circ}{\rm C}}
%\newcommand{\degreeF}{^{\circ}{\rm F}}

%\newcommand{\CC}{C\nolinebreak\hspace{-.05em}\raisebox{.4ex}{\tiny\bf +}\nolinebreak\hspace{-.10em}\raisebox{.4ex}{\tiny\bf +}}
\def\Cpp{{C\nolinebreak[4]\hspace{-.05em}\raisebox{.4ex}{\tiny\bf ++}}}
\def\CC{{C\nolinebreak[4]\hspace{-.05em}\raisebox{.4ex}{}}\;}



% RRT
\newcommand{\RRT}{\mathcal{RRT}}
\newcommand{\RRTgoal}{\x_{goal}}
\newcommand{\RRTinit}{\x_{init}}
\newcommand{\RRTv}[1]{v_{#1}}
\newcommand{\RRTe}{e}
\newcommand{\RRTV}{V}
\newcommand{\RRTE}{E}
\newcommand{\RRTsample}{x_{sample}}
\newcommand{\RRTnear}{x_{near}}
\newcommand{\RRTtoEx}{x_e}
\newcommand{\RRTxNew}{x_{new}}
\newcommand{\RRTuNew}{u_{new}}
\newcommand{\RRTdedge}[3]{(#1 \xrightarrow{#3} #2)}

% Graph
\newcommand{\Graph}{\mathbf{G}}
\newcommand{\graph}{\mathbf{G}}
\newcommand{\vertex}[1]{v_{#1}}
%\newcommand{\vertex}[1]{#1}
\newcommand{\dedge}[2]{e_{(\vertex{#1}\rightarrow\vertex{#2})}}
\newcommand{\edge}{e}
\newcommand{\vertexSet}{V}
\newcommand{\edgeSet}{E}
%\newcommand{\vertexSetD}{V^\delta}
%\newcommand{\edgeSetD}{E^\delta}
%\newcommand{\graphVE}[1]{\mathbf{G_{#1}(\vertexSet,\edgeSet)}}
%\newcommand{\graphD}[1]{\mathbf{G^\delta_{#1}(\vertexSet^\delta,\edgeSet^\delta)}}
\newcommand{\kPaths}{k\_paths}
\newcommand{\trajToNode}{\Gamma}
\newcommand{\Path}{\mathbf{P}}
\newcommand{\weight}{\mathbf{W}}

% Hybrid Automata
\newcommand{\HA}{\ensuremath{\mathcal{A}}}

\newcommand{\System}{\ensuremath{S}}
\newcommand{\Inputs}{\ensuremath{\mathcal{U}}}
\newcommand{\inputsignal}{\mathbf{u}}
\newcommand{\Flow}{\ensuremath{\mathcal{F}}}
\newcommand{\Modes}{\ensuremath{\mathcal{M}}}
\newcommand{\ContStates}{\ensuremath{X}}
\newcommand{\UnsafeStates}{\ensuremath{U}}
\newcommand{\InitStates}{\ensuremath{X}_\init}
\newcommand{\InitModes}{\ensuremath{M}_\init}
\newcommand{\Inv}{\ensuremath{\mathcal{I}}}
%\newcommand{\Transitions}{\ensuremath{\Delta}}
\newcommand{\Transitions}{\scr{T}}
\newcommand{\HybridStates}{\ensuremath{\mathcal{X}}}
\newcommand{\HybridStateSet}{\ensuremath{X}}
\newcommand{\discreteMode}{q}
\newcommand{\contState}{x}
\newcommand{\ResetMap}{\mathcal{R}}
\newcommand{\Guards}{\mathcal{G}}
\newcommand{\Init}{\mathcal{X}_{0}}
\newcommand{\Err}{\ContStates_{f}}
\newcommand{\initmode}{q_\init}
\newcommand{\cInit}{\ensuremath{X}_0}
\newcommand{\Unsafe}{\mathcal{X}_f}
%\newcommand{\reachSet}[2]{R^{#1}_{\HA}({#2})}
%\newcommand{\Hflow}{\mathcal{H}_{\HA}}
\newcommand{\Hflow}{\mathcal{H}}
\newcommand{\reachSet}{R}

% Trajectories
\newcommand{\discTraj}{q_h}
\newcommand{\hybridTraj}{\tau_h}
\newcommand{\traj}{\pi}
\newcommand{\trajSeg}[2]{\pi^{\mode_{#1}}_{\tau_{#2}}}
\newcommand{\SegTraj}{\mathbf{S}_\pi}
\newcommand{\dwell}{\tau}
\newcommand{\tran}{\delta}
\newcommand{\tbegin}{b}
\newcommand{\tend}{e}
%\newcommand{\graph}{G}
\newcommand{\trajStore}{TS}
\newcommand{\trajSet}{TS}
\newcommand{\candTraj}{CT}
\newcommand{\candTrajSet}{CTS}
\newcommand\Cost{\mathsc{Cost}}


% State variables
\newcommand{\x}{\mathbf{x}}
\newcommand{\y}{\mathbf{y}}
\newcommand{\z}{\mathbf{z}}
%\newcommand{\u}{\mathbf{x}}
\newcommand{\w}{\mathbf{w}}
%\newcommand{\vec}[1]{\mathbf{#1}}
\newcommand{\dvx}{\mathbf{\dot{x}}}
\newcommand{\inp}{\mathbf{u}}
\newcommand{\mode}{\ensuremath{q}}
\newcommand{\dx}{\ensuremath{\dot{x}}}

% Metrics
\newcommand{\lmetric}[2]{{\Vert #2 \Vert}_#1}

% ODE
\newcommand{\odesolution}{\Phi}
\newcommand{\matA}{\mathbf{A}}
\newcommand{\matB}{\mathbf{B}}
\newcommand{\mata}{\mathbf{a}}


\newcommand \maximize{\mathbf{max.}\ }
\newcommand \minimize{\mathbf{min.}\ }

\newcommand \numsolve{\mathtt{NumericSolver}}
\newcommand\Flowmap{\mathsc{Flow}}
%%\newcommand\myipara[1]{\par\noindent\textit{#1:}}

% Arrows
%\newcommand{\contArrow}[2]{\underset{#2}{\overset{#1}{\leadsto}}}
\newcommand{\contArrow}[1]{\leadsto_{#1}}
\newcommand{\jumpArrow}[1]{\xrightarrow{#1}}


\newcommand\dist{\mathsf{d}}
\newcommand\src{\mathsf{src}}
\newcommand\dest{\mathsf{dest}}
\newcommand\timeElapse{\mathcal{T}}
\newcommand\simulate{\mathsc{sim}_\Delta}
%%\newcommand\sim{\mathsc{sim}}
\newcommand\cost{\mathsc{cost}}
\newcommand\reach[1]{\xrightarrow{#1}}
\newcommand\areach[1]{\overset{#1}{\rightsquigarrow}}
%\newcommand\areach[1]{\xrsquigarrow{#1}}

%\newcommand\areach[2]{\overset{#1}{\underset{#2}\rightsquigarrow}}

\newcommand\crel[1]{\xrightarrow{#1}}
\newcommand\drel[1]{\overset{#1}{\rightsquigarrow}}
\newcommand\rel[1]{\overset{#1}{\rightsquigarrow}}


\newcommand\intr{\mathsf{interior}}
\newcommand\assign{:= }
\newcommand\worklist{\mathsf{workList}}
\newcommand\exploredCells{\mathsf{V}}
\newcommand\exploredEdges{\mathsf{E}}
\newcommand\unsafeCells{\mathsf{V}_u}
\newcommand\initialCells{\mathsf{V}_0}
\newcommand\prb{\mathbb{P}}
\newcommand{\abstracteps}{\epsilon}
\newcommand{\refinedeps}{\delta}
\newcommand{\absgraph}{\scr{H}_{\epsilon}(\Delta)}
\newcommand{\refgraph}{\scr{H}_{\delta}(\Delta)}
\newcommand{\abscells}{{\scr{C}}}
\newcommand{\refcells}{{\scr{D}}}






%%%%%%%% from Sriram sty, rearrange!

\def\mathsc#1{\mbox{\sc #1}}

\newcommand\trp[1] {#1^{\scriptscriptstyle T}}
\newcommand\scrS{\mathcal{S}}
\newcommand\scrT{\mathcal{T}}
\newcommand\scrF{\mathcal{F}}
\newcommand\scrG{\mathcal{G}}
\newcommand\scrH{\mathcal{H}}
\newcommand \ints {\ensuremath \mathbb{Z}}
%%\newcommand \extreals {\ensuremath \mathcal{R}^{+}}
\newcommand \lin[1]{\mathit{Lin}(#1)}
\newcommand \conic[1]{\mathit{Cone}(#1)}
\newcommand \conv[1]{\mathit{Convex}(#1)}
\newcommand \false {\mathit{false}}
\newcommand \true  {\mathit{true}}
%\newcommand\pre{\preceq}
\newcommand \pres{\mathbf{pres}}
\newcommand \F {\mathfrak{F}}

\newcommand \abstractF {\F_A}
\newcommand \abstractDomain {\Sigma_A}
\newcommand \templateDomain {\Sigma_T}
\newcommand \dropped {\mathsc{x}}
\newcommand \ctop {\vec{c}_\top}
\newcommand \cbot {\vec{c}_\bot}
\newcommand \concT {\gamma_T}
\newcommand \abstractorder {\leq_A}
\newcommand \abstractLeq{\sqsubseteq}
\newcommand \abstractor {\sqcup}
\newcommand \abstractand {\sqcap}
\newcommand \bigabstractor {\bigsqcup}
\newcommand \setdef[2] { \left\{ #1 \mid #2 \right\}}

\newcommand \CH {\mathcal{CH}}


%%\newcommand \mathsc[1]{\mbox{\textsc{#1}}}
%%transpose



\newcommand \T {\mathcal{T}}
\newcommand \post {\ensuremath\mathit{post}}
\newcommand \dpost {\widehat{\mathit{post}}}
\newcommand \homg[1] {\mathsc{hom}(#1)}
\newcommand \cons{\mathsc{cons}}
\newcommand \widen {\nabla}
\newcommand \dual[1]{\widehat{#1}}
\newcommand \narrow {\/ \bigtriangleup \/ }
\newcommand \refine {\partial}
\newcommand \deta{\pi}

\newcommand \latcap {\sqcap}
\newcommand \latcup {\sqcup}
%\newcommand \guard {\xi}
\newcommand \templ{\gamma}
\newcommand \sizeof[1] { |#1| }
\newcommand \uset[2]{\underset{#1}{\underbrace{#2}}}
\newcommand \ith[2]{ {#1}^{(#2)}}
\newcommand \Cells {\ensuremath \mathcal{C}}
\newcommand \InitCells {\ensuremath \mathcal{C}_\init}
\newcommand \CellEdges {\ensuremath \mathcal{E}}
\newcommand \CellLabelingFunction{\ensuremath \mathcal{L}_\Cells}
\newcommand \EdgeLabelingFunction{\ensuremath \mathcal{L}_\CellEdges}
\newcommand \exprs[1] {\mathsf{EXP}(#1)}
\newcommand \complex {\ensuremath \mathcal{C}}
\newcommand \grb {Gr\"obner\xspace}
\newcommand \hi[1]{}
\newcommand \hihenny[1]{}
\newcommand \newhenny[1]{#1}
\newcommand\markmargin[2]
{[\marginpar[\hfill \mbox{#1}$\rightarrow$]{$\leftarrow$\mbox{#1}} 
{\sf #2]}}

\newcommand\highl[1]{\psframebox[fillstyle=solid,fillcolor=lightgray,linewidth=0pt]{\textbf{#1}}}


\newcommand \locs{\mathbf{L}}
\newcommand\en{\mathbf{en}}




\newcommand{\chapterheading}[1]
{\vfill  
\hfill \fbox{\begin{minipage}{5.5in}
\textsl{#1} \end{minipage} } \newpage}

\newcommand \Z{ \mathbf{Z}}

\newcommand \mand {\ \mathit{and}\ }
\newcommand \matb{\vec{b}}
\newcommand \matl{\vec{\lambda}}
\newcommand \vg {\vec{g}}
\newcommand \va{\vec{a}}
\newcommand\ve{\vec{e}}
\newcommand \vc {\vec{c}}
\newcommand \vf {\vec{f}}
\newcommand \vh {\vec{h}}
\newcommand \vv{\vec{v}}
\newcommand \vzero {\vec{0}}
\newcommand \vq {\vec{q}}
\newcommand \rank {\mathit{rn}}
\newcommand\vbeta{\vec{\rho}}
\newcommand \Petri {\ensuremath \mathcal{P}}
\newcommand \maxrank {\mathit{maxrank}}
\newcommand \lists{\mathit{list}}
\newcommand \nullsp{\mathit{null}} 
\newcommand \vl[1] {\vec{\lambda_{#1}}}
\newcommand\vlam{\vec{\lambda}}
\newcommand \st{\mathbf{s.t.}\ }


\newcommand \D {\mathit{D}}
\newcommand \I {\mathit{I}}
\newcommand \Loc{\mathit{Loc}}
\newcommand \lie{\mathcal{L}}
\newcommand \grad{\nabla}
\newcommand \cn {\mathsf{Cn}}

%\newcommand \diff[2] {\ensuremath \frac{d #1}{d #2}}

\renewcommand\paragraph[1] {\smallskip\par\noindent\textbf{#1}\ \ }
\newcommand \cplus {\uplus}

\newcommand\relop{\mathop{\bowtie}}
\newcommand \vt{\vec{t}}
\newcommand \vd{\vec{d}}
\newcommand\e{\mathsf{e}}
\newcommand\yl{\mathsf{l}}
\newcommand\yu{\mathsf{u}}
\newcommand\yb{\mathsf{b}}


\newcommand\init{\mathsf{init}}
\newcommand\unsafe{\mathsf{unsafe}}

\newcommand\ol[1]{\overline{#1}}
% \newcommand\state[1]{\vec{{\texttt{#1}}}}

\newcommand\scrP{\mathcal{P}}
\newcommand\scrX{\mathcal{X}}
\newcommand\scrC{\mathcal{C}}
\newcommand\interVal[1]{[\underline{#1}, \overline{#1}]}
\newcommand\HIDE[1]{}


\newcommand{\HErr}{\mathcal{X}_{f}}
\newcommand\Eq{\mathsf{Eq}}
\newcommand\Ineq{\mathsf{Ineq}}
%\newcommand \setof[1] { \left\{ #1 \right \}}
%\newcommand \pdiff[2] { \ensuremath \frac{\partial #1}{\partial #2}}
%\newcommand \reach[1]{\mathsf{Reach}(#1)}


\newcommand{\vertexSetD}{V^\delta}
\newcommand{\edgeSetD}{E^\delta}
\newcommand{\graphVE}[1]{\mathbf{G_{#1}(\vertexSet,\edgeSet)}}
\newcommand{\graphD}[1]{\mathbf{G^\delta_{#1}(\vertexSet^\delta,\edgeSet^\delta)}}



% plant and controller

\newcommand\pgm{\rho}
\newcommand\Outputs{Y}
%\newcommand\locs{L}
%\newcommand\sloc{l_i}
%\newcommand\eloc{l_o}
%\newcommand\op{op}
%\newcommand\expr{E}


\newcommand\symMem{\mu}
\newcommand\CFG{\Pi}
\newcommand\pLocs{L}
\newcommand\pEdges{E}
\newcommand\pLabels{\Phi}
\newcommand\pEdge[1]{edge(#1)}
\newcommand\pVars{\mathcal{V}}
\newcommand\pVar{v}
\newcommand\pTransRel{\rho}
\newcommand\pLoc{l}
\newcommand\pLocI{l_0}
\newcommand\pLocF{l_f}
\newcommand\pVarI{V_0}
\newcommand\pPaths{\mathcal{P}}
\newcommand\pPath{p}
\newcommand\pPathCons{\kappa}
\newcommand\pCons{\xi}
\newcommand\pSymMem{\sigma}
\newcommand{\domain}{\mathcal{D}}
\newcommand{\pre}{\mathit{pre}}
\newcommand{\update}{\mathit{update}}
%TODO: redef?? fixit!
%\newcommand{\C}{\mathtt{C}}

\newcommand{\sampletime}{{\tau_{s}}}

\newcommand\pgmF{\rho}
\newcommand\pgmA{\hat{\rho}}
\newcommand\plantStates{x}
\newcommand\abstractPlantStates{X}
\newcommand\controllerOutputs{u}
\newcommand\controllerStates{s}
\newcommand\PathCons{\kappa}
\newcommand\constraints{\psi}
%TODO: redinfe path...conflicting with prev def
%\newcommand\Path{\mathcal{P}}
\newcommand\absState{\mathcal{A}}

\newcommand\ts{\tau_s}

\newcommand\Cix{C_i^x}
\newcommand\Ciy{C_i^y}
\newcommand\Ciix{C_{i+1}^x}
\newcommand\Ciiy{C_{i+1}^y}
% \newcommand\si{s_i}    %conflicts with package siuntix
\newcommand\sii{s_{i+1}}
\newcommand\ui{u_i}
\newcommand\uii{u_{i+1}}
\newcommand\sip[1]{s_{i}^{p_{#1}}}
\newcommand\uip[1]{u_{i}^{p_{#1}}}
\newcommand\siip[1]{s_{i+1}^{p_{#1}}}
\newcommand\uiip[1]{u_{i+1}^{p_{#1}}}
\newcommand\kpi{\kappa^{p_i}}


% \renewcommand{\Si}{S_i}
\newcommand\Si{S_i}
\newcommand\Sii{S_{i+1}}
\newcommand\Ui{U_i}
\newcommand\Uii{U_{i+1}}
\newcommand\Sip[1]{S_{i}^{p_{#1}}}
\newcommand\Uip[1]{U_{i}^{p_{#1}}}
\newcommand\Siip[1]{S_{i+1}^{p_{#1}}}
\newcommand\Uiip[1]{U_{i+1}^{p_{#1}}}

\newcommand \ctrl{\mathsc{ctrl}}
\newcommand\sample{\mathsc{Sample}}

\newcommand\TrajOpt{TrajOpt\;}

\newcommand{\cellequivalence}{\equiv_{\scr{C}}}
\newcommand{\vk}{\mathbf{k}}



\newcommand{\Traj}{\tau}
\newcommand{\cstates}{\scr{X}}
\newcommand{\csys}{S}
\newcommand{\cell}{C}
% \newcommand{\quant}{Q}
%\newcommand\quant{\mathsc{Quant}}

\newcommand\fmincon{\texttt{fmincon}\xspace}


\newcommand{\pwa}{\rho}
\newcommand{\map}{f}
\newcommand{\amap}{f}
\newcommand{\gm}{\scr{T}}
\newcommand{\guard}{g}
\newcommand{\ds}{\scr{D}}
\newcommand{\prop}{\scr{P}}

\newcommand{\simmap}{\mathsc{sim}}
% \newcommand{\initmode}{q_0}
\newcommand{\initstate}{x_0}

\newcommand{\quant}{h}

\newcommand{\TSAbstraction}[2]{\mathcal{T}^{#1}(#2)}
\newcommand{\RTSAbstraction}[2]{\mathcal{R}^{#1}(#2)}

\newcommand{\epsilonmin}{\epsilon_\mathrm{min}}

\newcommand \vzr {\mathbf{0}}
\newcommand \vxp {\vx^{+}}


%\clubpenalty=10000
%\widowpenalty = 10000





\title{\LARGE \bf Piece-Wise Relational Models for Falsification of Safety Properties in Hybrid Systems.}


\author{Aditya Zutshi$^{1}$, Sriram Sankaranarayanan$^{1}$, Sergio Mover$^{1}$ and Jyotirmoy V. Deshmukh$^{2}$% <-this % stops a space
\thanks{*This work was not supported by any organization}% <-this % stops a space
\thanks{$^{1}$Dept. of Computer Science, University of Colorado, Boulder \newline
        {\tt\small \{zutshi,srirams,sergio.mover\}@colorado.edu}}%
\thanks{$^{2}$Toyota Technical Center, LA \newline
        {\tt\small jyotirmoy.deshmukh@toyota.com}}%
}
%\thanks{$^{1}$Author is with Faculty of Electrical Engineering, Mathematics and Computer Science,
%        University of Twente, 7500 AE Enschede, The Netherlands
%        {\tt\small albert.author@papercept.net}}%
%\thanks{$^{2}$Bernard D. Researcheris with the Department of Electrical Engineering, Wright State University,
%        Dayton, OH 45435, USA
%        {\tt\small b.d.researcher@ieee.org}}%


\begin{document}



\maketitle
\thispagestyle{empty}
\pagestyle{empty}


%%%%%%%%%%%%%%%%%%%%%%%%%%%%%%%%%%%%%%%%%%%%%%%%%%%%%%%%%%%%%%%%%%%%%%%%%%%%%%%%
\begin{abstract}

Models of embedded control systems are often so complex that
falsification and testing approaches can scale only by numerically
simulating them, and hence treating the systems as black-boxes. Common
approaches like sampling-based falsification, use numerical
simulations to search for violations of safety properties. However,
this relegates them to the `best effort' category. On the other hand,
\emph{rigorous} verification (and falsification) techniques require
symbolic models and are comparatively less scalable. We try to bridge
this gap by incorporating data-driven models in a sampling based
search. Using an efficient, on-the-fly discrete abstraction
search procedure, we compute models only for selective regions of the state
space, thus `enriching' the abstraction. We then show how the
`enriched' abstraction can be used to guide the search for violations
without refinement.
%We use grid-based discrete abstractions to partition
%the state-space of the black-box hybrid dynamical system. We then use
%an efficient search procedure to find abstract violations of safety
%properties
%Assuming a grid-based implicit abstraction over the
%state-space of the black-box hybrid dynamical system, we enumerate
%abstract violations of the safety property. We then model the system
%behavior along these violations and search for concrete violations.
The automatic falsification proceeds by (a) randomly exploring the
state-space abstraction using numerically computed trajectories and
enumerating abstract violations, (b) modeling the behavior of the
system along these abstract violations by using discrete-time
piecewise affine (PWA) \emph{relational} models, and finally, (c)
encoding the search for a concrete violation as a bounded model
checking (BMC) problem.  The BMC query is then a disjunction of
conjunctions of linear constraints (i.e. combinations of linear
programs). It can be solved using off-the-shelf SMT solvers or by
multiple runs of a linear programming solver. For any counter-example
found in this `enriched' abstraction, we check the existence of a
corresponding violation in the original system. We demonstrate the
efficacy of our technique on a few benchmarks.

% Models of embedded control systems are often so complex that they can
% be treated as effectively black-box systems. Sampling-based methods
% for testing such systems use numerical simulations to search for
% violations of properties (specified in a suitable requirement
% formalism, such as temporal logic). This constrains them to a `best
% effort' category unlike rigorous verification techniques, which
% however require a model amenable to symbolic analysis. This work tries
% to bridge the gap by presenting an automatic falsification technique
% which learns a data-driven abstraction of the black box system before
% searching for a violation. It proceeds by (a) modeling the behavior of
% the system using a piecewise affine (PWA) discrete-time {\em
% relational} model and, (b) encoding the search for a violation in the
% abstract model as a bounded model checking problem. The bounded model
% checking query in this instance can be viewed as a disjunction of
% conjunctions of linear constraints (i.e. linear programs) and can be
% solved directly using off-the-shelf SMT solvers or by multiple runs of
% a linear programming solver. For any counter-example found in the
% abstraction, we check the existence of a corresponding violation in
% the original system. We demonstrate the efficacy of our technique on a
% few dynamical systems examples.








% =========================================================================
% Sampling based falsification techniques for black box hybrid dynamical
% systems use numerical simulations to search for violations of
% properties (specified in temporal logic). This constraints them to a
% `best effort' category unlike rigorous verification techniques, which
% however require a white box model. This work tries to bridge the gap
% by presenting an automatic falsification technique which models the
% black box system before searching for a violation. It proceeds by (a)
% modeling the behavior of the system using a piecewise affine (PWA)
% discrete time model and, (b) encoding the search for a violation as a
% standard bounded model checking problem. The resulting problem is a
% disjunction of linear programs and can be solved directly using
% off-the-shelf SMT solvers or by multiple runs of a linear programming
% solver. If a counter-example is found, the existence of a
% corresponding violation in the original system is checked for
% reproducibility.



% In this work we address the problem of finding unsafe behaviors with
% given a safety property for a hybrid dynamical system. We treat the
% system as a black box and approach the problem in two distinct steps.
% First, we model the behavior of the system using a piecewise affine
% discrete time model. Such a model is incomplete in nature and is
% computed with respect to the given property. Next, we encode the
% search for falsification as a bounded model checking query and use an
% SMT solver to find a counter example. If found, we check the existence
% of the violation in the original system to ensure reproducibility.

%Time bounded LTL property in
%Models from verification
%model without data
%black box model/data -> structured model


\end{abstract}

%%%%%%%%%%%%%%%%%%%%%%%%%%%%%%
\section{Introduction}
\label{sec:intro}

% Testing of hybrid systems is a complex task, due to the interaction
% between discrete and continuous dynamics. Industrially developed
% hybrid systems, with embedded software interacting with physical
% components, are typically safety-critical. Thus, it is important to
% guarantee that the systems adhere to functional and safety
% specifications. Since the reachability problem for hybrid systems is
% known to in general be undecidable, exhaustive exploration of the
% state-space is intractable, and thus testing is an approach to attempt
% to achieve system correctness. However, to guarantee safety and other
% requirements, the testing procedures need to rely on a rigorous
% theoretical foundation. In this special session aims to collect and
% connect researchers, within academia as well as industry, working on
% model-based testing of hybrid systems. The session calls for papers
% proposing new theoretical research directions in the field, as well as
% papers describing practical applications of model-based testing of
% hybrid systems.


Model-based development is rapidly becoming the preferred paradigm for
developing embedded control software in an industrial context.  Tools
like \SIMULINK are often used to create models of closed-loop
dynamical systems, \ie, a plant model of the physical systems that we
wish to control, and a model for the embedded controllers.  Automatic
approaches for testing the safety and performance of such closed-loop
systems is challenging, and current approaches focus mostly on guided
testing using numerical simulations
\cite{annpureddy2011s,donze2010breach,deshmukh2015stochastic,dreossi2015efficient,akazaki}.
On the other hand, theoretical research on hybrid and timed systems
focusses on formal guarantees for restricted classes of dynamical
systems such as hybrid automata where the system state evolves
according to a vector field that is either constant, affine
\cite{frehse2011spaceex}, or polynomial \cite{chen2015reachability}.
In this paper, we propose an approach that leverages useful features
from both kinds of techniques.

We propose a two step process for finding violations of safety
properties. In the first step, we learn a piece-wise affine (PWA) {\em
relational} model from simulations of the dynamical system. In the
second step, we leverage symbolic techniques to exhaustively analyze
the model learned in the first phase. We remark that the approach we
propose is neither sound nor complete: a violation found by our
technique may not exist in the original system, and our technique may
miss violations that exist in the original system. A fair question is:
why is such a technique useful? There are two perspectives from which
we can answer this question.

In many settings, we may not have a physics-based model derived from
first principles, but may just have time-series data of the system
behavior. In such a setting, a data-driven model allows us to
generalize individual time-series behaviors. But, there are several
approaches in the literature to perform data-driven modeling; e.g.,
system identification techniques that learn dynamic models from data
\cite{ljung1999system}, auto-regressive models for forecasting
\cite{wei1994time}, machine learning techniques that learn static or
dynamic models from data
\cite{narendra1990identification,lu2009linear}.  Why then do we need a
new modeling technique? The answer to this question lies in our goal
for the second step, \ie, symbolic analysis of the learned model. For
this step, we need to have models that can be digested by existing
symbolic tools, which is harder to do with some of the aforementioned
heavier data-driven modeling methods. Hence, we use a lighter flavor
of data-driven modeling such as the one proposed for learning
relational models from data
\cite{zutshi2012timed,sankaranarayanan2011relational}. By learning a
{\em PWA relational model}, \ie, essentially a PWA discrete-time
dynamical system, we can use software analysis approaches such as
bounded model checking assisted by SMT solvers and linear programming
solvers.

The other perspective for our technique is the context when we have an
effectively black-box model, \ie, a model that has features such as
nonlinearities, delays or look-up tables that make symbolic analysis
rather challenging.  We note that the ultimate goal for our technique
is to find violations of safety properties in the {\em actual embedded
control system}, and not, per se, in the model. The aphorism
attributed to mathematician George E.P. Box is quite apt for our
context, and bears worth repeating,  ``all models are wrong, some
models are useful.'' We can sidestep analyzing a highly complex model
by approximating its behavior with a simpler relational model, which
in turn can give us promising tests to run on the actual system. This
gives our technique intrinsic value as a debugging tool.

% In our previous work~\cite{zutshi2014multiple}, we analyzed systems by
% restricting ourselves to their black box semantics. This enabled us to
% reason about the state-space reachability of the system without a
% direct analysis of its structure. Using a coarse abstraction which we
% searched on-the-fly, we could observe the local dynamics as required.
% This gave us an efficient procedure to find abstract counter-examples.
% However, to concretize the counter-examples, a `closer look' or
% refinement of the abstraction is required. The grid based state-space
% abstraction was refined by splitting the relevant cells (abstract
% states) into smaller ones. As noted previously, due to the curse of
% dimensionality, such a uniform splitting is an expensive operation,
% and can not scale to higher dimensions.

In \cite{zutshi2014multiple}, the authors investigate an on-the-fly
abstraction technique that is close to the approach we propose. A key
difference is that the abstraction explored in this work is an
existential {\em relational} abstraction. The approximate model
learned is a directed acyclic graph, where nodes represent regions of
the state-space, and an edge between two regions indicates a
transition between states in the regions. Our approach further
improves the existential abstraction by annotating the edges in the
graph-based abstraction with affine relations between the source and
destination states.

% We further our exploration of trajectory segment based methods; and
% explore an alternative approach to overcome the explosion in abstract
% states. Instead of selectively refining the abstraction, we compute a
% model of the black box system and use bounded model checking to find a
% concrete counter-example in the model. Due to modeling errors, this
% might not be reproducible in the original black-box system. 


More specifically, we use linear regression to quantitatively estimate
the discovered edges in the graph-based abstraction of the
state-space. These linear maps approximate locally observed behaviors,
and the resulting graph or the Piece-Wise Affine (PWA) relational
model can be interpreted as an infinite state discrete transition
system and model checked for time bounded safety properties. A
counter-example if found, can indicate the presence of a violation in
the system.  We then use this counter-example to guide the search
towards a counter-example in the original system.


%Furthermore, using linear programming, we can over-approximate the
%neighbhourhood of the counter-example in the model.

% In the conclusion, we discuss extensions to data driven
% approaches (instead of simulator driven), where instead of a
% simulator, a fixed set of data is provided as the behavioral
% description of the system. Using a combination of relational modeling,
% and program analysis we outline the future work for falsifying
% properties of SDCS.
% Bounded time behvaiors are modeled
% only one violation is being searched for
% mention explicitly the relation between older cegar work

The layout of the paper is as follows: In Section~\ref{sec:prelims},
we introduce the basic steps to obtain a PWA relational model. In
Section~\ref{sec:pwa-rel}, we explore the connection between the PWA
relational model and previous graph-based abstraction techniques such
as those in \cite{zutshi2014multiple}. We present extensions of
relational modeling in Section~\ref{sec:rel-mod}, and present
experimental validation of our technique on a few textbook examples of
dynamical systems with hybrid and polynomial dynamics in
Section~\ref{sec:res}.

%%%%%%%%%%%%%%%%%%%%%%%%%%%%%%

% %%%%%%%%%%%%%%%%%%%%%%%%%%%%%%
% %\subsection{Motivation}
% %\label{sec:mot}
% %\input{motivation.tex}
% %%%%%%%%%%%%%%%%%%%%%%%%%%%%%%

%%%%%%%%%%%%%%%%%%%%%%%%%%%%%%
\section{Related Work}
\label{sec:rel}
Piece-Wise Affine (PWA) models are quite popular in the modeling of
both continuous and hybrid dynamical systems. They are very
expressive, and can model non-linear continuous dynamics with
arbitrary accuracy~\cite{wen2008basis} and a large family of hybrid
systems~\cite{heemels2001equivalence}.

\mypara{Falsification of Properties of Hybrid Systems.} The current
state-of-the-art falsification approaches are based on numerical
simulations and can be grouped into robustness guided
S-Taliro~\cite{annpureddy2011s}, Breach~\cite{donze2010breach} and RRT
(Rapidly-exploring Random Tree)
based~\cite{nahhal_test_2007,Dang09,dreossi2015efficient}.  An
alternative approach, based on multiple shooting and CEGAR
(Counterexample Guided Abstraction Refinement) was proposed by the
authors in ~\cite{zutshi2014multiple}.  The alternative approach is
the verification approach, where a formal model (usually hybrid
automata) is first constructed (often manually) and then exhaustively
checked for violations.

\mypara{Identification of hybrid systems} using Piece-Wise affine
models is surveyed in~\cite{paoletti2007identification}.
\begin{enumerate}
\item Expressiveness/power of PWA models?
\item Fixed time caveats
\item Effects of refining parameters: time step $\Delta$ and grid size $\epsilon$.
\end{enumerate}

%\mypara{Model Checking PWA Models.} Model checking transition systems

%%%%%%%%%%%%%%%%%%%%%%%%%%%%%%

%%%%%%%%%%%%%%%%%%%%%%%%%%%%%%
\section{Prelims}
\label{sec:prelims}
In this section, we present the basics of relational abstractions and
fundamentals of regression analysis.

\subsection{Relational Abstraction}

For general hybrid dynamical systems, algorithmic ways for computing
analytical solutions do not exist. Discretization (along with suitable
abstractions) is often employed to transform the systems into a
discrete transition system. Relational abstractions is one such idea,
which abstracts continuous dynamics by discrete relations using
appropriate \textit{reachability invariants}. Both time independent
and time dependent relations have been proposed; the former captures
all reachable states over all time, whereas, the latter explicitly
includes time by relating reachable states to time. Thus it
can prove timing properties whereas, time independent relational
abstractions can not. In both cases, the resulting abstractions can be
interpreted as discrete transition systems and can be analyzed using
model checkers. We briefly discuss these two types of abstractions.

\mypara{Timeless Relational Abstraction}

Relational abstractions for hybrid systems were proposed in
\cite{Sankaranarayanan+Tiwari/2011/Relational}. For a hybrid automaton
model, they can summarize the continuous dynamics of each mode using a
binary relation over continuous states. The resulting relations are
timeless (independent) and hence valid for all time as long as
the mode invariant is satisfied. The relations take the general form
of $R(\x,\x') \bowtie 0$, where $\bowtie$ represents one of the
relational operators $=, \ge, \le, <, >$. For example, an abstraction
that captures the monotonicity with respect to time for the
differential equation $\dot{x} = 2$ is $x' > x$. The abstraction
capturing the relation between the set of ODEs: $\dot{x} = 2$,
$\dot{y} = 5$, is $5(x' - x) = 2(y' - y)$.

Such relation summaries, discretize the continuous evolution of a
given system into an infinite state transition system, in which safety
properties can be verified using $k$-induction, or falsified using
bounded model checkers. The relationalization procedure involves
finding suitable invariants in a chosen abstract domain, such as
affine abstractions, eigen abstractions or box abstractions
\cite{Sankaranarayanan+Tiwari/2011/Relational}.  For this, different
techniques like template based invariant generation
\cite{Gulwani+Tiwari/2008/Constraint,
Colon+Sankaranarayanan+Sipma/03/Linear} can be used.

\mypara{Timed and Time-Aware Relational Abstractions}

As the above discussed relations are timeless, we cannot reason over
the timing properties of the original system. Moreover, they are not
suitable for time-triggered systems. Timed relational
abstractions~\cite{zutshi2012timed} and time-aware relational
abstractions~\cite{mover2013time} were proposed to overcome these
shortcomings by explicitly including time in the relations. In
addition, both can be more precise than their timeless counterpart,
but unlike it, assume continuous affine dynamics.

To analyze time-triggered systems like an SDCS modeled by an affine
hybrid automata~\footnote{An affine hybrid automata is restricted to
the ODEs, resets and guards being affine.}, timed relational
abstractions can be computed directly from the solutions of the
underlying affine ODEs.  Recall that the solution of an affine ODE
$\dot{\x} = A\x$ can be represented using the matrix exponential as
$\x(t) = e^{tA}\x(0)$. This gives us the timed relation $\x' =
e^{tA}\x$.  Clearly, the relation is non-linear with respect to time.
However, for the case of SDCS with a fixed sampling time period, we
obtain linear relations in $\x$.  We demonstrated the usefulness of
timed relational abstractions for the case of linear systems
in \cite{zutshi2012timed}, and now incorporate the idea to discretize
black box dynamical systems.

Similar to timeless relational abstractions, time-aware relational
abstractions \cite{mover2013time} construct binary relations between
the current state $\x$ of the system and any future reachable state
$\x'$. The difference lies with the latter also constructing relations
between the current time $t$ and any future time $t'$. This is
achieved by a case by case analysis of the eigen structure of the
matrix $A$ (of an affine ODE). Separate abstractions are used for the
case of linear systems with constant rate, real eigen values and complex eigenvalues.

\subsection{Learning Dynamics using Simple Linear Regression}

\mypara{Regression Analysis}

In statistics, regression is the problem of finding a
\textit{predictor}, which can suitably predict the relationship
between the given set of observed input $\x$ and output $\y$ vectors.
In other words, assuming that $\y$ depends on $\x$, regression
strategies find either a parameterized or a non-parameterized
prediction function to explain the dependence. We now discuss simple
linear regression, which is parametric in nature and searches for an
affine predictor. It is also called \textit{ordinary least squares}.

\mypara{Ordinary Least Squares (OLS)}

Let the data set be comprised of $N$ input and output pairs $(\x,
y)$, where $\x\in\reals^n$ and $y\in\reals$. If $N>n$, which is
the case in the current context of \textit{finding the best fit}, the
problem is over-determined; there are more equations than
unknowns. Hence, a single affine function cannot be found which
satisfies the equation $\forall i\in\{1\ldots N\}. y_i = \vec{a}^T\x_i + b$. Instead, we
need to find the `best' choice for $\vec{a}$ and $b$. This is formally
defined using a loss function. For the case of simple linear
regression or OLS, the loss function is the sum of squares of the
errors in prediction.  The task is then to determine the vector of
coefficients $\vec{a}$ and an offset constant $b$, such that the least
square error of the affine predictor is minimized for the given data
set.
\begin{equation}
    \operatornamewithlimits{argmin}_{\vec{a}, b}\displaystyle\sum_{i=1}^{N}{\left(\y_i - (\vec{a}^T\x_i + b)\right)^2}
\label{reg}
\end{equation}
To ease the presentation, we can rewrite the above as a homogeneous
expression by augmenting $\x$ by $\hat{\x} = \begin{bmatrix}\x \\ 1\end{bmatrix}$ and replacing $\vec{a}$ and $b$ by the vector
$\vec{ab} = \begin{bmatrix}\vec{a} \\ b \end{bmatrix}$. The equation~\ref{reg} now becomes
\[
    \operatornamewithlimits{argmin}_{\vec{ab}}\displaystyle\sum_{i=1}^{N}{(\y_i - \vec{ab}^T \hat{\x}_i)^2}
\]
The solution of OLS can be analytically computed as
\[ Ab = (X^TX)^{-1}X^T\y\]
where $X$ is the matrix representing the horizontal stacking of all
$\hat{\x}$.
The details can be found in several texts on learning and statistics
\cite{friedman2001elements}.

Given a time invariant dynamical system $\x' = \simulate(\x, \Delta)$,
where $\x\in\reals^n$ we can use OLS to approximate its trajectories
at fixed time step $\Delta$ by a discrete map $\x' = A\x + \vb$, where
$A = \begin{bmatrix}\vec{a}_1^T \\ \vdots \\ \vec{a}_n^T \end{bmatrix}$ and $\vb =
\begin{bmatrix}b_1 \\ \vdots \\ b_n \end{bmatrix}$. This map is a
    global relational model for the system. The data set for the
    regression is a set of pairs $\setof{(\x,\x') | \x' =
    \simulate(\x, \Delta)}$ and can be generated on demand. Another
    set can be generated to compute the error $\delta$ as an interval
    of vectors, where each element is an interval $\delta_i \in
    [\delta_{min}, \delta_{max}]$.
    
    Affine maps are poor
    approximations for arbitrary non-linear functions. Hence, we use a
    collection of affine maps to approximate the local behaviors (in
    state-space) of the system $\simulate$. This results in a piece
    wise approximation, as described in the next section.

\subsection{Piecewise Affine (PWA) Transition System}

We define the PWA transition system as a transition system
(ref.~\secref{DiscSys}) $\rho:\tupleof{\pLocs,\pVars,\scrT, \pLocI,\Theta}$
where each transition $\tau \in \scrT$, is associated with a
transition relation of the form
\[
    \rho_{\tau}(\pVar,\pVar') \subseteq \setof{(\pVar,f_{\tau}(\pVar)) \;|\; g_\tau(\pVar) \land g'_\tau(f_{\tau}(\pVar))}
\]
where $f_{\tau}$ is an affine map relating the states and $g_{\tau}$,
$g'_{\tau}$ are affine guards (pre and post conditions) on the states.
The assertion on the initial states $\Theta$ is also affine.

We can now use a PWA transition system $\rho$ to approximate the
behavior of a hybrid dynamical system $\System$ defined by a simulator
$\simulate_{\System}$~\footnote{Due to $f_\tau$ being restricted to an affine
form, we can only approximate general hybrid systems.}. Each
transition of $\rho$ represents a discrete time step $\Delta$. Let the
state-space of $\System$ be given by $\x\in\HybridStates^n$, then the
states of $\rho$ are also given by $\x\in\HybridStates^n$. The affine guard
predicates $g_\tau$ are given by a conjunction of hyper-planes and
hence can be represented in matrix form as a polyhedron (defined by
$m$ constraints) in the state-space
\[
g:C\x - \vd \le 0
\]
where $C$ is an $m \times n$ matrix and $\vd$ is a vector of length
$m$. The affine maps $f_\tau$ can be represented in matrix form as
$f:\x \mapsto A\x+\vb$. To make $f_\tau$ `richer', we incorporate
an error term $\delta$ which is defined by a vector of intervals
$[\delta_{\min},\delta_{\max}]$. Hence,
\[
f:\x \mapsto A\x + \vb + \delta
\]
defines a set valued relation of the form $R:\setof{(\x,\x') \;|\; \x'
\in f_\tau(\x)}$.


% Given a state vector $\x\in\reals^n$, a guarded affine map
% $\gm:(\guard, \amap)$ defines the discretized consecution rule as a
% pair of an affine guard predicate $\guard:C\x - \vd \le 0$ and an affine map
% $\amap:A\x+\vb$, where $A \in \reals^{n \times n}$, $C \in
% \reals^{m \times n}$ are matrices and $\vb \in \reals^n$, $\vd
% \in \reals^m$ are vectors. A guarded affine map is satisfied if its
% guard is satisfied.

We now formalize the PWA affine transition system for a dynamical system.
\begin{definition}[PWA Transition System]

Given a hybrid dynamical system over a state-space
$\HybridStates^n$, a PWA transition system $\rho$ is given by the
tuple $\tupleof{\pLocs,\x,\scrT,\pLocI,\Theta}$, where $\tau \in
\scrT$ are affine transition relations and $\Theta$ is an affine
predicate over $\x$ and the states $\x\in\HybridStates^n$. The
transition relation is then defined by $\scrT$ with $n$ transition
relations as follows

\begin{equation}
    \scrT = \left\{
        \begin{array}{ll}
            g_1(\x) \land g'_1(\x') \implies \x' \in f_1(\x) \\
            \ldots\\
            g_n(\x) \land g'_n(\x') \implies \x' \in f_n(\x) \\
        \end{array}
    \right.
\end{equation}

\end{definition}

where $f_i:\x \mapsto A_i\x + \vb_i + \delta_i$,
$g_i(\x):C_i\x-\vd_i\le0$, $g'_i(\x'):C'_i\x'-\vd'_i\le0$ and $\x'$
denotes the next state of the system.

A PWA model is \textit{deterministic}, iff for every state
$\x\in\HybridStates^n$, a unique guarded affine map is satisfied and
all errors $\delta$ are singletons. However, in practice such a case
rarely exists. A PWA model will be usually non-deterministic, both due
to multiple choice of transition relations and the reachable states at
the $k^{th}$ time step (or after $k\Delta$ units of time) being a set.
Abusing the notation, we denote the set of states reachable in a
single step in $\rho$ by $\x' = \rho(\x)$.

A PWA transition system is \textit{complete}, iff for every state
$\x\in\HybridStates^n$, there exists at least one satisfied
transition relation $\scrT$. If this is not the case, the system can
deadlock, with no further executions. This usually results from
incorrect modeling of SDCS, and from here on, we do not consider such
cases.

% Given a dynamical system over a continuous state-space
% $\ContStates\in\reals^n$, a PWA model is a map $\ContStates \mapsto
% \setof{\gm_1, \gm_2, \ldots, \gm_n}$ from the state-space of the
% dynamical system to a finite set of guarded affine maps
% $\setof{\gm_1, \gm_2, \ldots, \gm_n}$. It defines the consecution
% rule by the affine map of the satisfied $\gm$.

% \begin{equation}
%     \pwa = \left\{
%         \begin{array}{ll}
%             \gm_1: \amap_1(\x) &\text{if}\; \guard_1(\x) \\
%             \gm_2: \amap_2(\x) &\text{if}\; \guard_2(\x) \\
%             \ldots & \ldots\\
%             \gm_n: \amap_n(\x) &\text{if}\; \guard_n(\x) \\
%         \end{array}
%     \right.
% \end{equation}

% \end{definition}

% where $\amap_i = A_i\x + \vb_i$, and $\guard_i(\x) \equiv C_i\x-\vd_i\le0$.
% Abusing the notation, we denote the next state computed by the PWA
% model as $\x' = \pwa(\x)$. A PWA model is \textit{deterministic}, iff
% for every state $\x\in\ContStates$, a unique guarded affine map is
% satisfied. A PWA model is \textit{complete}, iff for every state
% $\x\in\ContStates$, there exists at least one satisfied guarded affine
% map $\gm$.

%%%%%%%%%%%%%%%%%%%%%%%%%%%%%%

%%%%%%%%%%%%%%%%%%%%%%%%%%%%%%
\section{PWA Relational Modeling}
\label{sec:pwa-rel}

A discrete abstraction over a continuous space uses relations to
describe the behavior of the system over time. Such relations abstract
the underlying relation between concrete states, and concrete paths
can no longer be constructed directly. Instead, search procedueres
like Counter-Example Guided Refinement must be used to find concrete
paths. In general, search techniques are exponential with respect to
computational resources and memory. As an alternative, we introduce
data driven `enrichment' of abstractions to approximate the concrete
relations underlying the abstract relations. Informally, we use linear
regression to compute affine maps associated with the relations
between two abstract states. We then show how associating the computed
maps with the relations transforms a discrete abstraction into a PWA
transitions system.

\subsection{Graph Abstraction}

The abstraction of the hybrid dynamical system is defined by
paritioning the combiend state-input space $\HybridStates \times
\Inputs$ into hyper-rectangles or cells $C_i \in \C$. The cells are
pairwise disjoint $C_i \cap C_j \neq \emptyset$ iff $i \neq j$ and
their union $\bigcup_{C_i\in\C}C_i$ is the state-input
space $\HybridStates \times \Inputs$. This discrete abstraction over
the continuous space is implicit and defined using a quantization
function $Q_\epsilon$ parameterized by the precision $\epsilon$.
This further defines an equivalence relations such that
\[
    (\vx_i, \vu_i) \equiv (\vx_j, \vu_j) \;\text{iff}\; Q(\vx_i,
    \vu_i) = Q(\vx_j, \vu_j)
\]
In other words, $(\vx_i, \vu_i) \in C_i \; \text{iff} \; C_i =
Q(\vx_i, \vu_i)$.

\begin{definition}[] Let C be a d-quantized tiling. The d-quantized
    tiling-based plant abstraction (denoted Pd) is given by a graph
    (V, E), where V = C, and the set of edges E is defined such that
    (Cj, Ck) ∈ E iff: there is a state-input pair (x, u) ∈ Cj and a
state-input pair (x 0 , u) ∈ Ck such that x 0 = SIM(x, u, τ ).
\end{definition}


Given a numerical simulator $\simulate$ and an initial abstraction
defined by a quantization function $\quant_\epsilon$ and a time step
$\Delta$, we pick samples and simulate them to explore the abstraction
on the fly to obtain a graph $G$, which has a finite number of
abstract states $C$ (or cells) and edges $(C,C')$ iff $C
\areach{\Delta} C'$.


\begin{definition}[Abstract State Graph]. Let ∆ > 0 be a fixed time
    step. The abstract state graph H(∆) for time step ∆ is defined by
    the set of vertices C, and edges (Ci, Cj ) whenever Ci ∆ Cj . Let
    C0 denote the collection of initial cells in C, i.e., cells Ci
    such that Ci ∩ X0 6= ∅. Further, let Cu denote the set of unsafe
    cells, or cells Cj such that there is a state xj ∈ Cj that reaches
    an unsafe state y ∈ Xf within time 0 ≤ tj < ∆. The abstraction
H(∆) is given by D C, ∆ , C0, Cu E .  \end{definition}



\subsection{Enriched Abstraction}


\begin{figure}[!htbp]
\begin{center}
\tikzstyle{line} = [thick]
\tikzstyle{arw} = [->, thick,>=stealth,shorten <=2pt, shorten >=2pt]
\begin{tikzpicture}
\begin{scope}[scale=1.2]

    \draw [line] (-5.2,-1.2) rectangle (-3.8,0.2);
    \draw [line] (-1.7,-0.2) rectangle (-0.3,1.2);
    \draw [line] (-4.1,-0.7) .. controls +(2.0,2.0) and +(-1.6,-1.5) ..  (-1.2,0.8);
    \draw [line] (-4.2,-0.6) .. controls +(2.1,2.1) and +(-1.5,-1.4) ..  (-1.2,0.9);
    \draw [line] (-4.3,-0.5) .. controls +(2.2,2.2) and +(-1.4,-1.3) ..  (-1.3,0.95);

\node at (-4.5, -0.5) {$C$};
\node at (-1, 0.5) {$C'$};
\node at (-4.2,-0.8) {\scriptsize{$\x_i$}};
\node at (-1.0,0.90) {\scriptsize{$\x_i'$}};
\node at (-3,-0.1) {$\pi_i$};
\node at (-2.8, -1.5) {(a)};
\end{scope}

\begin{scope}[xshift=4.0cm,scale=1.2]
\draw [line] (-2.0,0) circle (0.5);
\draw [line] (2.0,0) circle (0.5);
\draw[arw] (-1.5,0) -- (1.5,0);
\node at (-2.0,0) {$C$};
\node at (2.0,0) {$C'$};
\node at (0,0.5) {\scriptsize{$R_{(C,C')}:\setof{\x' \in A\x + \vb + \delta}$}};
\node at (0.,-1.5) {(b)};
\end{scope}
\end{tikzpicture}
\end{center}
\vspace*{-.3cm}
\caption{(a) Trajectory segments $\traj_i$ are used to compute the
relation $R_{(C,C')}$ that annotates the edge in (b).
$R_{(C,C')}:\setof{\x' \in A\x + \vb + \delta}$ is an interval affine
relation defined by an affine map (matrix $A$ and vector $\vb$) and an
error interval (vector of intervals $\delta$).}
    \label{fig:enriched-edge}
\vspace*{-.3cm}
 \end{figure}


Recall that each edge $(C,C')$ of the graph $G$ denotes an observed
trajectory segment between the respective cells. The graph abstraction
$G$ only states that there exists a state $\x \in C$ from which the
system can evolve to a future state $\x' \in C'$. To increase the
precision we iteratively refined the abstraction by state
splitting. Instead, we now propose an `enrichment' $G^R$ of $G$ by computing a
set of local relations $R_{(C,C')}(\x,\x')$ for every edge $(C,C')$,
which non-deterministically describe relations between $\x \in C$ and $\x'
\in C'$. This is illustrated in \figref{enriched-edge}.
%(compare with \figref{segtraj}).

The enriched graph $G^R$ captures the underlying local forward
dynamics describing the evolution of the system in each abstract
state. We represent the dynamics using an affine model with an
interval error. Such a model can either be approximated using learning
methods or computed as a sound (over) approximation using reachability
set computation methods. Because we assume black box semantics, we
only present the former. Using regression analysis on the $start$ and
$end$ states of the witnessed trajectory segments between two cells,
we compute an approximate discrete map along with an error estimate.
Moreover, using the simulation function $\simulate$, additional
trajectory segments (or data) can be generated if required. The data
can be separated into a training set and testing set to compute the
map and the error respectively.  The latter case can be explored if
the symbolic dynamics of the system are known. A tool like
\flowstar~\cite{chen2013flow} can be used to find the reachable set
map.

Observe that $G^R$, a directed reachability graph, is rich enough to
search for concrete behaviors in the system. We call it a time
parameterized PWA relational abstraction. It can be interpreted as an
infinite state discrete transition system, and we can use
off-the-shelf bounded model checkers to find concrete violations of a
given safety property, and even other temporal properties.  We now
discuss the background required to present our ideas.



Instead of using a CEGAR like loop (used in~\cite{zutshi2014multiple}),
we use the generated trajectory segments to learn quantitative models
describing the local behavior of the system. These models are defined
by a set of relations $R \subseteq \HybridStates \times \HybridStates$
for each edge of the reachability graph.

%%%%%%%%%%%%%%%%%%%%%%%%%%%%%%

%%%%%%%%%%%%%%%%%%%%%%%%%%%%%%
\section{Bounded Model Checking for Black Box Systems}
\label{sec:bmc}
\section{Model Checking}

As a final step, the enriched graph is model checked for the given
safety property. If a counter-example is found, it is simulated using
the $\simulate$ function and if reproducible, a violation is
concluded. In all other cases the falsification results in an
`unknown' result.

% % include an algorithm
% We now describe the entire procedure in three steps.
% \begin{enumerate}
%     \item Given a black box system $\System$ specified by
%         $\simulate^\System$, we
%         use scatter-and-simulate with ($\Delta$-length) trajectory
%         segments to sample a finite reachability graph
%         $\scr{H}(\Delta)$.
%     \item Using regression, we enrich the graph relations and annotate
%         the edges with the estimated affine maps.
%     \item Interpreting the directed reachability graph as a transition
%         system, we use a bounded model checker to find fixed length
%         violations to safety properties.
% \end{enumerate}

\section{An Example: Van der Pol Oscillator}

We now describe the pesented technique using the Van der Pol
oscillator benchmark from~\cite{zutshi2014multiple}. Simulations
generated using uniformly random samples are shown in
\figref{vdp-cont}. We want to check the property $P_3$, indicated by
the red box, given the initial set indicated by the green box. The
abstraction is defined by $\quant_{0.2}(\x)$, which results in an
evenly gridded state-space, where each cell is of size $0.2 \times
0.2$ units.  Scatter-and-simulate is then used to construct the
abstract graph $G$, using $2$ samples per cell and the time step
$\Delta = 0.1s$. The complete process follows.

\begin{enumerate}
    \item{\emph{Abstract}}: Assume an implicit abstraction using the quantization
        function $\quant_{\epsilon=0.2}(\x)$. The corresponding reachability
    graph is $\scr{H}(\Delta=0.1)$.
\item{\emph{Discover}}: Using scatter-and-simulate, enumerate a finite number
    of cells (vertices) and the relations (edges) of the graph
    $\scr{H}(0.1)$. Associate the set of generated trajectory
    segments with their respective originating cells using a map $\ds:C
        \mapsto \setof{(\vx,\vx') |\vx \in C}$. The discovered
    abstraction is show in \figref{vdp-abs}. The red cells
    are unsafe and green cells are initial cells.
\item{\emph{Relationalize}}: For each cell $C$, perform regression analysis on
    the respective set of trajectory segments $D(C)$, and compute a
    set of affine relations $R_{C,C'}(\x,\x')$ between $\x \in C$ and $\x'
    \in C'$ s.t., $\text{edge} (C,C') \in \scr{H}(0.1)$. Annotate
    each edge in the graph $\scr{H}(0.1)$ with the respective
    relation.

    \figref{vdp-map} shows the cells and the trajectory segments which
    are part of the data sets constructed using the $1$-relational
    modeling. Against each cell, its unique identifier (integer
    co-ordinate) is mentioned.  Finally, \figref{vdp-graph} and
    \tabref{vdp-pwa} show the enriched graph $G^R$ with its transition
    relations. Note that self loops result when an observed trajectory
    segment has its $start$ and $end$ states in the same cell.

\item{\emph{Model Check}}: The graph $\scr{H}(0.1)$ can now be viewed as a
    transition system $\tupleof{C,\x,\scrT,C_0,\HybridStates_0}$,
    where $C \in vertices(\scr{H}(0.1))$ and $\x \in
    \HybridStates$. A transition $\tau \in \scrT$ is of the form
    $\tupleof{C, C',\rho_{\tau}}$, where $\rho_{\tau}(\x,\x')
    \subseteq \setof{R_{(C,C')}(\x,\x') |( C,C') \in
    edge(\scr{H}(0.1))}$.
\item{\emph{Check Counter-example}}: The infinite state (but finite location) transition system can
    be model checked to find a concrete counter-example, which if
    exists, can indicate the existence of a similar trace in the
    original black-box system $\System$. The latter check is carried
    out as before, using the numerical simulation function
    $\simulate$. For the given example, the model checker is unable to
    find a counter-example.
\end{enumerate}

\inclfig[width=0.95\linewidth]{vdp_cont_traces.pdf}{Van der
Pol: continuous trajectories. Red and green boxes indicate unsafe and
initial sets.}
\label{fig:vdp-cont}

\inclfig[width=0.95\linewidth]{vdp_abs.pdf}{The discovered
abstraction $\scr{H}(0.1)$. Red cells are unsafe cells and green cells are
initial cells.}
\label{fig:vdp-abs}

\inclfig[width=0.90\linewidth]{vdp_disc_map.pdf}{Cells and
trajectory segments used by $1$-relational modeling.}
\label{fig:vdp-map}

\inclfig[width=0.50\linewidth]{vdp_graph.pdf}{Enriched
graph $G^R$. The affine maps $f$ for the transition relations are show
in \tabref{vdp-pwa}.}
\label{fig:vdp-graph}

\begin{table}[h]
\centering
\caption{PWA model computed using OLS. The affine model for each edge $(C,C')$ in the graph \figref{vdp-graph} is given by $x'\in Ax + b + \delta$, where $\delta$ is a vector of intervals.}
\label{tab:vdp-pwa}
\begin{tabular}{@{}ccccc@{}}
\toprule
$C$ & $C'$ & $A$ & $b$ & $\delta$\\
\midrule
(0, -1)   & (0, -1)   & $\begin{bmatrix}0.99& 0.12 \\-0.12&1.63\end{bmatrix}$&$\begin{bmatrix}0\\0\end{bmatrix}  $&$\begin{bmatrix}[0, 0]\\ [0, 0]\end{bmatrix}$\\
(0, -1)   & (0, -2)   & $\begin{bmatrix}0.99& 0.12 \\-0.10&1.63\end{bmatrix}$&$\begin{bmatrix}0\\0\end{bmatrix}  $&$\begin{bmatrix}[0, 0]\\ [0, 0]\end{bmatrix}$\\
\midrule
(0, -2)   & (0, -2)   & $\begin{bmatrix}0.99& 0.12 \\-0.09&1.63\end{bmatrix}$&$\begin{bmatrix}0\\0\end{bmatrix}  $&$\begin{bmatrix}[0, 0]\\ [0, 0]\end{bmatrix}$\\
(0, -2)   & (0, -4)   & $\begin{bmatrix}0.99& 0.12 \\-0.07&1.62\end{bmatrix}$&$\begin{bmatrix}0\\0\end{bmatrix}  $&$\begin{bmatrix}[0, 0]\\ [0, 0]\end{bmatrix}$\\
\midrule
(0, -9)   & (-1, -15) & $\begin{bmatrix}0.99& 0.12 \\-0.09&1.61\end{bmatrix}$&$\begin{bmatrix}0\\-0.04\end{bmatrix} $&$\begin{bmatrix}[0, 0]\\ [0, 0.01]\end{bmatrix}$\\
\midrule
(-1, -15) & (-3, -24) & $\begin{bmatrix}0.95& 0.12 \\-1.29&1.47\end{bmatrix}$&$\begin{bmatrix}-0.01\\-0.41\end{bmatrix}$&$\begin{bmatrix}[0, 0]\\ [0, 0.01]\end{bmatrix}$\\
\midrule
(-3, -24) & (-4, -29) & $\begin{bmatrix}1.11&-0.17 \\-1.71&0.66\end{bmatrix}$&$\begin{bmatrix}-1.07\\-3.31\end{bmatrix}$&$\begin{bmatrix}[-0.04, 0.04]\\ [-0.07, 0.09]\end{bmatrix}$\\
\midrule
(-4, -29) & (-4, -29) & $\begin{bmatrix}0.99& 0.01 \\-0.41&1.02\end{bmatrix}$&$\begin{bmatrix}0\\-0.28\end{bmatrix} $&$\begin{bmatrix}[0, 0]\\ [0, 0]\end{bmatrix}$\\
\midrule
(-1, -10) & (-2, -16) & $\begin{bmatrix}0.97& 0.12 \\-0.71&1.56\end{bmatrix}$&$\begin{bmatrix}0\\-0.11\end{bmatrix} $&$\begin{bmatrix}[0, 0]\\ [0, 0.01]\end{bmatrix}$\\
(-1, -10) & (-2, -15) & $\begin{bmatrix}0.97& 0.12 \\-0.70&1.58\end{bmatrix}$&$\begin{bmatrix}0\\-0.08\end{bmatrix} $&$\begin{bmatrix}[0, 0]\\ [0, 0.01]\end{bmatrix}$\\
\midrule
(-2, -16) & (-3, -23) & $\begin{bmatrix}0.92& 0.12 \\-1.84&1.39\end{bmatrix}$&$\begin{bmatrix}-0.02\\-0.73\end{bmatrix}$&$\begin{bmatrix}[0, 0]\\ [0, 0]\end{bmatrix}$\\
\midrule
(-3, -23) & (-5, -29) & $\begin{bmatrix}3.14&-0.81 \\-0.96&0.23\end{bmatrix}$&$\begin{bmatrix}-3.13\\-4.99\end{bmatrix}$&$\begin{bmatrix}[-0.04, 0.03]\\ [-0.03, 0.04]\end{bmatrix}$\\
\midrule
(-2, -15) & (-3, -22) & $\begin{bmatrix}0.93& 0.12 \\-1.75&1.40\end{bmatrix}$&$\begin{bmatrix}-0.02\\-0.68\end{bmatrix}$&$\begin{bmatrix}[0, 0]\\ [0, 0]\end{bmatrix}$\\
\midrule
(-3, -22) & (-5, -29) & $\begin{bmatrix}3.08&-0.72 \\-1.69&0.40\end{bmatrix}$&$\begin{bmatrix}-2.77\\-4.57\end{bmatrix}$&$\begin{bmatrix}[-0.02, 0.02]\\ [-0.01, 0.02]\end{bmatrix}$\\
\bottomrule
\end{tabular}
\end{table}



% (0, -1)   & (0, -1)   &  \begin{bmatrix}0.9944498 & 0.12911959 \\-0.12013468&1.63250531\end{bmatrix} &\begin{bmatrix}-0.00003683\\-0.00085411\end{bmatrix}&\begin{bmatrix}[-0.0, 0.0]\\ [-0.0, 0.0]\end{bmatrix}\\
% (0, -1)   & (0, -2)   &  \begin{bmatrix}0.99523928& 0.1290447  \\-0.10236736&1.6314103 \end{bmatrix} &\begin{bmatrix}-0.00013019\\-0.00289192\end{bmatrix}&\begin{bmatrix}[-0.0, 0.0]\\ [-0.0, 0.0]\end{bmatrix}\\
% \midrule
% (0, -2)   & (0, -2)   &  \begin{bmatrix}0.99565348& 0.12916002 \\-0.09352462&1.6341424 \end{bmatrix} &\begin{bmatrix}-0.00015574\\-0.00339664\end{bmatrix}&\begin{bmatrix}[-0.0, 0.0]\\ [-0.0, 0.0]\end{bmatrix}\\
% (0, -2)   & (0, -4)   &  \begin{bmatrix}0.99668599& 0.12892455 \\-0.0725035 &1.62937613\end{bmatrix} &\begin{bmatrix}-0.00037391\\-0.00796189\end{bmatrix}&\begin{bmatrix}[-0.0, 0.0]\\ [-0.0, 0.0]\end{bmatrix}\\
% \midrule
% (0, -9)   & (-1, -15) &  \begin{bmatrix}0.99985362& 0.12858788 \\-0.09447884&1.61476417\end{bmatrix} &\begin{bmatrix}-0.0018997 \\-0.04200363\end{bmatrix}&\begin{bmatrix}[-0.0, 0.0]\\ [-0.0, 0.01]\end{bmatrix}\\
% \midrule
% (-1, -15) & (-3, -24) &  \begin{bmatrix}0.95473217& 0.12473489 \\-1.29198505&1.47155614\end{bmatrix} &\begin{bmatrix}-0.01248412\\-0.41100129\end{bmatrix}&\begin{bmatrix}[-0.0, 0.0]\\ [-0.0, 0.01]\end{bmatrix}\\
% \midrule
% (-3, -24) & (-4, -29) &  \begin{bmatrix}1.11789749&-0.17261206 \\-1.71017779&0.66006597\end{bmatrix} &\begin{bmatrix}-1.0745344 \\-3.31246863\end{bmatrix}&\begin{bmatrix}[-0.04, 0.04]\\ [-0.07, 0.09]\end{bmatrix}\\
% \midrule
% (-4, -29) & (-4, -29) &  \begin{bmatrix}0.99796588& 0.01012817 \\-0.41529747&1.02432211\end{bmatrix} &\begin{bmatrix}-0.00135199\\-0.28016983\end{bmatrix}&\begin{bmatrix}[-0.0, 0.0]\\ [-0.0, 0.0]\end{bmatrix}\\
% \midrule
% (-1, -10) & (-2, -16) &  \begin{bmatrix}0.97381953& 0.12768367 \\-0.71242741&1.56764585\end{bmatrix} &\begin{bmatrix}-0.00320265\\-0.11964102\end{bmatrix}&\begin{bmatrix}[-0.0, 0.0]\\ [-0.0, 0.01]\end{bmatrix}\\
% (-1, -10) & (-2, -15) &  \begin{bmatrix}0.97373058& 0.12859163 \\-0.70173855&1.58540872\end{bmatrix} &\begin{bmatrix}-0.0015415 \\-0.08616376\end{bmatrix}&\begin{bmatrix}[-0.0, 0.0]\\ [-0.0, 0.01]\end{bmatrix}\\
% \midrule
% (-2, -16) & (-3, -23) &  \begin{bmatrix}0.92850123& 0.12242812 \\-1.84720528&1.39159826\end{bmatrix} &\begin{bmatrix}-0.02334236\\-0.73348865\end{bmatrix}&\begin{bmatrix}[-0.0, 0.0]\\ [-0.0, 0.0]\end{bmatrix}\\
% \midrule
% (-3, -23) & (-5, -29) &  \begin{bmatrix}3.14947585&-0.81150646 \\-0.96159406&0.23543254\end{bmatrix} &\begin{bmatrix}-3.13860326\\-4.99696623\end{bmatrix}&\begin{bmatrix}[-0.04, 0.03]\\ [-0.03, 0.04]\end{bmatrix}\\
% \midrule
% (-2, -15) & (-3, -22) &  \begin{bmatrix}0.93156623& 0.12260264 \\-1.75117219&1.40104348\end{bmatrix} &\begin{bmatrix}-0.02214125\\-0.68423569\end{bmatrix}&\begin{bmatrix}[-0.0, 0.0]\\ [-0.0, 0.0]\end{bmatrix}\\
% \midrule
% (-3, -22) & (-5, -29) &  \begin{bmatrix}3.08806462&-0.72332776 \\-1.69271079&0.40308293\end{bmatrix} &\begin{bmatrix}-2.77908467\\-4.57411339\end{bmatrix}&\begin{bmatrix}[-0.02, 0.02]\\ [-0.01, 0.02]\end{bmatrix}\\






\subsection{Search Parameters}

The search parameters for S3CAM-R include the parameters of S3CAM:
$N$, $\epsilon$ and $\Delta$. Additionally, they also include the
maximum error budget $\delta_{max}$ for OLS and the maximum length of
segmented trajectory for building $k$-relational models.

We have already discussed the effects of $N$, $\epsilon$ and $\Delta$
on S3CAM's performance. However, they also have an effect on
relational modeling. A finer grid with small cells produces more
accurate models than a coarser grid with large cells. Similarly,
small time length trajectory segments result in more accurate
modeling.
%An increase in the number of samples can increases the
%accuracy of regression.

\paragraph{Maximum Model Error ($\boldsymbol{\delta}_{\bold{max}}$) .}
Given a $\delta_{max}$, we keep increasing $k$ during the
$k$-relational modeling process till $\delta_i \le \delta_{max}$ is
satisfied. A $k_{max}$ is introduced to bound the longest
segmented trajectory which can be considered.

% \paragraph{Bounded k relations ($\boldsymbol{k}_{\bold{max}}$.) }

\subsection{Reasons for Failure}
The approach can fail to find a counter-example, even when it exists, in
one of three ways.

\begin{itemize}

    \item \emph{S3CAM Fails} No abstract counter-example is found by
        S3CAM. We can remedy it by increasing search budgets and/or
        restarting.

    \item \emph{BMC Fails} An abstract counter-example is found, but
        the BMC fails to find a concrete counter-example in the PWA
        relational model. The failure can be attributed to either (a)
        a spurious abstract counter-examples or (b) a poorly estimated
        model. The former can be addressed by restarting but the
        latter requires that the maximum model error $\delta_{max}$ be
        decreased.

    \item \emph{Inaccurate PWA Modeling} An abstract counter-example
        is found, and it is successfully concretized in the PWA
        relational model by the BMC. However, it is not reproducible
        in the black box system. This happens when the PWA relational
        model is not precise enough.

\end{itemize}

%%%%%%%%%%%%%%%%%%%%%%%%%%%%%%


%%%%%%%%%%%%%%%%%%%%%%%%%%%%%%
\section{Implementation}
\label{sec:impl}

The implementation was prototyped as S3CAM-R, an extension to our
previously mentioned tool S3CAM (\chapref{case}). OLS regression
routines were used from Scikit-learn~\cite{pedregosa2011scikit}, a
Python module for machine learning. SAL~\cite{SAL-SRI}
with Yices2~\cite{dutertre2014yices} was the model checker and the SMT
solver.

We used S3CAM to discover the abstraction graph $G$, which was then
trimmed of the nodes which did not contribute to the error search
(nodes from which error nodes were not reachable). In our experiments we
used $1$-relational modeling to create the transition system from $G$.
Using SAL, we checked for the given safety property. If a concrete
trace was obtained from the BMC, it was further checked for validity
by simulating using $\simulate$ to see if it was indeed an error
trace.  If not, $100$ samples from its associated abstract state was
sampled to check the presence of a counter-example in it
neighborhood. This can be further extended by obtaining different
discrete sequences or discrete trace sequences from the model checker.

We tabulate our preliminary evaluation in \tabref{res-rel}. We used
few of the benchmarks described in \chapref{case}, including the Van
der Pol oscillator, Brusselator, Lorenz attractor and the Navigation
benchmark. As before, we ran S3CAM-R $10$ times with different seeds
and averaged the results. We tabulate both the total time taken and
the time taken by SAL to compute the counter-example and compare
against a newer implementation of S3CAM. Note that different
abstraction parameters are used for S3CAM and S3CAM-R, due to the
differences in which they operate.

\begin{table*}[!htbp]
\centering
\caption{Avg. timings for benchmarks. The \textbf{BMC} column lists
time taken by the BMC engine. The total time in seconds (rounded off
to an integer) is noted under \textbf{S3CAM-R} and \textbf{S3CAM}.
$TO$ signifies time $>5hr$, after which the search was killed.}
\label{tab:res-rel}
\begin{tabular}{@{}llll@{}}
\toprule
Benchmark & BMC & S3CAM-R & S3CAM\\
\midrule
Van der Pol ($\prop_3$)   &$<$1 & $130$ & $24$\\
Brusselator               &$<$1  & $4$    & $2$\\
Lorenz                    &$<$1 & $122$  & $35$\\
%B.Ball                    &  &   & \\
Nav ($\prop_P$)           &$4480$ &$9423$   &$603$ \\
Nav ($\prop_Q$)           &$970$  &$8105$   &$546$ \\
Nav ($\prop_R$)           &-  &$TO$   & $2003$\\
Nav ($\prop_S$)           &-  &$TO$   & $2100$\\
Bouncing Ball             &-  &$TO$   & $450$\\

\bottomrule
\end{tabular}
\end{table*}

% 1m8s, 2m56.184s, 1m31.059s[f], 1m32.665s,  1m45.312s, 1m22.397s, 2m1.451s[f], 1m12.118s, 1m26.596s
% 0.12, 0.06,         0.33       0.13,       0.22,     FAIL,       ,X           0.08        0.15

The results show promise, but clearly S3CAM performs better. Also,
S3CAM-R timed out on some benchmarks like Nav $\prop_R$, Nav $\prop_S$
and the bouncing ball. However, we need to explore more benchmarks to
be conclusive. S3CAM-R's performance can be explained by the
difficulty in finding a good abstraction ($\quant_\epsilon$), which
needs to be fine enough, to obtain good prediction models but coarse
enough, to be manageable by the current SMT solvers. Specifically, to
obtain a good quality counter-example from the model checker (and
avoid false positives), the transition system should be created from
models with high accuracy, for which a finer abstraction is required.
However, the exploration of a finer abstraction is exponentially more
complex and results in a much bigger transition system, which the
current state of the art bounded model checkers (which use SMT
solvers) can not handle under reasonable resources.

%%%%%%%%%%%%%%%%%%%%%%%%%%%%%%

% %%%%%%%%%%%%%%%%%%%%%%%%%%%%%%
% \section{Experimental Results}
% \label{sec:res}
% \input{results.tex}
% %%%%%%%%%%%%%%%%%%%%%%%%%%%%%%

%%%%%%%%%%%%%%%%%%%%%%%%%%%%%%
\section{Conclusions}
\label{sec:concl}

We have presented a methodology to find falsifications in black box
dynamical systems. Combining the ideas from abstraction based search
\cite{zutshi2014multiple}, with relational
abstractions~\cite{sankaranarayanan2011relational}, we proposed
$k$-relational modeling to compute richer abstractions. Simple linear
regression can be used to estimate the local dynamics of the
trajectory segments and compute PWA relational models which can be
interpreted as a transition system.  These can then be checked for
safety violations using an off-the-shelf bounded model checker.
Finally, we implemented the ideas as a tool PWA-Rel and demonstrated
its performance on a few examples.

%\todo{memory required: pre-simulated data}
%\todo{extension to LTL/MTL props}

% \subsection{Improvements}

% The approach seems promising as it avoids an expensive refinement
% step. But, before it can be applied to more complex systems, we need
% to overcome its shortcomings.  As a future extension, we are
% investigating ideas to improve the performance of the underlying three
% main sub-processes, namely, abstraction, modeling, and BMC. We detail
% these below.

% \begin{enumerate}

% \item \emph{Abstraction} The abstract domain we have used, has been
%     confined to the interval or rectangular domain. Such a domain is
%     very coarse when compared to more general polyhedral domains, using
%     which, we can gain better precision. Due to polyhedral domains
%     being computational expensive, their integration in S3CAM needs to
%     be explored thoroughly.

% \item \emph{Modeling} We have used $k$-relational modeling to increase
%     the precision of the enriched abstractions, however, there are
%     several other refinement strategies that remain to be explored.
%     This includes refining the abstract states by using guard
%     predicates (similar to predicate abstraction), using non-linear
%     templates in parametric regression to compute more precise
%     quadratic relations, and using an adaptive time step instead of a
%     fixed one to address the non-linearities due to a long time
%     horizon.

% \item \emph{Model Checking} As of now, we use SMT solvers to search
%     for concrete counter-examples. Although they are very efficient,
%     owing to extensive engineering effort, reachability of transition
%     systems resulting from dynamical systems remains a difficult
%     problem. As the time horizon of the safety property increases, the
%     possible combinations of discrete transitions increase
%     exponentially. Hence, to find a counterexample which is a sequence
%     of discrete transitions over a `long' time horizon is not
%     tractable for most, but the simplest of dynamical systems.
%     Instead, we can use linear programming solvers, by enumerating
%     each path in the graph $G^R$. Note that the constraints are all
%     linear (conjunctions) along an abstract path. However, in the
%     worst case, the number of paths in a graph can be of the order
%     $n!$ where $n$ is the number of vertices of a graph. Hence, this
%     will not be feasible, unless, we can prioritize paths by using a
%     triage process similar to one used by S3CAM and using a budget on
%     the maximum number of paths.

%     Another approach to address the issue would be to use an adaptive
%     time discretization technique, where relations over both shorter
%     and longer time steps are computed, and the SMT solver can
%     `select' the time step precise enough to find a counter-example.

% \item \emph{SMT Solvers} One major impediment to our approach is the
%     fact that SMT solvers use the theory of reals with \emph{exact
%     precision}. This is important for verification approaches, but can
%     be relaxed for the problem of falsification. An SMT solver which
%     uses approximate reasoning but returns robust counter-examples
%     will be as useful, and perhaps more efficient.

% \end{enumerate}

% \subsection{Data Driven Analysis}

% Finally, we would like to mention that our approach is `simulation'
% driven. It can be easily modified to be `data driven', by working with
% a fixed set of data. Given a data set, we then need to automatically
% find a good abstraction and a high fidelity PWA transition system
% using which, we can summarize the black box hybrid dynamical system.
% Apart from model checking the transition system, one can extend it to
% the analysis of SDCS. More specifically, by combining the transition
% system model of a plant with the control software, a model checker
% like CBMC \cite{kroening2014cbmc} can be used to do a closed loop
% symbolic analysis of the SDCS. This can be an alternative to S3CAM-X.

%which will be
%very useful to for analyzing data for potential property violations.

%%%%%%%%%%%%%%%%%%%%%%%%%%%%%%

\bibliographystyle{abbrv}
\bibliography{refs}


\addtolength{\textheight}{-12cm}   % This command serves to balance the column lengths
                                  % on the last page of the document manually. It shortens
                                  % the textheight of the last page by a suitable amount.
                                  % This command does not take effect until the next page
                                  % so it should come on the page before the last. Make
                                  % sure that you do not shorten the textheight too much.


% \section*{ACKNOWLEDGMENT}

% The preferred spelling of the word �acknowledgment� in America is without an �e� after the �g�. Avoid the stilted expression, �One of us (R. B. G.) thanks . . .�  Instead, try �R. B. G. thanks�. Put sponsor acknowledgments in the unnumbered footnote on the first page.




\end{document}
